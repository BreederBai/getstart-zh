%%
%% preface.tex -- Flight Gear documentation: The FlightGear Manual
%% Chapter file
%%
%% Written by Michael Basler, started September 1998.
%%
%% Copyright (C) 2002 Michael Basler
%%
%%
%% This program is free software; you can redistribute it and/or
%% modify it under the terms of the GNU General Public License as
%% published by the Free Software Foundation; either version 2 of the
%% License, or (at your option) any later version.
%%
%% This program is distributed in the hope that it will be useful, but
%% WITHOUT ANY WARRANTY; without even the implied warranty of
%% MERCHANTABILITY or FITNESS FOR A PARTICULAR PURPOSE.  See the GNU
%% General Public License for more details.
%%
%% You should have received a copy of the GNU General Public License
%% along with this program; if not, write to the Free Software
%% Foundation, Inc., 675 Mass Ave, Cambridge, MA 02139, USA.
%%
%% $Id: preface.tex,v 0.6 2002/09/09 michael
%% (Log is kept at end of this file)

%%%%%%%%%%%%%%%%%%%%%%%%%%%%%%%%%%%%%%%%%%%%%%%%%%%%%%%%%%%%%%%%%%%%%%%%%%%%%%%%%%%%%%%%%%%%%%%
\IfLanguageName{french}{
\chapter{Pr\'{e}face}
}{}
\IfLanguageName{italian}{
\chapter{Prefazione}
}{}
\ifchinese
  \chapter{{\\}序言}
\fi
\label{preface}
%%%%%%%%%%%%%%%%%%%%%%%%%%%%%%%%%%%%%%%%%%%%%%%%%%%%%%%%%%%%%%%%%%%%%%%%%%%%%%%%%%%%%%%%%%%%%%%
\ifchinese
\FlightGear{} 是一个自由的飞行模拟器,由互联网上一群对飞行模拟和编程狂热爱好的人开发。这本《FlightGear 手册》希望介绍给初学者如何快速上手 \FlightGear{},并享受飞行的乐趣。这里不会提供介绍 \FlightGear{} 全部特性和插件的文档,然而会让新手了解如何开始和探索 \FlightGear{} 所提供的各种体验。 
\fi

\IfLanguageName{french}{
\FlightGear{} est un simulateur de vol libre et gratuit d\'{e}velopp\'{e} grace \`{a} Internet
par une communaut\'{e} de passionn\'{e}s de simulation de vol et de programmation. "Le
manuel FlightGear" a pour but d'offrir aux d\'{e}butants un guide \`{a} l'installation
et \`{a} l'utilisation de \FlightGear{} et aux premi\`{e}res heures de vol. Il n'a pas pour
vocation de fournir une documentation compl\`{e}te de toutes les fonctionnalit\'{e}s et
ajouts de \FlightGear{} mais, plut\^{o}t, d'apporter au nouvel utilisateur les
meilleures bases pour lui permettre d'explorer ce que \FlightGear{} a \`{a} offrir.
}{}

\IfLanguageName{italian}{
\FlightGear{} \`{e} un simulatore di volo gratuito sviluppato in modo cooperativo su internet da un
gruppo appassionati di simulatori di volo e di programmazione.  Il "Manuale di FlightGear" \`{e}
destinato a dare ai principianti una guida per ottenere FlightGear installato e funzionante sul
proprio computer, e il piacere di volare virtualmente da casa propria. Non \`{e} destinato a fornire
una documentazione completa di tutte le caratteristiche e componenti aggiuntivi del simulatore,
ma, invece, mira a far comprendere ad un nuovo utente ci\`{o} che esso ha di meglio da offrire.
}{}

\ifchinese
此版本的文档是为 \FlightGear{} \version{} 所写。使用早期版本 \FlightGear{} 的用户也能在此文档中找到有用的信息,然而很多文档中提到的新特性,可能无法使用。

此文档分成三个部分,各部分的内容和结构罗列如下。
\fi

\IfLanguageName{french}{
Cette version du manuel a \'{e}t\'{e} \'{e}crite pour la version \version{} de \FlightGear{}.
Les utilisateurs de versions pr\'{e}c\'{e}dentes de \FlightGear{} trouveront toujours
une utilit\'{e} \`{a} ce document, mais certaines des fonctionnalit\'{e}s qui y sont d\'{e}crites
peuvent \^{e}tre absentes de leur version.

Ce guide est scind\'{e} en trois parties et structur\'{e} de la fa\c{c}on suivante :
}{}

\IfLanguageName{italian}{
Questa versione del manuale \`{e} stata scritta per FlightGear versione \version{}. Gli utenti che
usano versioni precedenti di FlightGear potranno ancora trovare questo documento utile, ma
alcune delle caratteristiche descritte potrebbero non essere presenti.

Questa guida \`{e} divisa in tre parti ed \`{e} strutturata come segue.
}{}

\medskip

\noindent
\ifchinese
\textbf{第一部分:安装}
\fi
\IfLanguageName{french}{
\textbf{Partie I : Installation}
}{}
\IfLanguageName{italian}{
\textbf{Parte I: Installazione}
}{}
\medskip

 \noindent
\ifchinese
第~\ref{free} 章,\textit{想要自由的飞翔?来玩 \FlightGear{} 吧},介绍 \FlightGear{},并讲解其中的背景知识和哲学,还有系统要求。
\fi
\IfLanguageName{french}{
Le chapitre~\ref{free}, \textit{Vous voulez voler librement ? Choisissez \FlightGear{}} ! pr\'{e}sente
\FlightGear{}, la philosophie qui porte le projet et d\'{e}crit les pr\'{e}-requis syst\`{e}me.
}{}
\IfLanguageName{italian}{
Capitolo~\ref{free}, \textit{Vuoi avere un simulatore di volo libero? Usa \FlightGear{}!}: Fornisce la
filosofia di base che c'\`{e} dietro al programma e descrive i requisiti di sistema
}{}
 \medskip

 \noindent
\ifchinese
第~\ref{prefligh} 章,\textit{飞行前的准备:安装 \FlightGear{}},这里将提供如何安装二进制发行版\index{binary distribution 二进制发行版}和附加地景及飞行器的介绍。
\fi
\IfLanguageName{french}{
Dans le chapitre ~\ref{prefligh}, \textit{Pr\'{e}vol : installer \FlightGear{}}, vous trouverez les
instructions n\'{e}cessaires \`{a} l'installation des binaires\index{binary distribution} et des
sc\`{e}nes et avions additionnels.
}{}
\IfLanguageName{italian}{
Capitolo~\ref{prefligh}, \textit{Prima di volare: installazione di \FlightGear{}}: Fornisce le istruzioni
per installare il gioco e paesaggi/aerei
}{}
 \medskip

\noindent
\ifchinese
\textbf{第二部分:在 \FlightGear{} 中飞行}
\fi
\IfLanguageName{french}{
\textbf{Partie II : Voler avec \FlightGear{}}
}{}
\IfLanguageName{italian}{
\textbf{Parte II: Volare con FlightGear}
}{}
\medskip

 \noindent
\ifchinese
 第~\ref{takeoff} 章,\textit{起飞:如何启动程序},描述了如何实际启动已经安装的程序。
 还包括概述一些命令行参数和配置文件。
\fi
\IfLanguageName{french}{
  Le chapitre~\ref{takeoff}, \textit{D\'{e}collage : comment d\'{e}marrer le
  programme}, d\'{e}crit comment lancer le programme une fois installi\'{e}.
  Il comprend un survol des nombreuses options en ligne de commande ainsi
  que des fichiers de configuration.
}{}
\IfLanguageName{italian}{
Capitolo~\ref{takeoff}, \textit{Inizio: come avviare il programma}:
Descrive come avviare il programma. Comprende una panoramica sulle numerose
opzioni della riga di comando e sui file di configurazione
}{}
 \medskip

 \noindent
\ifchinese
第~\ref{flight} 章,\textit{飞行中:关于仪表、键位和菜单},描述了如何操作这个程序,
比如如何在 \FlightGear{} 中实际飞行。这里包括了一份完整的(希望是)预
定义键盘命令列表,一个菜单项概览,还详细讲述了仪表板和 HUD(Head Up Display,平视显示器)
的操作,当然还有使用鼠标的一些提示。
\fi
\IfLanguageName{french}{
  Le chapitre~\ref{flight}, \textit{En vol : tout sur les instruments,
  les raccourcis clavier et les menus}, d\'{e}crit comment utiliser le
  programme, c'est-\`{a}-dire \`{a} proprement parler comment voler avec
  \FlightGear{}\hspace{-1mm}. Ceci comprend une liste exhaustive (nous
  l'esp\'{e}rons) des raccourcis claviers pr\'{e}d\'{e}finis, une vue
  d'ensemble des entr\'{e}es des menus, des descriptions d\'{e}taill\'{e}es
  du tableau de bord et du collimateur t\^{e}te haute (Head Up Display, HUD),
  ainsi que des astuces sur l'utilisation des fonctions de la souris.
}{}
\IfLanguageName{italian}{
Capitolo~\ref{flight}, \textit{In volo: tutto su strumenti, tasti e men\`{u}}:
Descrive come volare con \FlightGear{}. Contiene l'elenco completo dei comandi
da tastiera predefiniti, spiega le voci dei vari men\`{u} e descrive dettagliatamente
il pannello degli strumenti e l'HUD. Inoltre d\`{a} dei consigli sull'utilizzo del
mouse e delle sue funzioni.
}{}
\medskip

 \noindent
\ifchinese
第~\ref{features} 章,\textit{特性},描述了一些 \FlightGear{} 中为高级用户提供的特殊特性。
\fi
%\IfLanguageName{english}{
%  Chapter~\ref{features}, \textit{Features} describes some of the special
%  features that \FlightGear{} offers to the advanced user.
%}{}
\IfLanguageName{french}{
  Le chapitre~\ref{features}, \textit{Fonctionnalit\'{e}s} d\'{e}crit certaines
  des fonctionnalit\'{e}s particuli\`{e}res que \FlightGear{} offre \`{a} l'utilisateur
  avanc\'{e}.
}{}
\IfLanguageName{italian}{
Capitolo~\ref{features}, \textit{Caratteristiche}:
Descrive alcune delle caratteristiche speciali che FlightGear offre per l'utente avanzato
}{}
 \medskip

\noindent
\ifchinese
\textbf{第三部分:教程}
\fi
%\IfLanguageName{english}{
%\textbf{Part III: Tutorials}
%}{}
\IfLanguageName{french}{
\textbf{Partie III : Tutoriels}
}{}
\IfLanguageName{italian}{
\textbf{Partie III : Tutorial}
}{}\medskip

 \noindent
\ifchinese
第~\ref{tutorials} 章,\textit{飞行教程},提供了为新飞行员准备的一些教程。
\fi
%\IfLanguageName{english}{
%  Chapter~\ref{tutorials}, \textit{Tutorials},
%  provides information on the many tutorials available for new pilots.
%}{}
\IfLanguageName{french}{
  Le chapitre~\ref{tutorials}, \textit{Tutoriels},
  fournit des informations sur les nombreux tutoriels disponibles pour les
  apprentis pilotes.
}{}
\IfLanguageName{italian}{
Capitolo~\ref{tutorials}, \textit{Tutorial}:
Fornisce informazioni sui molti tutorial disponibili per i nuovi utenti
}{}
\medskip

 \noindent
\ifchinese
第~\ref{basic} 章,\textit{基础飞行模拟教程},提供了一份基本飞行教程,以插图的形式举例讲解在 \FlightGear{} 中如何飞行。 
\fi
%\IfLanguageName{english}{
%  Chapter~\ref{basic}, \textit{A Basic Flight Simulator Tutorial},
%  provides a tutorial on the basics of flying, illustrated with many
%  examples on how things actually look in \FlightGear{}.
%}{}
\IfLanguageName{french}{
  Le chapitre~\ref{basic}, \textit{Tutoriel de base du simulateur de vol},
  propose un tutoriel sur les bases du vol, illustr\'{e} avec de nombreux
  exemples sur la fa\c{c}on dont les choses apparaissent dans \FlightGear{}.
}{}
\IfLanguageName{italian}{
Capitolo~\ref{basic}, \textit{Le basi dei simulatori di volo}:
Spiega le basi del volo attraverso molti esempi
}{}
\medskip

 \noindent
\ifchinese
第~\ref{crosscountry} 章,\textit{短途转场飞行教程},从默认安装的旧金山地区出发,做一次短途旅行。
\fi
%\IfLanguageName{english}{
%*  Chapter~\ref{crosscountry}, \textit{A Cross Country Flight Tutorial},
%  describes a simple cross-country flight in the San Fransisco area that
%  can be run with the default installation.
%}{}
\IfLanguageName{french}{
  Le chapitre~\ref{crosscountry}, \textit{Un tutoriel de vol \`{a} travers la campagne},
  d\'{e}crit un vol simple \`{a} travers la campagne dans la r\'{e}gion de San Fransisco,
  qui peut \^{e}tre r\'{e}alis\'{e} avec l'installation par d\'{e}faut.
}{}
\IfLanguageName{italian}{
Capitolo~\ref{crosscountry}, \textit{Tutorial: un volo nazionale}:
Descrive un semplice volo nazionale ambientato nella zona di San Francisco che pu\`{o} essere
eseguito con l'installazione di default
}{}
 \medskip

 \noindent
\ifchinese
第~\ref{IFR Tutorial} 章,\textit{IFR 转场飞行教程},讲述如何使用仪表飞行规则(IFR),完成同样的短途飞行。
\fi
%\IfLanguageName{english}{
%  Chapter~\ref{IFR Tutorial}, \textit{An IFR Cross Country Flight Tutorial},
%  describes a similar cross-country flight making use of the instruments to
%  successfully fly in the clouds under Instrument Flight Rules (IFR).
%}{}
\IfLanguageName{french}{
  Le chapitre~\ref{IFR Tutorial}, \textit{Un tutoriel de vol IFR \`{a} travers la campagne},
  propose un vol similaire \`{a} travers la campagne comprenant l'utilisation des instruments
  afin de permettre de voler sans danger dans les nuages sous le r\'{e}gime de vol aux
  instruments (Instrument Flight Rules, IFR).
}{}
\IfLanguageName{italian}{
Capitolo~\ref{IFR Tutorial}, \textit{Un volo nazionale IFR}:
Descrive un volo nazionale facendo uso degli strumenti di bordo per volare con successo tra
le nuvole seguendo le regole di un volo strumentale
}{}
 \medskip

\noindent
\ifchinese
\textbf{附录}
\fi
%\IfLanguageName{english}{
%\textbf{Appendices}
%}{}
\IfLanguageName{french}{
\textbf{Annexes}
}{}
\IfLanguageName{italian}{
\textbf{Appendici}
}{}
\medskip

 \noindent
\ifchinese
附录~\ref{missed},\textit{复飞:如果有什么拒绝工作},我们尝试帮助你解决一些使用 \FlightGear{} 时的常见问题。
\fi
%\IfLanguageName{english}{
%  In Appendix~\ref{missed}, \textit{Missed approach: If anything refuses to work},
%  we try to help you work through some common problems faced when using \FlightGear{}.
%}{}
\IfLanguageName{french}{
  Dans l'Annexe~\ref{missed}, \textit{Approche manqu\'{e}e : si rien ne fonctionne},
  nous essayons de vous aider \`{a} surmonter certaines difficult\'{e}s fr\'{e}quemment
  rencontr\'{e}es lors de l'utilisation de \FlightGear{}.
}{}
\IfLanguageName{italian}{
Appendice~\ref{missed}, \textit{Se qualcosa si rifiuta di funzionare}:
Spiega come risolvere alcuni problemi comuni che potrebbero verificarsi durante
l'utilizzo di FlightGear
}{}
 \medskip

 \noindent
\ifchinese
最后的附录~\ref{landing},\textit{降落:离开飞机前的一些思考},我们想给那些希望贡献到 \FlightGear{} 开发中的人一些信息。告诉大家还有哪些需要完成。
\fi
%\IfLanguageName{english}{
%  In the final Appendix~\ref{landing}, \textit{Landing: Some further thoughts before leaving the plane},
%  we would like to give credit to those who deserve it, sketch an overview on the development of
%  \FlightGear{} and point out what remains to be done.
%}{}
\IfLanguageName{french}{
  Dans la derni\`{e}re Annexe~\ref{landing}, \textit{Atterrissage : quelques r\'{e}flexions compl\'{e}mentaires
  avant de quitter l'avion}, nous souhaiterions remercier ceux qui le m\'{e}ritent, dresser une vue d'ensemble
  du d\'{e}veloppement de \FlightGear{} et souligner ce qu'il reste \`{a} accomplir.
}{}
\IfLanguageName{italian}{
Appendice~\ref{landing}, \textit{Riflessioni su FlightGear}:
Contiene i crediti di chi ha sviluppato FlightGear e illustra cosa rimane da fare nel simulatore
}{}
 \medskip

\ifchinese
\section{简明阅读}
\fi
%IfLanguageName{english}{
%section{Condensed Reading}
%{}

\IfLanguageName{french}{
\section{Lecture condens\'{e}e}
}{}

\IfLanguageName{italian}{
\section{Lecture condens\'{e}e}
}{}

\ifchinese
为那些不喜欢从头至尾阅读文档的人准备的,我们建议以下章节,可以让你更简单的飞上蓝天。
\fi
%IfLanguageName{english}{
%for those who don't want to read this document from cover to cover, we suggest reading the following
%sections in order to provide an easy way to get into the air:
%}{}

\IfLanguageName{french}{
Pour ceux qui ne veulent pas lire ce document de bout en bout, nous sugg\'{e}rons de lire les sections suivantes
dans l'ordre afin de disposer des minimas pour s'envoler :
}{}

\IfLanguageName{italian}{
Per coloro che non vogliono leggere questo manuale dall'inizio alla fine, si evidenziano le sezioni pi\`{u} importanti
indispensabili per poter usare il simulatore:
}{}

\ifchinese
\begin{tabular}{ll}
 安装:         & 第~\ref{prefligh} 章\\
 启动模拟器      & 第~\ref{takeoff} 章\\
 使用模拟器      & 第~\ref{flight} 章\\
\end{tabular}
\fi
%\IfLanguageName{english}{
%\begin{tabular}{ll}
% Installation :             &~\ref{prefligh}\\
% Starting the simulator :   &~\ref{takeoff}\\
% Using the simulator :      &~\ref{flight}\\
%\end{tabular}
}{}
\IfLanguageName{french}{
\begin{tabular}{ll}
 Installation :             &~\ref{prefligh}\\
 D\'{e}marrer le simulateur :   &~\ref{takeoff}\\
 Utiliser le simulateur :      &~\ref{flight}\\
\end{tabular}
}{}
\IfLanguageName{italian}{
\begin{tabular}{ll}
 Installazione :             &~\ref{prefligh}\\
 Avvio del simulatore :   &~\ref{takeoff}\\
 Utilizzo del simulatore :      &~\ref{flight}\\
\end{tabular}
}{}
\bigskip

\ifchinese
\section{为特别不耐烦的人}
\fi
%\IfLanguageName{english}{
%\section{Instructions For the Truly Impatient}
%}{}
\IfLanguageName{french}{
\section{Instructions pour les grands impatients}
}{}
\IfLanguageName{italian}{
\section{Istruzioni per i pi\`{u} impazienti}
}{}

\ifchinese
我们知道大多数人都讨厌阅读手册。如果你确定你的显卡支持 \Index{OpenGL} (请阅读相关文档,比如大多数 \Index{NVIDIA} 显卡)并且使用 Windows、Mac OS-X 或 Linux,那么你至少可以直接跳过这本手册的第一部分,并使用已经编译好的二进制程序\index{binaries!pre-compiled 预编译的!二进制程序}。相关下载和说明可以前往
\fi
%\IfLanguageName{english}{
%  We know most people hate reading manuals. If you are sure the graphics driver for your card supports \Index{OpenGL} (check documentation; for
%  instance in general \Index{NVIDIA} graphics cards do) and you are using Windows, Mac OS-X or Linux, you can probably skip at least Part I of
%  this manual and use pre-compiled binaries\index{binaries!pre-compiled}. These as well as instructions on how to set them up, can be found at
%}{}
\IfLanguageName{french}{
  Nous savons que la plupart des gens d\'{e}testent lire les manuels. Si vous \^{e}tes certain que votre carte graphique prend en charge
  \Index{OpenGL} (v\'{e}rifiez dans votre documentation; par exemple, la plupart des cartes graphiques \Index{NVIDIA} le font) et que vous
  utilisez Windows, Mac OS-X ou Linux, vous pouvez probablement au moins sauter la Partie I de ce manuel et utiliser des binaires
  pr\'{e}-compil\'{e}s\index{binaries!pre-compiled}. Ils peuvent \^{e}tre trouv\'{e}s, accompagn\'{e}s de leurs instructions d'installation, \`{a}
  l'adresse suivante :
}{}
\IfLanguageName{italian}{
  Sappiamo che la maggior parte delle persone odiano leggere i manuali. Se si \`{e} sicuri che la scheda grafica del proprio computer supporti
  OpenGL (consultare la documentazione; ad esempio  in generale le schede NVIDIA lo supportano) e si utilizza Windows, Mac OS-X o Linux,
  probabilmente si pu\`{o} ignorare almeno parte di questo manuale e usare i file d'installazione pre-compilati.

  Questi, oltre alle istruzioni su come impostarli, possono essere trovati sul sito
}{}
 \medskip

\web{http://www.flightgear.org/download/}.
 \medskip


\ifchinese
如果你使用 Linux,会发现 \FlightGear{} 已经在你的发行版包列表里了。

当你安装好二进制程序以后,可以看第 \ref{takeoff} 章来获取启动模拟器的细节。
\fi
 \noindent
 %\IfLanguageName{english}{
%  If you are running Linux, you may find that \FlightGear{} is bundled with your distribution.

%  Once you have downloaded and installed the binaries, see Chapter \ref{takeoff} for details on starting the simulator.
%}{}
\IfLanguageName{french}{
  Si vous utilisez Linux, vous devriez pouvoir trouver \FlightGear{} dans les paquetages de votre distribution.

  Une fois que vous avez t\'{e}l\'{e}charg\'{e} et install\'{e} les binaires, reportez-vous au Chapitre \ref{takeoff} pour obtenir
  des d\'{e}tails sur le d\'{e}marrage du simulateur.
}{}

\IfLanguageName{italian}{
  Se si usa Linux, \`{e} possibile che \FlightGear{} sia gi\`{a} integrato nel pacchetto con la distribuzione.
  Dopo aver scaricato e installato i file binari, vedere il Capitolo 4 per i dettagli su come avviare il simulatore.
}{}

\ifchinese
\section{延伸阅读}
\fi
%\IfLanguageName{english}{
%\section{Further Reading}
%}{}
\IfLanguageName{french}{
\section{Lecture compl\'{e}mentaire}
}{}
\IfLanguageName{italian}{
\section{Ulteriori documentazioni}
}{}


\ifchinese
虽然这种入门指南,就是要自成一体,不过我们依然强烈建议你阅读一下这些文档,特别是遇到问题的时候:
\fi
\noindent
%\IfLanguageName{english}{
% While this introductory guide is meant to be self contained, we strongly suggest having a look into further documentation,
% especially in case of trouble:
%}{}
\IfLanguageName{french}{
 Bien que ce guide d'introduction contienne toutes les informations n\'{e}cessaires, nous vous recommandons vivement de jeter
 un \oe{}il dans d'autres documents, notamment en cas de difficult\'{e}s :
}{}
\IfLanguageName{italian}{
 Sebbene questa guida introduttiva sia progettata per essere autosufficiente, si suggerisce di leggere le ulteriori
 documentazioni, in particolare in caso di problemi :
}{}

\begin{itemize}
\ifchinese
\item 一张手边备查的\textbf{快速参考}\index{leaflet 快速参考}可以在基本包的 \texttt{/FlightGear/Docs} 目录下找到并打印下来,也可以从这里下载
\fi
%\IfLanguageName{english}{
% \item A handy \textbf{leaflet}\index{leaflet} on operation for printout can be found in base package under \texttt{/FlightGear/Docs},
% and is also available from
%}{}
\IfLanguageName{french}{
 \item un petit \textbf{d\'{e}pliant}\index{leaflet} \`{a} imprimer. Bien pratique en vol, il peut \^{e}tre trouv\'{e} dans le paquetage de
 base dans le r\'{e}pertoire \texttt{/FlightGear/Docs} et est \'{e}galement disponible \`{a} l'adresse :
}{}
\IfLanguageName{italian}{
 \item Un pratico opuscolo riassuntivo \`{e} contenuto nel pacchetto di installazione all'indirizzo \texttt{/FlightGear/Docs} oppure al sito :
}{}

\web{http://www.flightgear.org/Docs/FGShortRef.pdf}.

\ifchinese
\item 额外有关特定特性和功能的\textbf{用户文档}可以在基础包的 \texttt{/Docs} 目录找到。
\fi
%IfLanguageName{english}{
%\item Additional \textbf{user documentation} on particular features and function is available within the
%base package under the directory \texttt{/Docs}.
%{}
\IfLanguageName{french}{
 \item des \textbf{documentations utilisateurs} compl\'{e}mentaires sur des fonctionnalit\'{e}s particuli\`{e}res sont disponibles
 dans le paquetage de base dans le r\'{e}pertoire \texttt{/Docs}.
}{}

\ifchinese
\item 还有一个官方的 \FlightGear{} \textbf{wiki}\index{wiki},可以前往 \web{http://wiki.flightgear.org}。
\fi
%\IfLanguageName{english}{
% \item There is an official \FlightGear{} \textbf{wiki}\index{wiki} available at \web{http://wiki.flightgear.org}.
%}{}
\IfLanguageName{french}{
 \item il existe un \textbf{wiki}\index{wiki} \FlightGear{} officiel disponible \`{a} l'adresse \web{http://wiki.flightgear.org}.
}{}
\IfLanguageName{italian}{
\item \'{E} disponibile anche un \textbf{wiki}\index{wiki} ufficiale di \FlightGear{} all'indirizzo
\web{http://wiki.flightgear.org} e un forum all'indirizzo \web{http://forum.flightgear.org}.
}{}
 \end{itemize}


%% Revision 0.00  1998/09/08  michael
%% Initial revision for version 0.41.
%% Revision 0.01  2002/01/01 michael
%% Included outline of the Guide, integrated former separate chapter ''Quickstart''
%% revision 0.5 2002/01/01 michael/martin
%% Hint on Linux distros
%% revision 0.6 2002/09/09 michael
%% minor corrections
%% revision 0.7 2005/11.10 stuart
%% Changes for moving compilation information to appendix
