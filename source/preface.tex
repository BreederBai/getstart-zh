%%
%% preface.tex -- Flight Gear documentation: The FlightGear Manual
%% Chapter file
%%
%% Written by Michael Basler, started September 1998.
%%
%% Copyright (C) 2002 Michael Basler
%%
%%
%% This program is free software; you can redistribute it and/or
%% modify it under the terms of the GNU General Public License as
%% published by the Free Software Foundation; either version 2 of the
%% License, or (at your option) any later version.
%%
%% This program is distributed in the hope that it will be useful, but
%% WITHOUT ANY WARRANTY; without even the implied warranty of
%% MERCHANTABILITY or FITNESS FOR A PARTICULAR PURPOSE.  See the GNU
%% General Public License for more details.
%%
%% You should have received a copy of the GNU General Public License
%% along with this program; if not, write to the Free Software
%% Foundation, Inc., 675 Mass Ave, Cambridge, MA 02139, USA.
%%
%% $Id: preface.tex,v 0.6 2002/09/09 michael
%% (Log is kept at end of this file)

%%%%%%%%%%%%%%%%%%%%%%%%%%%%%%%%%%%%%%%%%%%%%%%%%%%%%%%%%%%%%%%%%%%%%%%%%%%%%%%%%%%%%%%%%%%%%%%
\iflanguage{english}{
\chapter{Preface}
}{}
\iflanguage{french}{
\chapter{Pr\'{e}face}
}{}
\label{preface}
%%%%%%%%%%%%%%%%%%%%%%%%%%%%%%%%%%%%%%%%%%%%%%%%%%%%%%%%%%%%%%%%%%%%%%%%%%%%%%%%%%%%%%%%%%%%%%%
\iflanguage{english}{
\FlightGear{} is a free Flight Simulator developed cooperatively over the Internet
by a group of flight simulation and programming enthusiasts. "The
FlightGear Manual" is meant to give beginners a guide in getting
\FlightGear{} up and running, and themselves into the air. It is not
intended to provide complete documentation of all the features and
add-ons of \FlightGear{} but, instead, aims to give a new user the best
start to exploring what \FlightGear{} has to offer.
}{}

\iflanguage{french}{
\FlightGear{} est un simulateur de vol libre et gratuit d\'{e}velopp\'{e} grace \`{a} Internet
par une communaut\'{e} de passionn\'{e}s de simulation de vol et de programmation. "Le
manuel FlightGear" a pour but d'offrir aux d\'{e}butants un guide \`{a} l'installation 
et \`{a} l'utilisation de \FlightGear{} et aux premi\`{e}res heures de vol. Il n'a pas pour
vocation de fournir une documentation compl\`{e}te de toutes les fonctionnalit\'{e}s et
ajouts de \FlightGear{} mais, plut\^{o}t, d'apporter au nouvel utilisateur les
meilleures bases pour lui permettre d'explorer ce que \FlightGear{} a \`{a} offrir.
}{}

\iflanguage{english}{
This version of the document was written for \FlightGear{} version 2.6.0.
Users of earlier versions of \FlightGear{} will still find this document
useful, but some of the features described may not be present.

This guide is split into three parts and is structured as follows.
}{}

\iflanguage{french}{
Cette version du manuel a \'{e}t\'{e} \'{e}crite pour la version 2.6.0 de \FlightGear{}.
Les utilisateurs de versions pr\'{e}c\'{e}dentes de \FlightGear{} trouveront toujours
une utilit\'{e} \`{a} ce document, mais certaines des fonctionnalit\'{e}s qui y sont d\'{e}crites
peuvent \^{e}tre absentes de leur version.

Ce guide est scind\'{e} en trois parties et structur\'{e} de la fa\c{c}on suivante : 
}{}

\medskip

\noindent
\iflanguage{english}{
\textbf{Part I: Installation}
}{}
\iflanguage{french}{
\textbf{Partie I : Installation}
}{}
\medskip

 \noindent
\iflanguage{english}{
Chapter~\ref{free}, \textit{Want to have a free flight? Take \FlightGear{}}, introduces
\FlightGear{}, provides background on the philosophy behind it and describes the system requirements.
}{}
\iflanguage{french}{
Le chapitre~\ref{free}, \textit{Vous voulez voler librement ? Choisissez \FlightGear{}} ! pr\'{e}sente
\FlightGear{}, la philosophie qui porte le projet et d\'{e}crit les pr\'{e}-requis syst\`{e}me.
}{}
 \medskip

 \noindent
\iflanguage{english}{
In Chapter~\ref{prefligh}, \textit{Preflight: Installing \FlightGear{}}, you will find
instructions for installing the binaries\index{binary distribution} and additional scenery and aircraft.
}{}
\iflanguage{french}{
Dans le chapitre ~\ref{prefligh}, \textit{Pr\'{e}vol : installer \FlightGear{}}, vous trouverez les
instructions n\'{e}cessaires \`{a} l'installation des binaires\index{binary distribution} et des sc\`{e}nes et avions additionnels.
}{}
 \medskip

\noindent
\iflanguage{english}{
\textbf{Part II: Flying with \FlightGear{}}
}{}
\iflanguage{french}{
\textbf{Partie II : Voler avec \FlightGear{}}
}{}
\medskip

 \noindent
\iflanguage{english}{
  The following Chapter~\ref{takeoff}, \textit{Takeoff: How to start
  the program}, describes how to actually start the installed program.
  It includes an overview on the numerous command line options as well
  as configuration files.
}{}
\iflanguage{french}{
  Le chapitre~\ref{takeoff}, \textit{D\'{e}collage : comment d\'{e}marrer le
  programme}, d\'{e}crit comment lancer le programme une fois installi\'{e}.
  Il comprend un survol des nombreuses options en ligne de commande ainsi
  que des fichiers de configuration.
}{}

 \medskip

 \noindent
\iflanguage{english}{
  Chapter~\ref{flight}, \textit{In-flight: All about instruments,
  keystrokes and menus}, describes how to operate the program, i.e.
  how to actually fly with \FlightGear{}\hspace{-1mm}. This includes a
  (hopefully) complete list of pre-defined keyboard commands, an
  overview on the menu entries, detailed descriptions on the instrument
  panel and HUD (head up display), as well as hints on using the mouse
  functions.
}{}
\iflanguage{french}{
  Le chapitre~\ref{flight}, \textit{En vol : tout sur les instruments,
  les raccourcis clavier et les menus}, d\'{e}crit comment utiliser le
  programme, c'est-\`{a}-dire \`{a} proprement parler comment voler avec
  \FlightGear{}\hspace{-1mm}. Ceci comprend une liste exhaustive (nous
  l'esp\'{e}rons) des raccourcis claviers pr\'{e}d\'{e}finis, une vue
  d'ensemble des entr\'{e}es des menus, des descriptions d\'{e}taill\'{e}es
  du tableau de bord et du collimateur t\^{e}te haute (Head Up Display, HUD),
  ainsi que des astuces sur l'utilisation des fonctions de la souris.
}{}
 \medskip

 \noindent
\iflanguage{english}{
  Chapter~\ref{features}, \textit{Features} describes some of the special
  features that \FlightGear{} offers to the advanced user.
}{}
\iflanguage{french}{
  Le chapitre~\ref{features}, \textit{Fonctionnalit\'{e}s} d\'{e}crit certaines
  des fonctionnalit\'{e}s particuli\`{e}res que \FlightGear{} offre \`{a} l'utilisateur
  avanc\'{e}.
}{}
 \medskip

\noindent
\iflanguage{english}{
\textbf{Part III: Tutorials}
}{}
\iflanguage{french}{
\textbf{Partie III : Tutoriels}
}{}
\medskip

 \noindent
\iflanguage{english}{
  Chapter~\ref{tutorials}, \textit{Tutorials},
  provides information on the many tutorials available for new pilots.
}{}
\iflanguage{french}{
  Le chapitre~\ref{tutorials}, \textit{Tutoriels},
  fournit des informations sur les nombreux tutoriels disponibles pour les
  apprentis pilotes.
}{}
 \medskip

 \noindent
\iflanguage{english}{
  Chapter~\ref{basic}, \textit{A Basic Flight Simulator Tutorial},
  provides a tutorial on the basics of flying, illustrated with many
  examples on how things actually look in \FlightGear{}.
}{}
\iflanguage{french}{
  Le chapitre~\ref{basic}, \textit{Tutoriel de base du simulateur de vol},
  propose un tutoriel sur les bases du vol, illustr\'{e} avec de nombreux
  exemples sur la fa\c{c}on dont les choses apparaissent dans \FlightGear{}.
}{}
 \medskip

 \noindent
\iflanguage{english}{
  Chapter~\ref{crosscountry}, \textit{A Cross Country Flight Tutorial},
  describes a simple cross-country flight in the San Fransisco area that
  can be run with the default installation.
}{}
\iflanguage{french}{
  Le chapitre~\ref{crosscountry}, \textit{Un tutoriel de vol \`{a} travers la campagne},
  d\'{e}crit un vol simple \`{a} travers la campagne dans la r\'{e}gion de San Fransisco,
  qui peut \^{e}tre r\'{e}alis\'{e} avec l'installation par d\'{e}faut.
}{}
 \medskip

 \noindent
\iflanguage{english}{
  Chapter~\ref{IFR Tutorial}, \textit{An IFR Cross Country Flight Tutorial},
  describes a similar cross-country flight making use of the instruments to
  successfully fly in the clouds under Instrument Flight Rules (IFR).
}{}
\iflanguage{french}{
  Le chapitre~\ref{IFR Tutorial}, \textit{Un tutoriel de vol IFR \`{a} travers la campagne},
  propose un vol similaire \`{a} travers la campagne comprenant l'utilisation des instruments
  afin de permettre de voler sans danger dans les nuages sous le r\'{e}gime de vol aux
  instruments (Instrument Flight Rules, IFR).
}{}
 \medskip

\noindent
\iflanguage{english}{
\textbf{Appendices}
}{}
\iflanguage{french}{
\textbf{Annexes}
}{}
\medskip

 \noindent
\iflanguage{english}{
  In Appendix~\ref{missed}, \textit{Missed approach: If anything refuses to work},
  we try to help you work through some common problems faced when using \FlightGear{}.
}{}
\iflanguage{french}{
  Dans l'Annexe~\ref{missed}, \textit{Approche manqu\'{e}e : si rien ne fonctionne},
  nous essayons de vous aider \`{a} surmonter certaines difficult\'{e}s fr\'{e}quemment
  rencontr\'{e}es lors de l'utilisation de \FlightGear{}.
}{}
 \medskip

 \noindent
\iflanguage{english}{
  In the final Appendix~\ref{landing}, \textit{Landing: Some further thoughts before leaving the plane},
  we would like to give credit to those who deserve it, sketch an overview on the development of
  \FlightGear{} and point out what remains to be done.
}{}
\iflanguage{french}{
  Dans la derni\`{e}re Annexe~\ref{landing}, \textit{Atterrissage : quelques r\'{e}flexions compl\'{e}mentaires
  avant de quitter l'avion}, nous souhaiterions remercier ceux qui le m\'{e}ritent, dresser une vue d'ensemble
  du d\'{e}veloppement de \FlightGear{} et souligner ce qu'il reste \`{a} accomplir.
}{}
 \medskip

\iflanguage{english}{
\section{Condensed Reading}
}{}

\iflanguage{french}{
\section{Lecture condens\'{e}e}
}{}

\iflanguage{english}{
For those who don't want to read this document from cover to cover, we suggest reading the following
sections in order to provide an easy way to get into the air:
}{}

\iflanguage{french}{
Pour ceux qui ne veulent pas lire ce document de bout en bout, nous sugg\'{e}rons de lire les sections suivantes
dans l'ordre afin de disposer des minimas pour s'envoler :
}{}

\iflanguage{english}{
\begin{tabular}{ll}
 Installation :             &~\ref{prefligh}\\
 Starting the simulator :   &~\ref{takeoff}\\
 Using the simulator :      &~\ref{flight}\\
\end{tabular}
}{}
\iflanguage{french}{
\begin{tabular}{ll}
 Installation :             &~\ref{prefligh}\\
 D\'{e}marrer le simulateur :   &~\ref{takeoff}\\
 Utiliser le simulateur :      &~\ref{flight}\\
\end{tabular}
}{}
\bigskip

\iflanguage{english}{
\section{Instructions For the Truly Impatient}
}{}
\iflanguage{french}{
\section{Instructions pour les grands impatients}
}{}
\iflanguage{english}{
  We know most people hate reading manuals. If you are sure the graphics driver for your card supports \Index{OpenGL} (check documentation; for
  instance in general \Index{NVIDIA} graphics cards do) and you are using Windows, Mac OS-X or Linux, you can probably skip at least Part I of
  this manual and use pre-compiled binaries\index{binaries!pre-compiled}. These as well as instructions on how to set them up, can be found at
}{}
\iflanguage{french}{
  Nous savons que la plupart des gens d\'{e}testent lire les manuels. Si vous \^{e}tes certain que votre carte graphique prend en charge
  \Index{OpenGL} (v\'{e}rifiez dans votre documentation; par exemple, la plupart des cartes graphiques \Index{NVIDIA} le font) et que vous
  utilisez Windows, Mac OS-X ou Linux, vous pouvez probablement au moins sauter la Partie I de ce manuel et utiliser des binaires
  pr\'{e}-compil\'{e}s\index{binaries!pre-compiled}. Ils peuvent \^{e}tre trouv\'{e}s, accompagn\'{e}s de leurs instructions d'installation, \`{a}
  l'adresse suivante :
}{}
 \medskip

\web{http://www.flightgear.org/Downloads/}.
 \medskip

 \noindent
\iflanguage{english}{
  If you are running Linux, you may find that \FlightGear{} is bundled with your distribution.

  Once you have downloaded and installed the binaries, see Chapter \ref{takeoff} for details on starting the simulator.
}{}
\iflanguage{french}{
  Si vous utilisez Linux, vous devriez pouvoir trouver \FlightGear{} dans les paquetages de votre distribution.

  Une fois que vous avez t\'{e}l\'{e}charg\'{e} et install\'{e} les binaires, reportez-vous au Chapitre \ref{takeoff} pour obtenir
  des d\'{e}tails sur le d\'{e}marrage du simulateur.
}{}

\iflanguage{english}{
\section{Further Reading}
}{}
\iflanguage{french}{
\section{Lecture compl\'{e}mentaire}
}{}

\noindent
\iflanguage{english}{
 While this introductory guide is meant to be self contained, we strongly suggest having a look into further documentation,
 especially in case of trouble:
}{}
\iflanguage{french}{
 Bien que ce guide d'introduction contienne toutes les informations n\'{e}cessaires, nous vous recommandons vivement de jeter
 un \oe{}il dans d'autres documents, notamment en cas de difficult\'{e}s :
}{}

\begin{itemize}
\iflanguage{english}{
 \item A handy \textbf{leaflet}\index{leaflet} on operation for printout can be found in base package under \texttt{/FlightGear/Docs},
 and is also available from
}{}
\iflanguage{french}{
 \item un petit \textbf{d\'{e}pliant}\index{leaflet} \`{a} imprimer. Bien pratique en vol, il peut \^{e}tre trouv\'{e} dans le paquetage de
 base dans le r\'{e}pertoire \texttt{/FlightGear/Docs} et est \'{e}galement disponible \`{a} l'adresse :
}{}

\web{http://www.flightgear.org/Docs/FGShortRef.pdf}.

\iflanguage{english}{
 \item Additional \textbf{user documentation} on particular features and function is available within the
 base package under the directory \texttt{/FlightGear/Docs}.
}{}
\iflanguage{french}{
 \item des \textbf{documentations utilisateurs} compl\'{e}mentaires sur des fonctionnalit\'{e}s particuli\`{e}res sont disponibles
 dans le paquetage de base dans le r\'{e}pertoire \texttt{/FlightGear/Docs}.
}{}
\iflanguage{english}{
 \item There is an official \FlightGear{} \textbf{wiki}\index{wiki} available at \web{http://wiki.flightgear.org}.
}{}
\iflanguage{french}{
 \item il existe un \textbf{wiki}\index{wiki} \FlightGear{} officiel disponible \`{a} l'adresse \web{http://wiki.flightgear.org}.
}{}
 \end{itemize}


%% Revision 0.00  1998/09/08  michael
%% Initial revision for version 0.41.
%% Revision 0.01  2002/01/01 michael
%% Included outline of the Guide, integrated former separate chapter ''Quickstart''
%% revision 0.5 2002/01/01 michael/martin
%% Hint on Linux distros
%% revision 0.6 2002/09/09 michael
%% minor corrections
%% revision 0.7 2005/11.10 stuart
%% Changes for moving compilation information to appendix
