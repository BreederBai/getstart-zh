%%
%% FGShortRef.tex -- Flight Gear documentation: Short reference
%%
%% Written by Michael Basler, starting May 2001.
%%
%% Copyright (C) 2002 Michael Basler
%%
%% This program is free software; you can redistribute it and/or
%% modify it under the terms of the GNU General Public License as
%% published by the Free Software Foundation; either version 2 of the
%% License, or (at your option) any later version.
%%
%% This program is distributed in the hope that it will be useful, but
%% WITHOUT ANY WARRANTY; without even the implied warranty of
%% MERCHANTABILITY or FITNESS FOR A PARTICULAR PURPOSE.  See the GNU
%% General Public License for more details.
%%
%% You should have received a copy of the GNU General Public License
%% along with this program; if not, write to the Free Software
%% Foundation, Inc., 675 Mass Ave, Cambridge, MA 02139, USA.
%%
%% $Id: FGShortRef.tex,v 0.6 2002/09/09 michael
%% (Log is kept at end of this file)

\documentclass[10pt]{article}
\usepackage{graphicx}
\usepackage{times}
\usepackage{hyperref}
\usepackage{multicol}
\pagestyle{empty}
\usepackage{a4}
\usepackage[german,french,english]{babel}
\selectlanguage{english}

\newcommand{\Index}[1]{#1\index{#1}}
\newcommand{\FlightGear}{{\itshape\bfseries FlightGear}}
\newcommand{\TerraGear}{{\itshape\bfseries TerraGear}}
\newcommand{\SimGear}{{\itshape\bfseries SimGear}}
\newcommand{\PLIB}{{\itshape\bfseries PLIB}}
\newcommand{\JSBSim}{{\itshape\bfseries JSBSim}}
\newcommand{\web}[1]{\href{#1}{#1}}
\newcommand{\mail}[1]{\href{mailto:#1}{#1}}
\newcommand{\Cygwin}{{\itshape\bfseries Cygwin}}

\newcommand{\longpage}{\enlargethispage{\baselineskip}}
\newcommand{\shortpage}{\enlargethispage{-\baselineskip}}

\makeindex

\begin{document}
\longpage

\centerline{\large \textbf{\FlightGear{} Short Reference}}
\medskip

\scriptsize \noindent
%\footnotesize \noindent
 \FlightGear{} is a free flight simulator developed collectively over the
 Internet under the GPL.  For more information see  \web{http://www.flightgear.org/}\\

\hspace*{-8mm}
\begin{tabular}{ll}
\textbf{Program Start:}  & Linux/UNIX via fgfs under FlightGear/,\\
                         & Mac OS X via FlightGear.app under /Applications/,\\
                         & Windows via the \FlightGear{} wizard fgrun.exe under
         $\backslash$Program Files$\backslash$FlightGear$\backslash$bin$\backslash$Win32$\backslash$\\

\textbf{Engine Start:}   & Set ignition switch to BOTH (``\}'' three times). Set mixture to 100\%.
                           Set throttle to about 25\%. Operate starter using the ``s'' key. \\
                         & Once the engine has started, set throttle back to idle.
                           Release parking brake (``B''), if applied.
\end{tabular}
\medskip

%%%%%%%%%%%%%%%%%%%%%%%%%%%%%%%%%%%%%%%%%%%%%%%%%%%%%%%%%%%%
 \noindent
 \textbf{Keyboard controls:}
\begin{multicols}{2}
 \noindent
Table 1: \textit{Directional controls (activated \texttt{NumLock}})\\

\noindent
%%
%% tab2.tex -- Flight Gear documentation: The FlightGear Manual
%% Keyboard controls table 1/Main controls
%%
%% Written by Michael Basler, started September 1998.
%%
%% Copyright (C) 2002 Michael Basler
%%
%%
%% This program is free software; you can redistribute it and/or
%% modify it under the terms of the GNU General Public License as
%% published by the Free Software Foundation; either version 2 of the
%% License, or (at your option) any later version.
%%
%% This program is distributed in the hope that it will be useful, but
%% WITHOUT ANY WARRANTY; without even the implied warranty of
%% MERCHANTABILITY or FITNESS FOR A PARTICULAR PURPOSE.  See the GNU
%% General Public License for more details.
%%
%% You should have received a copy of the GNU General Public License
%% along with this program; if not, write to the Free Software
%% Foundation, Inc., 675 Mass Ave, Cambridge, MA 02139, USA.
%%
%% $Id: tab1.tex,v 0.6 2002/09/09 michael
%% (Log is kept at end of this file)
%%%%%%%%%%%%%%%%%%%%%%%%%%%%%%%%%%%%%%%%%%%%%%%%%%%%%%%%%%%%%%%%%%%%%%%%%%%%%%%%%%%%%%%%%%%%%%%%
\begin{tabular}{|l|l|}\hline
\IfLanguageName{english}{
  Key      &  Action\\\hline
 9 / 3     &  Throttle\index{throttle}\\
 4 / 6     &  Aileron\index{aileron}\\
 8 / 2     &  Elevator\index{elevator}\\
 0 / Enter &  Rudder\index{rudder}\\
 5         &  Center aileron/elevator/rudder\\
 7 / 1     &  Elevator \Index{trim}\\\hline
}{}
\IfLanguageName{french}{
  Touche      &  Action\\\hline
 9 / 3     &  Commande des gaz\index{gaz}\\
 4 / 6     &  Aileron\index{aileron}\\
 8 / 2     &  Gouverne de profondeur\index{gouverne de profondeur}\\
 0 / Entr\'{e}e &  gouverne de direction\index{gouverne de direction}\\
 5         &  Centrage des ailerons/gouverne de profondeur/direction\\
 7 / 1     &  Trim de la gouverne de profondeur \Index{trim}\\\hline
}{}
\IfLanguageName{italian}{
  Pulsante/i      &  Azione\\\hline
 9 / 3     &  Manetta (Throttle)\index{manetta}\index{throttle}\\
 4 / 6     &  Alettoni\index{Alettoni}\\
 8 / 2     &  Elevatore (timone di coda)\index{Elevatore}\\
 0 / Invio &  Timone di direzione\index{gouverne de direction}\\
 5         &  Centra alettoni, timone di coda e di direzione\\
 7 / 1     &  Regola il trim \Index{trim}\\\hline
}{}

\end{tabular}

%% revision 0.5 2002/02/15 michael
%% Initial revision

\bigskip

%%%%%%%%%%%%%%%%%%%%%%%%%%%%%%%%%%%%%%%%%%%%%%%%%%%%%%%%%%%%
 \noindent
Table 2: \textit{Engine controls}
\medskip

 \noindent
%%
%% tab7.tex -- Flight Gear documentation: The FlightGear Manual
%% Keyboard controls table 6/Engine related controls
%%
%% Written by Michael Basler, started September 1998.
%%
%% Copyright (C) 2002 Michael Basler
%%
%%
%% This program is free software; you can redistribute it and/or
%% modify it under the terms of the GNU General Public License as
%% published by the Free Software Foundation; either version 2 of the
%% License, or (at your option) any later version.
%%
%% This program is distributed in the hope that it will be useful, but
%% WITHOUT ANY WARRANTY; without even the implied warranty of
%% MERCHANTABILITY or FITNESS FOR A PARTICULAR PURPOSE.  See the GNU
%% General Public License for more details.
%%
%% You should have received a copy of the GNU General Public License
%% along with this program; if not, write to the Free Software
%% Foundation, Inc., 675 Mass Ave, Cambridge, MA 02139, USA.
%%
%% $Id: tab6.tex,v 0.6 2002/09/09 michael
%% (Log is kept at end of this file)
%%%%%%%%%%%%%%%%%%%%%%%%%%%%%%%%%%%%%%%%%%%%%%%%%%%%%%%%%%%%%%%%%%%%%%%%%%%%%%%%%%%%%%%%%%%%%%%%
\begin{tabular}{|l|l|}\hline
  \ifchinese
  按键   &    操作\\\hline
  !     &    选择第一号发动机\\
  @     &    选择第二号发动机\\
  \#    &    选择第三号发动机\\
  \$    &    选择第四号发动机\\
  $\sim$$ &  选择所有发动机\\\hline
  \{    &    减少所选发动机的磁电机档位\\
  \}    &    增加所选发动机的磁电机档位\\
  s     &    为所选的发动机点火启动\\
  M / m &    贫油/富油所选发动机的油气混合比\\
  N / n &    减少/增加所选发动机的螺旋桨转速 \\\hline
  \fi
\iffalse
\IfLanguageName{english}{
Key      &  Action\\ \hline
   !     & Select 1st engine\\
   @	   & Select 2nd engine\\
  \#     & Select 3rd engine\\
  \$     & Select 4th engine\\
  $\sim$ & Select all engines\\\hline
  \{     & Decrease magneto on selected engine\\
  \}     & Increase magneto on selected engine\\
   s     & Fire starter on selected engine(s)\\
  M / m  & Lean/Enrich selected engine mixture\\
  N / n  & Decrease/Increase selected propeller RPM\\\hline
}{}
\fi
\IfLanguageName{french}{
Touche     &  Action\\ \hline
   !     & S\'{e}l\'{e}ctionne le premier moteur\\
   @	 & S\'{e}l\'{e}ctionne le deuxi\`{e}me moteur\\
  \#     & S\'{e}l\'{e}ctionne le troisi\`{e}me moteur\\
  \$     & S\'{e}l\'{e}ctionne le quatri\`{e}me moteur\\
  $\sim$ & S\'{e}l\'{e}ctionne tous les moteurs\\\hline
  \{     & D\'{e}cr\'{e}mente le magn\'{e}to sur le moteur s\'{e}lectionn\'{e}\\
  \}     & Incr\'{e}mente le magn\'{e}to sur le moteur s\'{e}lectionn\'{e}\\
   s     & D\'{e}marrage du/des moteur(s) s\'{e}lectionn\'{e}(s)\\
  M / m  & Appauvrir/Enrichir le m\'{e}lange du moteur s\'{e}lectionn\'{e}\\
  N / n  & Diminue/Augmente le nombre de tours par minute du moteur s\'{e}lectionn\'{e}\\\hline
}{}
\IfLanguageName{italian}{
Pulsante/i &  Azione\\ \hline
   !     & Seleziona il 1\textdegree{} motore\\
   @	   & Seleziona il 2\textdegree{} motore\\
  \#     & Seleziona il 3\textdegree{} motore\\
  \$     & Seleziona il 4\textdegree{} motore\\
  $\sim$ & Seleziona tutti i motori\\\hline
  \{     & Diminuisce la forza dei magneti sui motori selezionati\\
  \}     & Aumenta la forza dei magneti sui motori selezionati\\
   s     & Avvia i(l) motore/i selezionato/i\\
  M / m  & Impoverisce/Arricchisce la miscela del motore selezionato\\
  N / n  & Diminuisce/Aumenta il passo dell'elica del motore selezionato\\\hline
}{}
\end{tabular}

%% revision 0.5 2002/02/15 michael
%% Initial revision

\medskip

%%%%%%%%%%%%%%%%%%%%%%%%%%%%%%%%%%%%%%%%%%%%%%%%%%%%%%%%%%%%
 \noindent
Table 3: \textit{Miscellaneous aircraft controls}
\medskip

 \noindent
%%
%% tab8.tex -- Flight Gear documentation: The FlightGear Manual
%% Keyboard controls table 7/Miscellaneous
%%
%% Written by Michael Basler, started September 1998.
%%
%% Copyright (C) 2002 Michael Basler
%%
%%
%% This program is free software; you can redistribute it and/or
%% modify it under the terms of the GNU General Public License as
%% published by the Free Software Foundation; either version 2 of the
%% License, or (at your option) any later version.
%%
%% This program is distributed in the hope that it will be useful, but
%% WITHOUT ANY WARRANTY; without even the implied warranty of
%% MERCHANTABILITY or FITNESS FOR A PARTICULAR PURPOSE.  See the GNU
%% General Public License for more details.
%%
%% You should have received a copy of the GNU General Public License
%% along with this program; if not, write to the Free Software
%% Foundation, Inc., 675 Mass Ave, Cambridge, MA 02139, USA.
%%
%% $Id: tab7.tex,v 0.5 2002/15/02 michael
%% (Log is kept at end of this file)
%%%%%%%%%%%%%%%%%%%%%%%%%%%%%%%%%%%%%%%%%%%%%%%%%%%%%%%%%%%%%%%%%%%%%%%%%%%%%%%%%%%%%%%%%%%%%%%%
\begin{tabular}{|l|l|}\hline
  \ifchinese
  按键       &    操作\\\hline
  b         &    启用所有\Index{刹车}\\
  , / .     &    启用左右刹车\\
            &    (用于\Index{差动刹车})\\
  l         &    切换\Index{尾轮锁定}\\
  B         &    切换停留刹车 \index{刹车}\index{停留刹车}\\
  g / G     &    收起/放下起落架 \index{机轮}\index{起落架}\\
  空格       &    按键通话(PTT)\\
  - / \_    &    多人飞行时文字聊天菜单/输入\\
  $[$ / $]$ &    收起/放下\Index{襟翼}\\
  j / k     &    收起/张开\Index{扰流板}\\
  CTRL-B    &    切换\Index{减速板}\\\hline
  \fi
\iffalse
\IfLanguageName{english}{
Key           &  Action\\\hline
  b           & Apply  all \Index{brakes}\\
  , / .       & Apply left/right brake \\
              & (useful for \Index{differential braking})\\
  l           & Toggle \Index{tail-wheel lock}\\
  B           & Toggle parking brake \index{brakes}\index{parking brake}\\
  g/G         & Raise/lower landing gear\index{gear}\index{landing gear}\\
  Space       & Push To Talk (PTT)\\
  - / \_      & MP text chat menu/entry\\
  $[$ / $]$   & Retract/extend \Index{flaps}\\
  j / k       & Retract/extend \Index{spoilers}\\
  Ctrl-B      & Toggle \Index{speed brakes}\\ \hline
}{}
\fi
\IfLanguageName{french}{
Touche        &  Action\\\hline
  b           & Appliquer tous les \Index{freins}\\
  , / .       & Appliquer le frein gauche/droit\\
              & (utile pour le \Index{freinage diff\'{e}rentiel})\\
  l           & Active le \Index{verrouillage de la roue de queue}\\
  B           & Active le frein de parking \index{freins}\index{freins de parking}\\
  g/G         & Monter/descendre le train d'atterrissage\index{train d'atterrissage}\index{train d'atterrissage}\\
  Espace      & Appuyez pour parler (Push To Talk, PTT)\\
  - / \_      & Entr\'{e}e/menu du clavardage clavier du mode multijoueurs\\
  $[$ / $]$   & Rentre/d\'{e}ploie les \Index{volets}\\
  j / k       & Rentre/d\'{e}ploie les \Index{a\'{e}rofreins}\\
  Ctrl-B      & Active les \Index{freins de vitesse}\\ \hline
}{}
\IfLanguageName{italian}{
Pulsante/i     &  Azione\\\hline
  b           & Applica tutti i \Index{freni} (tenere premuto)\\
  , / .       & Applica/Toglie il freno di sinistra/destra \\
  l           & Mette/Toglie il \Index{blocco delle ruote di coda}\\
  B           & Toglie/Mette i \Index{freni di parcheggio}\\
  g/G         & Alza/Abbassa il carrello d'atterraggio\index{carrello d'atterraggio}\\
  Space       & Tenendolo premuto si parla usando FGCom (PTT)\\
  - / \_      & Comandano la chat scritta\\
  $[$ / $]$   & Ritrae/Estende i \Index{flaps}\\
  j / k       & Ritrae/Estende i \Index{deflettori}\\
  Ctrl-B      & Applica/Toglie gli \Index{aerofreni}\\ \hline
}{}
\end{tabular}

%% revision 0.5 2002/02/15 michael
%% Initial revision

\bigskip


%%%%%%%%%%%%%%%%%%%%%%%%%%%%%%%%%%%%%%%%%%%%%%%%%%%%%%%%%%%%
 \noindent
Table 4: \textit{General simulator controls}
\medskip

 \noindent
%%
%% tab6.tex -- Flight Gear documentation: Installation and Getting Started
%% Keyboard controls table 5/key actions for autopilot enabled
%%
%% Written by Michael Basler, started September 1998.
%%
%% Copyright (C) 2002 Michael Basler (pmb@epost.de)
%%
%%
%% This program is free software; you can redistribute it and/or
%% modify it under the terms of the GNU General Public License as
%% published by the Free Software Foundation; either version 2 of the
%% License, or (at your option) any later version.
%%
%% This program is distributed in the hope that it will be useful, but
%% WITHOUT ANY WARRANTY; without even the implied warranty of
%% MERCHANTABILITY or FITNESS FOR A PARTICULAR PURPOSE.  See the GNU
%% General Public License for more details.
%%
%% You should have received a copy of the GNU General Public License
%% along with this program; if not, write to the Free Software
%% Foundation, Inc., 675 Mass Ave, Cambridge, MA 02139, USA.
%%
%% $Id: tab5.tex,v 0.6 2002/09/09 michael
%% (Log is kept at end of this file)
%%%%%%%%%%%%%%%%%%%%%%%%%%%%%%%%%%%%%%%%%%%%%%%%%%%%%%%%%%%%%%%%%%%%%%%%%%%%%%%%%%%%%%%%%%%%%%%%
\begin{tabular}{|l|l|}\hline
 Key           &         Action\\\hline
    8 / 2      &         Altitude adjust\\
    0 / ,      &         Heading adjust\\
    9 / 3      &         Autothrottle adjust \\\hline
 \end{tabular}

%% revision 0.5 2002/02/15 michael
%% Initial revision
\medskip

%%%%%%%%%%%%%%%%%%%%%%%%%%%%%%%%%%%%%%%%%%%%%%%%%%%%%%%%%%%%
\noindent
Table 5: \textit{View controls (de-activated \texttt{NumLock})}
\medskip

 \noindent
 %%
%% tab3.tex -- Flight Gear documentation: The FlightGear Manual
%% Keyboard controls table 2/View directions
%%
%% Written by Michael Basler, started September 1998.
%%
%% Copyright (C) 2002 Michael Basler
%%
%%
%% This program is free software; you can redistribute it and/or
%% modify it under the terms of the GNU General Public License as
%% published by the Free Software Foundation; either version 2 of the
%% License, or (at your option) any later version.
%%
%% This program is distributed in the hope that it will be useful, but
%% WITHOUT ANY WARRANTY; without even the implied warranty of
%% MERCHANTABILITY or FITNESS FOR A PARTICULAR PURPOSE.  See the GNU
%% General Public License for more details.
%%
%% You should have received a copy of the GNU General Public License
%% along with this program; if not, write to the Free Software
%% Foundation, Inc., 675 Mass Ave, Cambridge, MA 02139, USA.
%%
%% $Id: tab2.tex,v 0.6 2002/09/09 michael
%% (Log is kept at end of this file)
%%%%%%%%%%%%%%%%%%%%%%%%%%%%%%%%%%%%%%%%%%%%%%%%%%%%%%%%%%%%%%%%%%%%%%%%%%%%%%%%%%%%%%%%%%%%%%%%
\begin{tabular}{|c|l|}\hline
\iflanguage{english}{
   Numpad Key  &  View direction\index{view directions}\\\hline
    Shift-8  & Forward\\
    Shift-7  & Left/forward\\
    Shift-4  & Left\\
    Shift-1  & Left/back\\
    Shift-2  & Back\\
    Shift-3  & Right/back\\
    Shift-6  & Right\\
    Shift-9  & Right/forward\\\hline
}{}
\iflanguage{french}{
   Touche pav\'{e} num\'{e}rique & Angle de vue\index{angle de vue}\\\hline
    Shift-8  & Vers l'avant\\
    Shift-7  & Avant/Gauche\\
    Shift-4  & Gauche\\
    Shift-1  & Arri\`{e}re/Gauche\\
    Shift-2  & Arri\`{e}re\\
    Shift-3  & Arri\`{e}re/Droit\\
    Shift-6  & Droit\\
    Shift-9  & Avant/Droit\\\hline
}{}
\end{tabular}

%% revision 0.5 2002/02/15 michael
%% Initial revision

\medskip


%%%%%%%%%%%%%%%%%%%%%%%%%%%%%%%%%%%%%%%%%%%%%%%%%%%%%%%%%%%%
\medskip

 \noindent
 Table 6: \textit{Autopilot controls}
\medskip

\noindent
%%
%% tab5.tex -- Flight Gear documentation: Installation and Getting Started
%% Keyboard controls table 4/autopilot controls
%%
%% Written by Michael Basler, started September 1998.
%%
%% Copyright (C) 2002 Michael Basler
%%
%%
%% This program is free software; you can redistribute it and/or
%% modify it under the terms of the GNU General Public License as
%% published by the Free Software Foundation; either version 2 of the
%% License, or (at your option) any later version.
%%
%% This program is distributed in the hope that it will be useful, but
%% WITHOUT ANY WARRANTY; without even the implied warranty of
%% MERCHANTABILITY or FITNESS FOR A PARTICULAR PURPOSE.  See the GNU
%% General Public License for more details.
%%
%% You should have received a copy of the GNU General Public License
%% along with this program; if not, write to the Free Software
%% Foundation, Inc., 675 Mass Ave, Cambridge, MA 02139, USA.
%%
%% $Id: tab4.tex,v 0.6 2002/09/09 michael
%% (Log is kept at end of this file)
%%%%%%%%%%%%%%%%%%%%%%%%%%%%%%%%%%%%%%%%%%%%%%%%%%%%%%%%%%%%%%%%%%%%%%%%%%%%%%%%%%%%%%%%%%%%%%%%
\begin{tabular}{|l|l|}\hline
 Key              &         Action\\\hline
    Ctrl + A      &         Altitude hold\index{altitude hold} toggle on/off\\
    Ctrl + G      &         Follow glide slope 1 toggle on/off\\
    Ctrl + H      &         Heading hold\index{heading hold} toggle on/off\\
    Ctrl + N      &         Follow NAV 1 radial toggle on/off\\
    Ctrl + S      &         Autothrottle\index{autothrottle} toggle on/off\\
    Ctrl + T      &         Terrain follow toggle on/off\\
    Ctrl + U      &         Add 1000 ft. to your altitude (emergency)\\
    Enter		      &         Increase autopilot heading\\
    F6 		    		&         Toggle autopilot target:\\
                  &         current heading/waypoint\\
    F11           &         Autopilot altitude dialog\\
    F12           &         Autopilot heading dialog\\\hline
\end{tabular}

%% revision 0.5 2002/02/15 michael
%% Initial revision
\medskip

%%%%%%%%%%%%%%%%%%%%%%%%%%%%%%%%%%%%%%%%%%%%%%%%%%%%%%%%%%%%
 \noindent
 Table 7: \textit{Display controls}
\medskip

 \noindent
%%
%% tab4.tex -- Flight Gear documentation: Installation and Getting Started
%% Keyboard controls table 3/Additional view options
%%
%% Written by Michael Basler, started September 1998.
%%
%% Copyright (C) 2002 Michael Basler (pmb@epost.de)
%%
%%
%% This program is free software; you can redistribute it and/or
%% modify it under the terms of the GNU General Public License as
%% published by the Free Software Foundation; either version 2 of the
%% License, or (at your option) any later version.
%%
%% This program is distributed in the hope that it will be useful, but
%% WITHOUT ANY WARRANTY; without even the implied warranty of
%% MERCHANTABILITY or FITNESS FOR A PARTICULAR PURPOSE.  See the GNU
%% General Public License for more details.
%%
%% You should have received a copy of the GNU General Public License
%% along with this program; if not, write to the Free Software
%% Foundation, Inc., 675 Mass Ave, Cambridge, MA 02139, USA.
%%
%% $Id: tab3.tex,v 0.6 2002/09/09 michael
%% (Log is kept at end of this file)
%%%%%%%%%%%%%%%%%%%%%%%%%%%%%%%%%%%%%%%%%%%%%%%%%%%%%%%%%%%%%%%%%%%%%%%%%%%%%%%%%%%%%%%%%%%%%%%%
\begin{tabular}{|l|l|}\hline
 Key              &         Action\\\hline
 P                &    Toggle \Index{instrument panel} on/off \\
 c                &    Toggle3D/2D cockpit
 											 \index{2D cockpit} (if both are available)
 											 \index{3D cockpit}\index{cockpit}\\
 s                &    Cycle panel style full/mini\\
 Shift-F5/F6      &    Shift the panel in y direction\\
 Shift-F7/F8      &    Shift the panel in x direction\\
 Shift-F3					&    Read a panel from a property list\\
 i/I              &    Minimize/maximize HUD              \\
 h/H              &    Change color  of HUD/toggle HUD off\\
                  &    forward/backward      \\   \hline
  x/X             &    Zoom in/out\\
   v              &    Cycle \Index{view modes} (pilot, chase, tower)\\ \hline
   W              &    Toggle \Index{full screen mode} on/off (3dfx only)\\
   z/Z            &    Change \Index{visibility} (fog)  forward/backward \\
   F8             &    Toggle fog on/off\\
   F2			 				& 	 Refresh Scenery tile cache\\
   F4			 				& 	 Force Lighting update\\
   F9             &    Toggle texturing on/off\\
   F10      			&    Toggle menu on/off\\ \hline   
 \end{tabular}

%% revision 0.5 2002/02/15 michael
%% Initial revision
\bigskip

\end{multicols}

 \noindent
 \textbf{Mouse controlled functions:}
 There are three mouse modes, which can be swapped between by clicking the right mouse button.

 \begin{enumerate}
 \item In \textbf{normal} mode (pointer cursor), the panel and cockpit controls can be
 operated using the mouse. To change a control, click with the left/middle mouse button
 on the corresponding knob/lever. Generally, the left side of the control decreases the setting,
 while the right side increases the setting. The left mouse button makes small changes while the
 middle button makes larger ones. The scrollwheel may be used on some controls.
 Press Ctrl-c to view panel/cockpit hotspots.

 \item In \textbf{control} mode (cross hair cursor), the mouse is used to directly control
 the aircraft in the absence of a joystick. Moving the mouse controls the aileron (left/right)
 and elevator (forwards/backwards). Holding the left mouse button down allows control of the rudder (left/right), while holding the middle mouse button controls throttle (forwards/backwards). The scrollwheel controls
 elevator trim. Using auto-coordination (\texttt{-$ $-enable-auto-coordination}) is recommended.

 \item In \textbf{view} mode (arrow cursor), you can control the view direction using the mouse.
 Clicking the left mouse button resets the view direction. Holding the middle button down while
 moving the mouse shifts the viewpoint. The scrollwheel may be used to control the field of view.

\end{enumerate}

 \noindent
 Short Reference by M. Basler, S. Buchanan for \FlightGear{} version 2.12.0.\\
 Published under the GPL (\web{http://www.gnu.org/copyleft/gpl.html})

\end{document}

%% Revision 0.4  2001/01/05  michael
%% Initial revision based on getting Started 0.4
%% Revision 0.5 2002/02/15   michael
%% based on \inputs of revised table files from Getting Started 0.5
