%%
%% tutorials.tex -- Flight Gear documentation: Installation and Getting Started
%% Chapter file
%%
%% Written by Michael Basler, started September 1998.
%%
%% Copyright (C) 2002 Michael Basler
%%
%%
%% This program is free software; you can redistribute it and/or
%% modify it under the terms of the GNU General Public License as
%% published by the Free Software Foundation; either version 2 of the
%% License, or (at your option) any later version.
%%
%% This program is distributed in the hope that it will be useful, but
%% WITHOUT ANY WARRANTY; without even the implied warranty of
%% MERCHANTABILITY or FITNESS FOR A PARTICULAR PURPOSE.  See the GNU
%% General Public License for more details.
%%
%% You should have received a copy of the GNU General Public License
%% along with this program; if not, write to the Free Software
%% Foundation, Inc., 675 Mass Ave, Cambridge, MA 02139, USA.
%%
%% $Id: flight.tex,v 0.5 0.6 2002/09/09 michael
%% (Log is kept at end of this file)

%%%%%%%%%%%%%%%%%%%%%%%%%%%%%%%%%%%%%%%%%%%%%%%%%%%%%%%%%%%%%%%%%%%%%%%%%%%%%%%%%%%%%%%%%%%%%%%
\chapter{Tutorials\label{tutorials}}
%%%%%%%%%%%%%%%%%%%%%%%%%%%%%%%%%%%%%%%%%%%%%%%%%%%%%%%%%%%%%%%%%%%%%%%%%%%%%%%%%%%%%%%%%%%%%%%

\section{In-flight Tutorials}

\FlightGear{} contains an in-flight tutorial system, where a simulated instructor provides a
virtual `lesson'. To access tutorials, Select Start Tutorial from the Help menu. Tutorials are
only available on some aircraft.

The tutorial system works particularly well with the Festival TTS system (described above).

\subsection{Cessna 172P tutorials}

A number of flight tutorials exist for the Cessna 172p, based around Half-Moon Bay airport
(KHAF) near San Francisco, provided in the base package. To start the tutorials, select the 
Cessna 172P aircraft, and a starting airport of KHAF, using the wizard, or the command line:

\begin{verbatim}
$ fgfs --aircraft=c172p --airport=KHAF
\end{verbatim}

When the simulator has loaded, select Start Tutorial from the Help menu. You
will then be presented with a list of the tutorials available. Select a tutorial
and press Next. A description of the tutorial is displayed. Press Start
 to start the tutorial.

Each tutorial consists of your instructor giving you a number of directions and
observing how you perform them, If you fail to follow the directions, the
instructor will provide information on how to correct your deviation. At the end
of the tutorial, the number of deviations is shown. The fewer deviations you
make, the better you have performed in the tutotial.

For simplicity, run tutorials with AI aircraft turned off from
the Options item on the AI/ATC menu. Otherwise, ATC messages 
may make it difficult to hear your instructor.

To ask your instructor to repeat any instructions., press `+'. You can pause the
tutorial at any time using the `p' key. To stop the tutorial select Stop Tutorial from the Help menu. 

\section{FlightGear Tutorials}

A range of \FlightGear{} tutorials are available from various sources targetted at different users. 

Eric Brasseur has written a very good tutorial for people completely new to \FlightGear{}, and flying in general. It also describes many basic principles of flight. Accessible here: 

\medskip
\web{http://www.4p8.com/eric.brasseur/flight\_simulator\_tutorial.html}.
\medskip

Secondly, there is an excellent \Index{tutorial} written by David
Megginson \index{Megginson, David} -- being one of the main developers
of \FlightGear{} -- on flying a basic airport circuit specifically
using \FlightGear{}. This document includes a lot of screen shots,
numerical material etc., and is available from

\medskip
\web{http://www.flightgear.org/Docs/Tutorials/circuit}.
\medskip

A tutorial describing cross-country flight in FlightGear can be found in the following section.

\section{Other Tutorials}

There are many non-\FlightGear{} specific \Index{tutorial}s, many of which are applicable. First, a quite
comprehensive manual of this type is the \Index{Aeronautical Information Manual}, published by the \Index{FAA}, and being online
available at

\medskip
\web{http://www.faa.gov/ATPubs/AIM/}.
\medskip
\noindent

This is the Official Guide to Basic Flight Information and ATC Procedures by the FAA. It
contains a lot of information on flight rules, flight safety, navigation, and more. If
you find this a bit too hard reading, you may prefer the \Index{FAA Training Book},

\medskip
\web{http://avstop.com/AC/FlightTraingHandbook/},
\medskip
\noindent

which covers all aspects of flight, beginning with the theory of flight and the working
of airplanes, via procedures like takeoff and landing up to emergency situations. This is
an ideal reading for those who want to learn some basics on flight but don't (yet) want
to spend bucks on getting a costly paper pilot's handbook.

While the handbook mentioned above is an excellent introduction on \Index{VFR} (visual
flight rules), it does not include flying according to \Index{IFR} (instrument flight
rules). However, an excellent introduction into navigation and flight according to
Instrument Flight Rules written by Charles Wood\index{Wood, Charles} can be found at

\web{http://www.navfltsm.addr.com/}.

Another comprehensive but yet readable text is John Denker's\index{Denker, John}
''\Index{See how it flies}'', available at
\medskip

\web{http://www.monmouth.com/~jsd/how/htm/title.html}.
\medskip

 \noindent
This is a real online text book, beginning with Bernoulli's principle, drag and power,
and the like, with the later chapters covering even advanced aspects of VFR as well as
IFR flying


%% Revision 0.00  1998/09/08  michael
%% Initial revision for version 0.53.
%% Revision 0.01  1998/09/20  michael
%% several extensions and corrections, added Fig.1.
%% revision 0.10  1998/10/01  michael
%% final proofreading for release
%% revision 0.11  1998/11/01  michael
%% Complete revision of keyborad controls, interesting places
%% revision 0.12  1999/03/07  michael
%% Corrected rudder key
%% revision 0.20  1999/06/04  michael
%% HUD completely rewritten, added panel section with picture, and menu section
%% updated keystrokes
%% revision 0.3 2000/04/20 michael
%% again updated and added keystrokes
%% revised menu entries
%% picture of new panel and re-written panel section
%% added mouse control section
%% Updated many keys, notably autopilot related, added two new tables
%% revision 0.4 2001/05/12 michael
%% updated/added many keystrikes, updated/added panel description
%% (radio stack etc.), new panel pic, panel before HUD now
%% short description of VOR/NDB
%% revision 0.41 2001/01/01 michael
%% added section on flight school material
%% added hints to user configurable *.xml files
%% revision 0.5 2002/01/01 michael
%% revised all changed keybindings now mostly read off of keyboard.xml
%% restructured tables more logically and put into separate files 
%% for inclusion in Short Reference
%% New panel picture and revised descirption of panel according to new features
%% New HUD picture
%% revision 0.6 2002/09/05 michael
%% Several corrections/tweaks in plus renumbering of tables
%% Tweaks in menu entries
%% Added 3D cockpit picture
%% Changing numbers in radios
%% revision 0.7 2005/10/29 Stuart Buchanan
%% Re-ordering for inclusion of Cross Country tutorial
