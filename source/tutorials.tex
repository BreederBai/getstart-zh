%%
%% tutorials.tex -- Flight Gear documentation: The FlightGear Manual
%% Chapter file
%%
%% Written by Michael Basler, started September 1998.
%%
%% Copyright (C) 2002 Michael Basler
%%
%%
%% This program is free software; you can redistribute it and/or
%% modify it under the terms of the GNU General Public License as
%% published by the Free Software Foundation; either version 2 of the
%% License, or (at your option) any later version.
%%
%% This program is distributed in the hope that it will be useful, but
%% WITHOUT ANY WARRANTY; without even the implied warranty of
%% MERCHANTABILITY or FITNESS FOR A PARTICULAR PURPOSE.  See the GNU
%% General Public License for more details.
%%
%% You should have received a copy of the GNU General Public License
%% along with this program; if not, write to the Free Software
%% Foundation, Inc., 675 Mass Ave, Cambridge, MA 02139, USA.
%%
%% $Id: flight.tex,v 0.5 0.6 2002/09/09 michael
%% (Log is kept at end of this file)

%%%%%%%%%%%%%%%%%%%%%%%%%%%%%%%%%%%%%%%%%%%%%%%%%%%%%%%%%%%%%%%%%%%%%%%%%%%%%%%%%%%%%%%%%%%%%%%
\chapter{Tutorials\label{tutorials}}
%%%%%%%%%%%%%%%%%%%%%%%%%%%%%%%%%%%%%%%%%%%%%%%%%%%%%%%%%%%%%%%%%%%%%%%%%%%%%%%%%%%%%%%%%%%%%%%

If you are new to flying, an advanced simulator such as \FlightGear{} can seem
daunting: You are presented with a cockpit of an aircraft with little 
information on how to fly it.

In real life, when learning to fly you would have an instructor sitting next to
you to teach you how to fly and keep you safe.

While we cannot provide a personal instructor for every virtual pilot, there are
a number of tutorials available that you can follow to become a proficient
virtual pilot.

\section{In-flight Tutorials}

\FlightGear{} contains an in-flight tutorial system, where a simulated 
instructor provides a virtual `lesson'. These vary between aircraft from
 simple tutorials teaching you how to start the engines on the aircraft,
to full lessons teaching you how to fly for the first time.
To access tutorials, Select Start Tutorial from the Help menu. 

The tutorial system works particularly well with the Festival TTS system 
(described above).

For simplicity, run tutorials with AI aircraft turned off from
the Options item on the AI/ATC menu. Otherwise, ATC messages 
may make it difficult to hear your instructor.

Each tutorial consists of a number of discrete steps which you must complete.
Your instructor will provide directions on how to complete each step, and
observer how you perform them, providing additional guidance if required.

Within a tutorial, to ask your instructor to repeat any instructions, press 
`+'. You can pause the tutorial at any time using the `p' key. To stop the 
tutorial select Stop Tutorial from the Help menu. 

\subsection{Cessna 172P tutorials}

If this is your first time flying, a number of tutorials exist for the
Cessna 172P designed to teach you the basics of flight, in a similar way to a
real flight school. The tutorials are based around Half-Moon Bay (KHAF) and
Livermore Municipal (KLVK) airports near San Francisco. Both these airports are
provided in the base package. To start the tutorials, select the Cessna 172P
aircraft, and a starting airport of KHAF or KLVK, using the wizard, or the
command line:

\begin{verbatim}
$ fgfs --aircraft=c172p --airport=KHAF
\end{verbatim}

When the simulator has loaded, select Start Tutorial from the Help menu. You
will then be presented with a list of the tutorials available. Select a tutorial
and press Next. A description of the tutorial is displayed. Press Start
 to start the tutorial.

\section{FlightGear Tutorials}

The following chapters provide \FlightGear{} specific tutorials
to take the budding aviator from their first time in an aircraft to flying in 
the clouds, relying on their instruments for navigation. If you have never flown
a small aircraft before, following the tutorials provides an excellent 
introduction to flight.

Outside of this manual, there is an excellent tutorial written by David
Megginson \index{Megginson, David} -- being one of the main developers
of \FlightGear{} -- on flying a basic airport circuit specifically
using \FlightGear{}. This document includes a lot of screen shots,
numerical material etc., and is available from

\medskip
\web{http://www.flightgear.org/Docs/Tutorials/circuit}.
\medskip

\section{Other Tutorials}

There are many non-\FlightGear{} specific \Index{tutorial}s, many of which are 
applicable. First, a quite comprehensive manual of this type is the 
\Index{Aeronautical Information Manual}, published by the \Index{FAA}, 
and available at

\medskip
\web{http://www.faa.gov/ATPubs/AIM/}.
\medskip
\noindent

This is the Official Guide to Basic Flight Information and ATC Procedures by 
the FAA. It contains a lot of information on flight rules, flight safety, 
navigation, and more. If you find this a bit too hard work, you may prefer 
the \Index{FAA Training Book},

\medskip
\web{http://avstop.com/AC/FlightTraingHandbook/},
\medskip
\noindent

which covers all aspects of flight, beginning with the theory of flight and the
working of airplanes, via procedures like takeoff and landing up to emergency 
situations. This is an ideal reading for those who want to learn some basics on 
flight but don't (yet) want to spend bucks on getting a costly paper pilot's 
handbook.

While the handbook mentioned above is an excellent introduction on \Index{VFR} 
(Visual Flight Rules), it does not include flying according to \Index{IFR} 
(Instrument Flight Rules). However, an excellent introduction into navigation 
and flight according to Instrument Flight Rules written by Charles Wood
\index{Wood, Charles} can be found at

\web{http://www.navfltsm.addr.com/}.

Another comprehensive but yet readable text is John Denker's\index{Denker, John}
''\Index{See how it flies}'', available at
\medskip

\web{http://www.monmouth.com/~jsd/how/htm/title.html}.
\medskip

 \noindent
This is a real online text book, beginning with Bernoulli's principle, drag and
power, and the like, with the later chapters covering even advanced aspects of 
VFR as well as IFR flying

%% Revision 0.00  1998/09/08  michael
%% Initial revision for version 0.53.
%% Revision 0.01  1998/09/20  michael
%% several extensions and corrections, added Fig.1.
%% revision 0.10  1998/10/01  michael
%% final proofreading for release
%% revision 0.11  1998/11/01  michael
%% Complete revision of keyborad controls, interesting places
%% revision 0.12  1999/03/07  michael
%% Corrected rudder key
%% revision 0.20  1999/06/04  michael
%% HUD completely rewritten, added panel section with picture, and menu section
%% updated keystrokes
%% revision 0.3 2000/04/20 michael
%% again updated and added keystrokes
%% revised menu entries
%% picture of new panel and re-written panel section
%% added mouse control section
%% Updated many keys, notably autopilot related, added two new tables
%% revision 0.4 2001/05/12 michael
%% updated/added many keystrikes, updated/added panel description
%% (radio stack etc.), new panel pic, panel before HUD now
%% short description of VOR/NDB
%% revision 0.41 2001/01/01 michael
%% added section on flight school material
%% added hints to user configurable *.xml files
%% revision 0.5 2002/01/01 michael
%% revised all changed keybindings now mostly read off of keyboard.xml
%% restructured tables more logically and put into separate files 
%% for inclusion in Short Reference
%% New panel picture and revised descirption of panel according to new features
%% New HUD picture
%% revision 0.6 2002/09/05 michael
%% Several corrections/tweaks in plus renumbering of tables
%% Tweaks in menu entries
%% Added 3D cockpit picture
%% Changing numbers in radios
%% revision 0.7 2005/10/29 Stuart Buchanan
%% Re-ordering for inclusion of Cross Country tutorial
