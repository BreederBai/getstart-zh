%%
%% getstart.tex -- Flight Gear documentation: The FlightGear Manual
%% Chapter file
%%
%% Written by Michael Basler % Bernhard Buckel, starting September 1998.
%%
%% Copyright (C) 2002 Michael Basler
%%                  & Bernhard Buckel
%%
%% This program is free software; you can redistribute it and/or
%% modify it under the terms of the GNU General Public License as
%% published by the Free Software Foundation; either version 2 of the
%% License, or (at your option) any later version.
%%
%% This program is distributed in the hope that it will be useful, but
%% WITHOUT ANY WARRANTY; without even the implied warranty of
%% MERCHANTABILITY or FITNESS FOR A PARTICULAR PURPOSE.  See the GNU
%% General Public License for more details.
%%
%% You should have received a copy of the GNU General Public License
%% along with this program; if not, write to the Free Software
%% Foundation, Inc., 675 Mass Ave, Cambridge, MA 02139, USA.
%%
%% $Id: missed.tex,v 0.6 2002/09/09 michael
%% (Log is kept at end of this file)

%%%%%%%%%%%%%%%%%%%%%%%%%%%%%%%%%%%%%%%%%%%%%%%%%%%%%%%%%%%%%%%%%%%%%%%%%%%%%%%%%%%%%%%%%%%%%%%
\IfLanguageName{english}{
\chapter{Missed approach: If anything refuses to work}
}{}
\IfLanguageName{french}{
\chapter{Approche manqu\'{e}e : si rien ne fonctionne}
}{}
\label{missed}
%%%%%%%%%%%%%%%%%%%%%%%%%%%%%%%%%%%%%%%%%%%%%%%%%%%%%%%%%%%%%%%%%%%%%%%%%%%%%%%%%%%%%%%%%%%%%%%

\IfLanguageName{english}{
\markboth{\thechapter.\hspace*{1mm} MISSED APPROACH}{\thesection\hspace*{1mm} ???}
}{}
\IfLanguageName{french}{
\markboth{\thechapter.\hspace*{1mm} APPROCHE MANQUEE}{\thesection\hspace*{1mm} ???}
}{}

\IfLanguageName{english}{
In the following section, we tried to sort some \Index{problems} according to operating system,
but if you encounter a problem, it may be a wise idea to look beyond ``your'' operating system -- just in case. If you are experiencing problems, we would strongly advise you to first check the \Index{FAQ} maintained by Cameron Moore\index{Moore Cameron} at
}{}
\IfLanguageName{french}{
Dans la section suivante, nous avons essay\'{e} de trier quelques \Index{probl\`{e}mes} qui peuvent \^{e}tre rencontr\'{e}s, en fonction des syst\`{e}mes d'exploitation. Cependant, si jamais vous rencontrez un probl\`{e}me, sachez qu'il peut parfois \^{e}tre une bonne id\'{e}e de regarder plus loin que ``votre'' syst\`{e}me d'exploitation - au cas o\`{u}. Si vous rencontrez des difficult\'{e}s, nous vous recommandons vivement de consulter en premier lieu la \Index{FAQ} maintenue par Cameron Moore\index{Moore Cameron} \`{a} l'adresse suivante :
}{}

\medskip

\web{http://wiki.flightgear.org/Frequently\_asked\_questions}.
\medskip

\IfLanguageName{english}{
Moreover, the source code contains a directory \texttt{docs-mini} containing numerous
ideas on and solutions to special problems. This is also a good place to go for further reading.
}{}
\IfLanguageName{french}{
De plus, le code source comporte un r\'{e}pertoire \texttt{docs-mini} qui contient de nombreuses id\'{e}es
et solutions pour des probl\`{e}mes sp\'{e}cifiques. Il s'agit donc \'{e}galement d'un bon endroit pour trouver d'avantage d'informations.
}{}

%%%%%%%%%%%%%%%%%%%%%%%%%%%%%%%%%%%%%%%%%%%%%%%%%%%%%%%%%%%%%%%%%%%%%%%%%%%%%%%%%%%%%%%%%%%%%%%
\IfLanguageName{english}{
\section{FlightGear Problem Reports}\index{problem report}
}{}
\IfLanguageName{french}{
\section{Signaler des probl\`{e}mes relatifs \`{a} FlightGear}\index{signaler un probl\`{e}me}
}{}
%%%%%%%%%%%%%%%%%%%%%%%%%%%%%%%%%%%%%%%%%%%%%%%%%%%%%%%%%%%%%%%%%%%%%%%%%%%%%%%%%%%%%%%%%%%%%%%

\IfLanguageName{english}{
The best place to look for help is generally the \Index{mailing lists}, specifically the  \textbf{[Flightgear-User]} mailing list. If you happen to be running a Git version of \FlightGear{}, you may want to subscribe to the \textbf{[Flightgear-Devel]} list. Instructions for subscription can be found at
}{}
\IfLanguageName{french}{
Le meilleur endroit pour obtenir de l'aide est g\'{e}n\'{e}ralement de faire appel aux \Index{listes de diffusion}, et tout particuli\`{e}rement \`{a} la liste de diffusion \textbf{[Flightgear-User]}, ainsi que les \Index{forums}. Si jamais vous utilisez une version Git de \FlightGear{}, vous pourriez vouloir vous inscrire \`{a} la liste de diffusion \textbf{[Flightgear-Devel]}. Les informations pour s'y inscrire peuvent \^{e}tre trouv\'{e}es \`{a} l'adresse :
}{}

 \medskip
\web{http://www.flightgear.org/mail.html}.
 \medskip

\noindent
\IfLanguageName{english}{
It's often the case that someone has already dealt with the issue you're dealing with, so it may be worth your time to search the mailing list archives at
}{}
\IfLanguageName{french}{
Bien souvent, vous n'\^{e}tes pas le premier \`{a} rencontrer ce type de difficult\'{e}. Donc une recherche sur les archives des listes de diffusion devrait vous permettre de trouver une solution rapide. Ces archives peuvent \^{e}tre consult\'{e}es \`{a} l'adresse :
}{}

 \medskip

\web{http://sourceforge.net/mailarchive/forum.php?forum\_name=flightgear-users}

\web{http://sourceforge.net/mailarchive/forum.php?forum\_name=flightgear-devel}.
 \medskip

\noindent

\IfLanguageName{english}{
You should also consider searching the \Index{FlightGear forums} for help, instructions and archives at
}{}
\IfLanguageName{french}{
Vous devriez \'{e}galement consid\'{e}rer visiter les \Index{forums FlightGear} pour rechercher de l'aide, des instructions et des archives \`{a} l'adresse :
}{}

 \medskip
\web{http://www.flightgear.org/forums}.
 \medskip

\IfLanguageName{english}{
There are numerous developers and users reading those lists and forums, so questions are generally answered. However, messages of the type
\textit{FlightGear does not compile on my system. What shall I do?}
 \noindent
are hard to answer without any further detail given, aren't they? Here are some things to consider including in your message when you report a problem:
}{}
\IfLanguageName{french}{
De nombreux d\'{e}veloppeurs et utilisateurs lisent ces listes et forums, donc les questions trouvent g\'{e}n\'{e}ralement une r\'{e}ponse. Cependant, avouez qu'il est difficile de r\'{e}pondre \`{a} des messages du type :
\textit{Je n'arrive pas \`{a} compiler FlightGear sur mon syst\`{e}me, que dois-je faire ?}
 \noindent
si vous ne donnez pas plus de d\'{e}tails, non ? Voici donc quelques \'{e}l\'{e}ments qu'il serait bon d'inclure dans votre message lorsque vous signalez un probl\`{e}me :
}{}

 \medskip

\begin{itemize}
\IfLanguageName{english}{
\item \textbf{Operating system:} (Linux Fedora 17\ldots/Windows Seven 64 bits\ldots)
\item \textbf{Computer:} (Pentium Dual Core, 2,3GHz\ldots)
\item \textbf{Graphics board/chip:} (ATI Radeon HD 770 XT/NVidia GeForce GTX 590\ldots)
\item \textbf{Compiler/version:} (GCC version 4.6.3\ldots)
\item \textbf{Versions of relevant libraries:} (PLIB 1.8.5, OpenSceneGraph 3.0.1\ldots)
\item \textbf{Type of problem:} (Linker dies with message\ldots)
\item \textbf{Steps to recreate the problem:} Start at KSFO, turn off brakes \ldots
}{}
\IfLanguageName{french}{
\item \textbf{Syst\`{e}me d'exploitation :} (Linux Fedora 17\ldots/Windows Seven 64 bits\ldots)
\item \textbf{Ordinateur :} (Pentium Dual Core, 2,3 GHz\ldots)
\item \textbf{Carte graphique/processeur :} (ATI Radeon HD 770 XT/Nvidia GeForce GTX 590\ldots)
\item \textbf{Compilateur/version :} (GCC version 4.6.3\ldots)
\item \textbf{Versions des librairies concern\'{e}es :} (PLIB 1.8.5, OpenSceneGraph 3.0.1\ldots)
\item \textbf{Type de probl\`{e}me :} (Le compilateur s'arr\^{e} avec le message suivant\ldots)
\item \textbf{Etapes pour reproduire le probl\`{e}me :} D\'{e}marrer \`{a} KSFO, l\^{a}cher les freins\ldots
}{}
\end{itemize}

\IfLanguageName{english}{
For getting a trace of the output which \FlightGear{} produces, the following command may come in handy (may need to be modified on some OSs or may not work on others at all, though):
}{}
\IfLanguageName{french}{
Afin d'obtenir une trace de la sortie que \FlightGear{} produit, la commande suivante peut s'av\'{e}rer utile (elle devra \'{e}ventuellement \^{e}tre adapt\'{e}e sur certains syst\`{e}mes d'exploitation ou peut ne pas fonctionner du tout sur d'autres, d'ailleurs) :
}{}

\medskip

\texttt{\%FG$\underline{~}$ROOT/BIN/fgfs >log.txt 2>\&1}
\medskip

\IfLanguageName{english}{
\textbf{One final remark:} Please avoid posting binaries to these lists or forums! List subscribers are widely distributed, and some users have low bandwidth and/or metered connections. Large messages may be rejected by the mailing list administrator. Thanks.
}{}
\IfLanguageName{french}{
\textbf{Une derni\`{e}re petite remarque :} Merci d'essayer d'\'{e}viter de poster du code binaire sur ces forums ou sur ces listes ! Il y a de nombreux abonn\'{e}s et personnes consultant ces informations, et certains disposent de bandes passantes limit\'{e}es et/ou factur\'{e}es. Des messages trop volumineux pourraient \^{e}tre refus\'{e}s par l'administrateur des listes de diffusion. Merci.
}{}

%%%%%%%%%%%%%%%%%%%%%%%%%%%%%%%%%%%%%%%%%%%%%%%%%%%%%%%%%%%%%%%%%%%%%%%%%%%%%%%%%%%%%%%%%%%%%%%
\IfLanguageName{english}{
\section{General problems}\index{problems!general}
}{}
\IfLanguageName{french}{
\section{Probl\`{e}mes g\'{e}n\'{e}raux}\index{probl\`{e}mes!g\'{e}n\'{e}raux}
}{}
%%%%%%%%%%%%%%%%%%%%%%%%%%%%%%%%%%%%%%%%%%%%%%%%%%%%%%%%%%%%%%%%%%%%%%%%%%%%%%%%%%%%%%%%%%%%%%%
\begin{itemize}

\IfLanguageName{english}{
\item{\FlightGear{} runs SOOO slow.}\\
 If \FlightGear{} says it's running with something like 1 fps
 (frame per second) or below you typically don't have working hardware
 \Index{OpenGL} support. There may be several reasons for this. First,
 there may be no OpenGL hardware drivers available for older
 cards. In this case it is highly recommended to get a new board.

 Second, check if your drivers are properly installed. Several
 cards need additional OpenGL support drivers besides the
 ``native'' windows ones.

\item{Either \texttt{configure} or \texttt{make} dies with not found \PLIB{} headers or
 libraries.}\\
  Make sure you have the latest version of \PLIB{} ($>$ version 1.8.4) compiled and installed.
  Its headers like \texttt{pu.h} have to be under \texttt{/usr/include/plib} and its libraries, like \texttt{libplibpu.a} should be under \texttt{/lib}. Double check there are no stray \PLIB{} headers/libraries sitting elsewhere!

  Besides, check carefully the error messages of \texttt{configure}. In several cases, it
  says what is missing.
}{}
\IfLanguageName{french}{
\item{\FlightGear{} fonctionne siiiiiii lentement.}\\
 Si \FlightGear{} fonctionne, disons \`{a} quelque chose comme une image par seconde
 (\textit{fps, frame per second}), ou moins, c'est que vous n'avez pas de mat\'{e}riel prenant en charge
 \Index{OpenGL}. Il peut y avoir plusieurs raisons \`{a} cela. Tout d'abord, il peut effectivement n'y avoir
 aucun pilote mat\'{e}riel OpenGL disponible pour des cartes anciennes. Dans ce cas, il vous est vivement
 recommand\'{e} d'envisager l'achat d'une nouvelle carte.

 Ensuite, v\'{e}rifiez que vos pilotes sont correctement install\'{e}s. Plusieurs cartes n\'{e}cessites des pilotes
 compl\'{e}mentaires pour faire fonctionner OpenGL en compl\'{e}ment des pilotes ``natifs'' du gestionnaire de fen\^{e}tres.

\item{\texttt{configure} ou \texttt{make} \'{e}chouent car ils ne trouvent pas les en-t\^{e}tes ou biblioth\`{e}ques \PLIB{}.}\\
  Soyez certains de disposer de la derni\`{e}re version de \PLIB{} ($>$ version 1.8.4) compil\'{e}e et install\'{e}e.
  Ses en-t\^{e}tes comme \texttt{pu.h} doivent se situer dans le r\'{e}pertoire \texttt{/usr/include/plib} et ses biblioth\`{e}ques,
  comme\texttt{libplibpu.a}, dans le r\'{e}pertoire \texttt{/lib}. V\'{e}rifiez \`{a} nouveaux qu'il n'y a pas d'autre en-t\^{e}tes ou biblioth\`{e}ques \PLIB{} parasites pr\'{e}sentes ailleurs !

  Enfin, v\'{e}rifiez attentivement les messages d'erreur(s) de \texttt{configure}. Dans de nombreux cas, ils donnent des informations pr\'{e}cieuses sur les \'{e}l\'{e}ments manquants.
}{}

\end{itemize}

%%%%%%%%%%%%%%%%%%%%%%%%%%%%%%%%%%%%%%%%%%%%%%%%%%%%%%%%%%%%%%%%%%%%%%%%%%%%%%%%%%%%%%%%%%%%%%%
\IfLanguageName{english}{
\section{Potential problems under Linux}\index{problems!Linux}
}{}
\IfLanguageName{french}{
\section{Probl\`{e}mes potentiels sous Linux}\index{probl\`{e}mes!Linux}
}{}
%%%%%%%%%%%%%%%%%%%%%%%%%%%%%%%%%%%%%%%%%%%%%%%%%%%%%%%%%%%%%%%%%%%%%%%%%%%%%%%%%%%%%%%%%%%%%%%
\IfLanguageName{english}{
Since we don't have access to all possible flavors of Linux distributions, here are some
thoughts on possible causes of problems. (This Section includes contributions by Kai
Troester.)\index{Troester, Kai}
}{}
\IfLanguageName{french}{
Comme nous n'avons pas acc\`{e}s \`{a} toutes les versions possibles des distributions Linux, voici quelques-
unes des causes possibles de probl\`{e}mes sous cet environnement. (Cette section comprend des contributions
de Kai Troester.)\index{Troester, Kai}
}{}

\begin{itemize}

\IfLanguageName{english}{
\item{Wrong library versions}\\
  This is a rather common cause of grief especially when you prefer to
  install the libraries needed by \FlightGear{} by hand. Be sure that
  especially the Mesa library contains support for the
  \Index{3DFX} board and that \Index{GLIDE} libraries are installed and can be
  found. If a \texttt{ldd \`which fgfs\`} complains about missing
  libraries you are in trouble.

  You should also be sure to \emph{always} keep the \emph{latest} version
  of \PLIB{} on your system. Lots of people have
  failed miserably to compile \FlightGear{} just because of an outdated
  plib.
}{}
\IfLanguageName{french}{
\item{Mauvaises versions des biblioth\`{e}ques}\\
  C'est une origine assez commune de griefs tout sp\'{e}cialement lorsque vous
  pr\'{e}f\'{e}rez installer les biblioth\`{e}ques n\'{e}cessaires \`{a} \FlightGear{}
  \`{a} la main. V\'{e}rifiez bien que, en particulier, la biblioth\`{e}que Mesa comprend bien
  la prise en charge de la carte \Index{3DFX} et que les biblioth\`{e}ques \Index{GLIDE} sont install\'{e}es et qu'elles peuvent \^{e}tre
  trouv\'{e}es. Si un \texttt{ldd \`which fgfs\`} se plaint de biblioth\`{e}ques manquantes, alors vous aurez des difficult\'{e}s.

  Soyez \'{e}galement certain de \emph{toujours} disposer de la \emph{derni\`{e}re} version
  de \PLIB{} sur votre syst\`{e}me. De nombreuses personnes ont lamentablement \'{e}chou\'{e} \`{a} compiler \FlightGear{} simplement
  \`{a} cause d'une version trop ancienne de plib.
}{}

\IfLanguageName{english}{
\item{Missing \Index{permissions}}\\
 In case you are using \Index{XFree86} before release 4.0 the \FlightGear{} binary may need to be  setuid root in order to be capable of  accessing some accelerator boards (or a special kernel module as described earlier in this document) based on 3DFX chips.
  So you can either issue a

  \texttt{chown root.root /usr/local/bin/fgfs ;}\\
  \texttt{chmod 4755 /usr/local/bin/fgfs}

  to give the \FlightGear{} binary the proper rights or install
  the 3DFX module. The latter is the ``clean''
  solution and strongly recommended!
}{}
\IfLanguageName{french}{
\item{Droits \Index{manquants}}\\
 Si vous utilisez \Index{XFree86} d'une version ant\'{e}rieure \`{a} 4.0, le binaire \FlightGear{} peut n\'{e}cessiter de disposer du bit setuid
 root afin de pouvoir acc\'{e}der \`{a} certaines cartes d'acc\'{e}l\'{e}ration (ou un module sp\'{e}cial du noyau comme d\'{e}crit pr\'{e}c\'{e}demment dans ce document) bas\'{e}es sur des puces 3DFX.
  Vous pouvez alors essayer un :

  \texttt{chown root.root /usr/local/bin/fgfs ;}\\
  \texttt{chmod 4755 /usr/local/bin/fgfs}

  pour donner au binaire \FlightGear{} les droits appropri\'{e}s ou installer le module 3DFX. Cette derni\`{e}re solution \'{e}tant la plus  ``propre'' et donc fortement recommand\'{e}e !
}{}

\IfLanguageName{english}{
\item{Non-default install options}\\
  \FlightGear{} will display a lot of diagnostics while starting up.
  If it complains about bad looking or missing files, check that you
  installed them in the way they are supposed to be installed (i.e\. with the latest
  version and in the proper location). The canonical location \FlightGear{}
  wants its data files under \texttt{/usr/local/lib}.
  Be sure to grab the latest versions of everything that might be needed!
}{}
\IfLanguageName{french}{
\item{Options d'installation particuli\`{e}res}\\
  \FlightGear{} affichera un nombre important d'informations de diagnostic lors de son lancement.
  S'il se plaint de fichiers incorrects ou manquants, v\'{e}rifiez que vous les avez install\'{e}s de
  la mani\`{e}re dont ils sont suppos\'{e}s \^{e}tre install\'{e}s (c'est-\`{a}-dire dans leur version
  la plus r\'{e}cente et \`{a} l'emplacement pr\'{e}vu). L'emplacement canonique de \FlightGear{}
  attend ses donn\'{e}es dans le r\'{e}pertoire \texttt{/usr/local/lib}.
  Soyez certain de r\'{e}cup\'{e}rer les derni\`{e}res versions de tout ce qui peut \^{e}tre n\'{e}cessaire !
}{}

\IfLanguageName{english}{
\item{Compile problems in general}\\
  Make sure you have the latest (official) version of gcc. Old versions of
  gcc are a frequent source of trouble! On the other hand, some versions
  of the RedHat 7.0 reportedly have certain problems compiling \FlightGear{} as they include
  a preliminary version of gcc.
}{}
\IfLanguageName{french}{
\item{Probl\`{e}mes plus g\'{e}n\'{e}raux de compilation}\\
  Soyez certain de disposer de la derni\`{e}re version officielle de gcc. D'anciennes version de gcc
  la cause de nombreux probl\`{e}mes ! D'un autre c\^{o}t\'{e}, certaines version de RedHat 7.0 sont connues pour avoir
  des difficult\'{e}s de compilation de \FlightGear{}, car elles incluent une version pr\'{e}liminaire de gcc.
}{}

 \end{itemize}

%%%%%%%%%%%%%%%%%%%%%%%%%%%%%%%%%%%%%%%%%%%%%%%%%%%%%%%%%%%%%%%%%%%%%%%%%%%%%%%%%%%%%%%%%%%%%%%
\IfLanguageName{english}{
\section{Potential problems under Windows}\index{problems!Windows}
}{}
\IfLanguageName{french}{
\section{Probl\`{e}mes potentiels sous Windows}\index{probl\`{e}mes!Windows}
}{}
%%%%%%%%%%%%%%%%%%%%%%%%%%%%%%%%%%%%%%%%%%%%%%%%%%%%%%%%%%%%%%%%%%%%%%%%%%%%%%%%%%%%%%%%%%%%%%%
\begin{itemize}
\IfLanguageName{english}{
\item{The executable refuses to run.}\\
 You may have tried to start the executable directly either by
 double-clicking \texttt{fgfs.exe} in Windows Explorer or by invoking it
 within a MS-DOS shell. Double-clicking via Explorer never works
 (unless you set the environment variable \texttt{FG\_ROOT}
 in \texttt{autoexec.bat} or otherwise). Rather double-click \texttt{fgrun}.
  For more details, check Chapter~\ref{takeoff}.

 Another cause of grief might be that you did not download the
 most recent versions of the base package files required by \FlightGear{}, or
 you did not download any of them at all. Have a close look
 at this, as the scenery/texture format is still under development and may
 change frequently.  For more details, check Chapter~\ref{prefligh}.

 Next, if you run into trouble at runtime, do not use Windows utilities for unpacking the
 \texttt{.tar.gz}. If you did, try it in the Cygnus shell with \texttt{tar xvfz}
 instead.
}{}
\IfLanguageName{english}{
\item{L'ex\'{e}cutable refuse de se lancer.}\\
 Vous pouvez avoir essay\'{e} de lancer l'ex\'{e}cutable directement en double-cliquant sur \texttt{fgfs.exe}
 dans l'explorateur Windows ou en l'invoquant au travers d'une invite de commandes MS-DOS. Double-cliquer via
 l'explorateur de fonctionne jamais (sauf si vous avez d\'{e}fini la variable d'environnement \texttt{FG\_ROOT}
 dans l'\texttt{autoexec.bat} ou d'une autre mani\`{e}re). Pr\'{e}f\'{e}rez le double-clic sur \texttt{fgrun}.
 Pour plus de d\'{e}tails, consultez le chapitre~\ref{takeoff}.

 Une autre cause de grief peut s'expliquer par le fait que vous n'avez pas t\'{e}l\'{e}charg\'{e} les versions les
 plus r\'{e}centes du paquetage de base n\'{e}cessaires \`{a} \FlightGear{}, ou que vous ne les avez pas t\'{e}l\'{e}charg\'{e}
 du tout. Jetez souvent un \oe{}il \`{a} ceux-ci, car le format des sc\`{e}nes et des textures fait toujours l'objet d'un
 d\'{e}veloppement intensif. Pour plus de d\'{e}tails, reportez-vous au chapitre~\ref{prefligh}.

 Ensuite, si vous rencontrez un probl\`{e}me au d\'{e}marrage, n'utilisez pas les utilitaires Windows pour d\'{e}compresser
 les fichiers \texttt{.tar.gz}. Si vous l'avez fait, essayez de le faire dans l'invite de commandes Cygnus en pr\'{e}f\'{e}rant un \texttt{tar xvfz} \`{a} la place.
}{}

\IfLanguageName{english}{
\item{\FlightGear{} ignores the command line parameters.}\\
 There can be a problem with passing command line options containing a
 ''='' on the command line. Instead create a batch job to include your options and run that instead.
}{}
\IfLanguageName{french}{
\item{\FlightGear{} ignore les param\`{e}tres de ligne de commande.}\\
 Il peut y avoir une difficul\'{e} \`{a} passer des options de ligne de commande contenant un caract\`{e}re ''='' sur la ligne de commande. Pr\'{e}f\'{e}rez plut\^{o}t la cr\'{e}ation d'un fichier \textit{batch} pour y inclure vos options et lancez plut\^{o}t celui-ci.
}{}

\IfLanguageName{english}{
\item{I am unable to build \FlightGear{} under \Index{MSVC}/\Index{MS DevStudio}.}\\
 By default, \FlightGear{} is build with GNU GCC. The Win32 port of GNU GCC is known as
 \Index{Cygwin}. For hints on \textit{Makefiles} required for MSVC or MSC DevStudio have a look into
}{}
\IfLanguageName{french}{
\item{Je ne parviens pas \`{a} compiler \FlightGear{} avec \Index{MSVC}/\Index{MS DevStudio}.}\\
 Par d\'{e}faut, \FlightGear{} est compil\'{e} avec GNU GCC. Le portage Win32 de GNU GCC est connu sous le nom de
 \Index{Cygwin}. Pour obtenir des astuces sur les fichiers \textit{Makefile} n\'{e}cessaires pour MSVC ou MSC DevStudio veuillez consulter :
}{}

  \medskip

 \web{https://www.gitorious.org/fg/flightgear/trees/next}.
  \medskip

 \noindent
\IfLanguageName{english}{
In principle, it should be possible to compile \FlightGear{} with the project files provided with the source code.
}{}
\IfLanguageName{french}{
En princpe, il devrait \^{e}tre possible de compiler \FlightGear{} \`{a} l'aide des fichiers de projet fournis avec le code source.
}{}

\IfLanguageName{english}{
\item{Compilation of \FlightGear{} dies.}\\
 There may be several reasons for this, including true bugs. However, before trying to do
 anything else or report a problem, make sure you have the latest version of the
 \Cygwin{} compiler. In case of doubt, start
 \texttt{setup.exe} anew and download and install the most recent versions of bundles
 as they possibly may have changed.
}{}
\IfLanguageName{french}{
\item{La compilation de \FlightGear{} \'{e}choue.}\\
 Il peut y avoir plusieurs raisons \`{a} cela, y compris l'existence de v\'{e}ritables anomalies. Cependant, avant
 de tenter quoi que ce soit ou de signaler un probl\`{e}me, soyez certain de disposer de la derni\`{e}re version du compilateur
 \Cygwin{}. En cas de doute, lancez \`{a} nouveau \texttt{setup.exe} et t\'{e}l\'{e}chargez et installez la version la plus r\'{e}cente
 de l'ensemble, car il est possible que cette version ait chang\'{e}.
}{}

\end{itemize}


%% revision 0.10  1998/10/01  bernhard
%% added win stuff michael
%% final proofreading for release
%% revision 0.11  1998/11/01  michael
%% Remark on mini-OpenGL drivers, new general Section
%% Access violation error under win32 added
%% Command line problem in win32 added
%% revision 0.12  1999/03/07  bernhard
%% Remark on EGCS compiler
%% revision 0.12  1999/03/07  michael
%% Added Contribution by Kai Troester
%% Reworked Win32 Stuff
%% revision 0.20  1999/06/04  michael
%% added hint to FAQ, gfc problem
%% revision 0.22 XXX
%% added hint on install.exe with Cygnus
%% revision 0.3 2000/04/20 michael
%% hint on revised PLIB paths, removed outdated stuff
%% revision 0.3 2000/04/20 michael
%% Deleted some outdated stuff
%% Added Old PLIB problem
%% revision 0.31 2000/05/01 michael
%% Reworked/Added Achnowledgements
%% Added hint on gunzip/tar from Marc Anderson
%% revision 0.4 2001/05/12 michael
%% added some points (RedHat...), deleted some outdated ones
%% complete rewrite and check still to be done ... sometime
%% revision 0.5 2002/01/01 michael
%% removed several outdated issues
%% Added problem linking MetaKit
%% revision 0.601 2002/09/14 michael
%% activated two links
