%%
%% getstart.tex -- Flight Gear documentation: The FlightGear Manual
%% Chapter file
%%
%% Written by Michael Basler, started September 1998.
%%
%% Copyright (C) 2002 Michael Basler
%%
%%
%% This program is free software; you can redistribute it and/or
%% modify it under the terms of the GNU General Public License as
%% published by the Free Software Foundation; either version 2 of the
%% License, or (at your option) any later version.
%%
%% This program is distributed in the hope that it will be useful, but
%% WITHOUT ANY WARRANTY; without even the implied warranty of
%% MERCHANTABILITY or FITNESS FOR A PARTICULAR PURPOSE.  See the GNU
%% General Public License for more details.
%%
%% You should have received a copy of the GNU General Public License
%% along with this program; if not, write to the Free Software
%% Foundation, Inc., 675 Mass Ave, Cambridge, MA 02139, USA.
%%
%% $Id: prefligh.tex,v 0.6 2002/09/09 michael
%% (Log is kept at end of this file)

%%%%%%%%%%%%%%%%%%%%%%%%%%%%%%%%%%%%%%%%%%%%%%%%%%%%%%%%%%%%%%%%%%%%%%%%%%%%%%%%%%%%%%%%%%%%%%%
\chapter{Preflight: Installing \FlightGear{} \label{prefligh}}
%%%%%%%%%%%%%%%%%%%%%%%%%%%%%%%%%%%%%%%%%%%%%%%%%%%%%%%%%%%%%%%%%%%%%%%%%%%%%%%%%%%%%%%%%%%%%%%

To run \FlightGear{} you need to install the binaries. Once you've done this you may install additional scenery and aircraft if you wish.

Pre-compiled binaries for the latest release are available for

\begin{itemize}
\item Windows - any flavor,
\item Macintosh OSX,
\item Linux,
\item SGI Irix.
\end{itemize}

To download them go to

\medskip
\web{http://www.flightgear.org/Downloads/binary.shtml}
\medskip

and follow the instructions provided on the page.

%%%%%%%%%%%%%%%%%%%%%%%%%%%%%%%%%%%%%%%%%%%%%%%%%%%%%%%%%%%%%%%%%%%%%%%%%%%%%%%%%%%%%%%%%%%%%%%
\section{Installing scenery}\index{scenery!additional}\index{additional scenery}\index{scenery}
%%%%%%%%%%%%%%%%%%%%%%%%%%%%%%%%%%%%%%%%%%%%%%%%%%%%%%%%%%%%%%%%%%%%%%%%%%%%%%%%%%%%%%%%%%%%%%%

Detailed \FlightGear{} scenery is available for the entire world, allowing
you to fly everywhere from the Himalaya mountains to rural Kansas.
The \FlightGear{} base package contains scenery for a small area around San
Francisco, so to fly elsewhere you will need to download additional scenery.

Each piece of scenery is packaged into a compressed archive, or tarball, in
a 10 degree by 10 degree chunk. Each tarball is named after the 10x10 degree
chunk it represents, for example w130n50.tgz.

You can download scenery from a clickable map here:

\medskip
\web{http://www.flightgear.org/Downloads/scenery.html}
\medskip

Alternatively, you can support the \FlightGear{} project by purchasing a complete
set of scenery for the entire world from here:

\medskip
\web{http://www.flightgear.org/cdrom/}
\medskip

Once you have downloaded the tarball onto your computer, you need to find the
\texttt{Scenery} directory of your \FlightGear{} installation.

\begin{itemize}
\item For Windows, this directory is likely to be

\texttt{c:$\backslash$Program Files$\backslash$FlightGear$\backslash$data$\backslash$Scenery}.
\item For Unices, it is usually

\texttt{/usr/local/share/FlightGear/data/Scenery}.
\end{itemize}

To install the scenery, uncompress the tarball into the \texttt{Scenery}
directory. Most operating system provide tools to uncompress tarballs. If you cannot
uncompress the tarball, install an extractor program such as 7-zip
(\web{http://www.7-zip.org/}.

Note that you should not decompress the numbered scenery files inside the tarball like
958402.gz - this will be done by \FlightGear{} on the fly.

Once you have uncompressed the tarball, the \texttt{Terrain} and \texttt{Objects} direcories
will contain additional sub-directories with your new scenery inside.

To use the new scenery, simply select a starting airport within the new scenery.
If you are using the \FlightGear{} Launcher, you will need to press the Refresh
button before you select your airport.

\subsection{MS Windows Vista}
If you are using Windows Vista, you may find that Windows installs downloaded scenery
(and aircraft) to your Virtual Store:

\noindent
{ \footnotesize{\texttt{c:$\backslash$Users$\backslash$(Your
Name)$\backslash$AppData$\backslash$Local$\backslash$VirtualStore$\backslash$Program
Files$\backslash$FlightGear$\backslash$Scenery}}}

If it does this, you need to copy the \texttt{Terrain} and \texttt{Objects}
directories manually to your real \FlightGear{} \texttt{Scenery} directory
as described above.

\subsection{FG\_SCENERY}\index{FG\_SCENERY}

If you would prefer to keep your downloaded scenery separate from the core
installation, you can do so by setting your \texttt{FG\_SCENERY} environmental
variable.

This is where \FlightGear{} looks for scenery files. It consists of a list
of directories that will be searched in order. The directories are separated
by ``:'' on Unix and ``;'' on Windows.

For example, a \texttt{FG\_SCENERY} environmental variable set to

\noindent
\texttt{/home/joebloggs/WorldScenery:/usr/local/share/Flightgear/data/Scenery}

\noindent
searches for scenery in
\noindent
\texttt{/home/joebloggs/WorldScenery}
first, followed by
\noindent
\texttt{/usr/local/share/Flightgear/data/Scenery}.

\subsection{Creating your own scenery}

If you are interested in generating your own scenery, have a look at TerraGear -
the tools that generate the scenery for \FlightGear{}:

\medskip
\web{http://www.terragear.org/}
\medskip

The most actively maintained source tree of the TerraGear toolchain is
co-located at the FlightGear landuse data Mapserver:

\medskip
\web{http://mapserver.flightgear.org/git/gitweb.pl}.
\medskip

%%%%%%%%%%%%%%%%%%%%%%%%%%%%%%%%%%%%%%%%%%%%%%%%%%%%%%%%%%%%%%%%%%%%%%%%%%%%%%%%%%%%%%%%%%%%%%%
\section{Installing aircraft}\index{aircraft!installation}\index{installing aircraft}\label{install_aircraft}
%%%%%%%%%%%%%%%%%%%%%%%%%%%%%%%%%%%%%%%%%%%%%%%%%%%%%%%%%%%%%%%%%%%%%%%%%%%%%%%%%%%%%%%%%%%%%%%

The base \FlightGear{} package contains only a small subset of the aircraft that are available for \FlightGear{}.
Developers have created a wide range of aircraft, from WWII fighters like the Spitfire, to passenger planes like the Boeing 747.

You can download aircraft from

\medskip
\web{http://www.flightgear.org/Downloads/aircraft/index.shtml}
\medskip

Simply download the file and uncompress it into the
\texttt{data/Aircraft} subdirectory of your installation. The
aircraft are downloaded as \texttt{.tar.gz} files. Some computers will download
them as \texttt{.tar.gz.zip} files. If so, simply rename the file to
\texttt{.tar.gz} before uncompressing them. If you are successful, there will be a new
sub-directory in your \texttt{FG\_ROOT/data/Aircraft} directory containing the
aircraft. Next time you run \FlightGear{}, the new aircraft will be available.

%%%%%%%%%%%%%%%%%%%%%%%%%%%%%%%%%%%%%%%%%%%%%%%%%%%%%%%%%%%%%%%%%%%%%%%%%%%%%%%%%%%%%%%%%%%%%%%
\section{Installing documentation}\index{documentation!installation}
%%%%%%%%%%%%%%%%%%%%%%%%%%%%%%%%%%%%%%%%%%%%%%%%%%%%%%%%%%%%%%%%%%%%%%%%%%%%%%%%%%%%%%%%%%%%%%%

Most of the packages named above include the complete \FlightGear{}
documentation including a PDF version of \textit{The FlightGear
Manual} intended for pretty printing using Adobe's Acrobat Reader,
available from
\medskip

\web{http://www.adobe.com/acrobat}
 \medskip

 \noindent
 Moreover, if properly installed, the \.html version can be accessed via
\FlightGear{}'s \texttt{help} menu entry.

Besides, the source code contains a directory \texttt{docs-mini} containing numerous
ideas on and solutions to special problems. This is also a good place for further
reading.

%% Revision 0.00  1998/09/08  michael
%% Initial revision for version 0.53.
%% Revision 0.01  1998/09/20  michael
%% several extensions and corrections
%% revision 0.10  1998/10/01  michael
%% final proofreading for release
%% revision 0.11  1998/11/01  michael
%% support files Section completely re-written
%% revision 0.20  1999/06/04  michael
%% some updates and corrections, corrected links
%% revision 0.3 2000/04/20 michael
%% minor updates, reference to click map
%% revision 0.4 2001/05/12 michael
%% completely re-written, added MacOS, Debian, SGI secions
%% separate sections on global scenery and help
%% revision 0.5 2002/01/01 michael
%% minor tweaks
%% Installing on Mac newly written based on material by Darrell
%% revision 0.6 2002/01/01 michael
%% Added installation of W. Riley's VMap0 scenery
%% Dave Perry edits: p.35 removed "again using a.", p.36 added /Scenery, etc. to the directory
%% structures to match what I thought was intended.
%% revision 0.7 2005/11/10 stuart
%% Update for v0.9.8 installers, removing out of date build information.
%% revision 0.8 2008/06/10 stuart
%% Update for v1.0.0 installers, simplifying instructions
%% Revision 16/10/08: changed "pdf" to "PDF" and removed an erroneous instance of "\." (backslash dot) at the same location.

