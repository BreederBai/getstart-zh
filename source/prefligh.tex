%%
%% getstart.tex -- Flight Gear documentation: The FlightGear Manual
%% Chapter file
%%
%% Written by Michael Basler, started September 1998.
%%
%% Copyright (C) 2002 Michael Basler
%%
%%
%% This program is free software; you can redistribute it and/or
%% modify it under the terms of the GNU General Public License as
%% published by the Free Software Foundation; either version 2 of the
%% License, or (at your option) any later version.
%%
%% This program is distributed in the hope that it will be useful, but
%% WITHOUT ANY WARRANTY; without even the implied warranty of
%% MERCHANTABILITY or FITNESS FOR A PARTICULAR PURPOSE.  See the GNU
%% General Public License for more details.
%%
%% You should have received a copy of the GNU General Public License
%% along with this program; if not, write to the Free Software
%% Foundation, Inc., 675 Mass Ave, Cambridge, MA 02139, USA.
%%
%% $Id: prefligh.tex,v 0.6 2002/09/09 michael
%% (Log is kept at end of this file)

%%%%%%%%%%%%%%%%%%%%%%%%%%%%%%%%%%%%%%%%%%%%%%%%%%%%%%%%%%%%%%%%%%%%%%%%%%%%%%%%%%%%%%%%%%%%%%%
\chapter{Preflight: Installing \FlightGear{}}
\label{prefligh}
%%%%%%%%%%%%%%%%%%%%%%%%%%%%%%%%%%%%%%%%%%%%%%%%%%%%%%%%%%%%%%%%%%%%%%%%%%%%%%%%%%%%%%%%%%%%%%%

To run \FlightGear{} you need to install the binaries. Once you've done this you may install additional scenery and aircraft if you wish.

Pre-compiled binaries for the latest release are available for

\begin{itemize}
\item Windows - any flavor,
\item Mac OS X,
\item Linux.
\end{itemize}

To download them go to

\medskip
\web{http://www.flightgear.org/Downloads/binary.shtml}
\medskip

and follow the instructions provided on the page.

%%%%%%%%%%%%%%%%%%%%%%%%%%%%%%%%%%%%%%%%%%%%%%%%%%%%%%%%%%%%%%%%%%%%%%%%%%%%%%%%%%%%%%%%%%%%%%%
\section{Installing scenery}\index{scenery!additional}\index{additional scenery}\index{scenery}
%%%%%%%%%%%%%%%%%%%%%%%%%%%%%%%%%%%%%%%%%%%%%%%%%%%%%%%%%%%%%%%%%%%%%%%%%%%%%%%%%%%%%%%%%%%%%%%

Detailed \FlightGear{} scenery is available for the entire world, allowing
you to fly everywhere from the Himalaya mountains to rural Kansas.
The \FlightGear{} base package contains scenery for a small area around San
Francisco, so to fly elsewhere you will need to download additional scenery.

Each piece of scenery is packaged into a compressed archive, or tarball, in
a 10 degree by 10 degree chunk. Each tarball is named after the 10x10 degree
chunk it represents, for example w130n50.tgz.

You can download scenery from a clickable map here:

\medskip
\web{http://www.flightgear.org/Downloads/scenery.html}
\medskip

Alternatively, you can support the \FlightGear{} project by purchasing a complete
set of scenery for the entire world from here:

\medskip
\web{http://www.flightgear.org/dvd/}
\medskip

Once you have downloaded the tarball onto your computer, you need to find the
\texttt{Scenery} directory of your \FlightGear{} installation.

\begin{itemize}
\item For Windows, this directory is likely to be

\texttt{c:$\backslash$Program Files$\backslash$FlightGear$\backslash$data$\backslash$Scenery}.
\item For Unices, it is usually

\texttt{/usr/local/share/FlightGear/data/Scenery}.
\item For Mac OS X, it is usually either

\texttt{/Applications/FlightGear.app/Contents/Resources/data/Scenery}.

\end{itemize}

To install the scenery, uncompress the tarball into the \texttt{Scenery}
directory. Most operating system provide tools to uncompress tarballs. If you cannot
uncompress the tarball, install an extractor program such as 7-zip
(\web{http://www.7-zip.org/}).

Note that you should not decompress the numbered scenery files inside the tarball like
958402.gz - this will be done by \FlightGear{} on the fly.

Once you have uncompressed the tarball, the \texttt{Terrain} and \texttt{Objects} directories
will contain additional sub-directories with your new scenery inside.

To use the new scenery, simply select a starting airport within the new scenery.
If you are using the \FlightGear{} Launcher, you will need to press the Refresh
button before you select your airport.

\subsection{MS Windows Vista/7}
If you are using Windows Vista or Windows 7, you may find that Windows installs downloaded scenery
(and aircraft) to your Virtual Store:

\noindent
{ \footnotesize{\texttt{c:$\backslash$Users$\backslash$(Your
Name)$\backslash$AppData$\backslash$Local$\backslash$VirtualStore$\backslash$Program
Files$\backslash$FlightGear$\backslash$Scenery}}}

If it does this, you need to copy the \texttt{Terrain} and \texttt{Objects}
directories manually to your real \FlightGear{} \texttt{Scenery} directory
as described above.

\subsection{Mac OS X \label{sceneryOnMac}}
You may install the downloaded scenery data and aircraft using the GUI launcher. Pressing \textit{Install Add-On data} on the \textit{Advanced Features >> Others} tab opens up the file browser window. Selecting one or more scenery data files will install the scenery data into /Applications/FlightGear.app/Contents/Resources/data/Scenery. Acceptable formats for the scenery data are one of zip, tar.gz, tgz, tar, and extracted folder. If the installation via the GUI launcher failure for some reason, you still have an alternative way to install the data. Opening the data folder by pressing ``Open data folder'' on the Others tab will pop up an Finder window for the data folder. Dragging an aircraft folder to data/Aircraft folder (or a scenery folder to data/Scenery folder) under the data folder will get the job done.

\subsection{FG\_SCENERY}\index{FG\_SCENERY}

If you would prefer to keep your downloaded scenery separate from the core
installation, you can do so by setting your \texttt{FG\_SCENERY} environment
variable.

This is where \FlightGear{} looks for Scenery files. It consists of a list
of directories that will be searched in order. The directories are separated
by ``:'' on Unix (including Mac OS X) and ``;'' on Windows.

For example, on Linux a \texttt{FG\_SCENERY} environment variable set to

\noindent
\texttt{/home/joebloggs/WorldScenery:/usr/local/share/Flightgear/data/Scenery}

\noindent
searches for scenery in

\noindent
\texttt{/home/joebloggs/WorldScenery}

\noindent
first, followed by

\noindent
\texttt{/usr/local/share/Flightgear/data/Scenery}.

\medskip
On Windows, a \texttt{FG\_SCENERY} environment variable set to
\texttt{c:$\backslash$Program Files$\backslash$FlightGear$\backslash$data$\backslash$Scenery;c:$\backslash$data$\backslash$WorldScenery}

\noindent
searches for scenery in

\noindent
\texttt{c:$\backslash$Program Files$\backslash$FlightGear$\backslash$data$\backslash$Scenery}


\noindent
first, followed by

\noindent
\texttt{c:$\backslash$data$\backslash$WorldScenery}

\medskip
Setting up environment variables on different platforms is beyond the scope of this document.

\subsection{Fetch Scenery as you fly}

\FlightGear{} is able to fetch the Scenery as you fly, if you have a permanent
Internet connection at your disposal. Create an empty `working'
directory for \TerraSync{}, writable to the user and point
\FlightGear{} at this directory using the \texttt{FG\_SCENERY} variable
(as explained above). Do \textbf{not} let \TerraSync{} download Scenery
into your pre-installed Scenery directory.

Within \FlightGear{} itself, select the Scenery Download option under the 
Environment menu. Then simply select the directory you created above and enable automatic scenery download.

There is a chicken/egg problem when you first start up in a brand new
area.  \FlightGear{} is expecting the Scenery to be there now but it may
not have been fetched yet. Therefore, once \TerraSync{} has loaded the new
tile (which you can see from the Status section of the Scenery Download
dialog), select the Manual Refresh button to reload the scenery. If this
still causes problems, restart \FlightGear{}.

One major benefit of \TerraSync{} is that it always fetches the latest and
greatest Scenery from the FlightGear World (Custom) Scenery Project and
therefore allows you to pick up incremental updates independant of the
comprehensive World Scenery releases, which are generally synchronized
with \FlightGear{} releases.

\subsubsection{Running \TerraSync{} as a separate tool}

It is also possible to run \TerraSync{} as an external tool. 

On Mac OS X or Windows, just checking ``Download scenery on the fly'' on
the GUI launcher launches \TerraSync{} as a separate process automatically 
for downloading the Scenery around your aircraft, so you don't have to 
specify the atlas option or FG\_SCENERY at all. 

Alternatively you can run the terrasync program directory. It talks to
\FlightGear{} using the `Atlas' protocol, so call \FlightGear{} with the:

\texttt{-$ $-atlas=socket,out,1,localhost,5505,udp}

command line parameter and tell \TerraSync{} about the port number you're
using as well as the respective directory:

\texttt{terrasync -p 5505 -S -d /usr/local/share/TerraSync}
\medskip

Note that \TerraSync{} (when called with the "\texttt{-S}" command line
switch, as recommended) is going to download Scenery via the Subversion
protocol over HTTP. Thus, if your Internet access is set up to use a
HTTP proxy, plase make yourself aware how to configure the "libsvn"
Subversion client for use of a proxy. If you are using Mac OS X 10.5,
The GUI launcher automatically specifies -S if svn is available.
\medskip

\subsection{Creating your own Scenery}

If you are interested in generating your own Scenery, have a look at TerraGear -
the tools that generate the Scenery for \FlightGear{}:

\medskip
\web{http://wiki.flightgear.org/TerraGear}
\medskip

The most actively maintained source tree of the TerraGear toolchain is
co-located at the \FlightGear{} landuse data Mapserver:

\medskip
\web{http://mapserver.flightgear.org/git/gitweb.pl}.
\medskip

%%%%%%%%%%%%%%%%%%%%%%%%%%%%%%%%%%%%%%%%%%%%%%%%%%%%%%%%%%%%%%%%%%%%%%%%%%%%%%%%%%%%%%%%%%%%%%%
\section{Installing aircraft}\index{aircraft!installation}\index{installing aircraft}\label{install_aircraft}
%%%%%%%%%%%%%%%%%%%%%%%%%%%%%%%%%%%%%%%%%%%%%%%%%%%%%%%%%%%%%%%%%%%%%%%%%%%%%%%%%%%%%%%%%%%%%%%

The base \FlightGear{} package contains only a small subset of the aircraft that are available for \FlightGear{}.
Developers have created a wide range of aircraft, from WWII fighters like the Spitfire, to passenger planes like the Boeing 747.

You can download aircraft from

\medskip
\web{http://www.flightgear.org/download/aircraft/}
\medskip

Simply download the file and uncompress it into the \texttt{data/Aircraft}
subdirectory of your installation. The aircraft are downloaded as \texttt{.zip}
files. Once you have uncompressed them, there will be a new sub-directory in your
\texttt{data/Aircraft} directory containing the aircraft. Next time you run
\FlightGear{}, the new aircraft will be available.

On Mac OS X, you may use the GUI launcher to install the downloaded aircraft files as described in section \ref{sceneryOnMac}.
 

%%%%%%%%%%%%%%%%%%%%%%%%%%%%%%%%%%%%%%%%%%%%%%%%%%%%%%%%%%%%%%%%%%%%%%%%%%%%%%%%%%%%%%%%%%%%%%%
\section{Installing documentation}\index{documentation!installation}
%%%%%%%%%%%%%%%%%%%%%%%%%%%%%%%%%%%%%%%%%%%%%%%%%%%%%%%%%%%%%%%%%%%%%%%%%%%%%%%%%%%%%%%%%%%%%%%

Most of the packages named above include the complete \FlightGear{}
documentation including a PDF version of \textit{The FlightGear
Manual} intended for pretty printing using Adobe's Acrobat Reader,
available from
\medskip

\web{http://www.adobe.com/products/acrobat/}
 \medskip

 \noindent
 Moreover, if properly installed, the HTML version can be accessed via
\FlightGear{}'s \texttt{help} menu entry.

Besides, the source code contains a directory \texttt{docs-mini} containing numerous
ideas on and solutions to special problems. This is also a good place for further
reading.

%% Revision 0.00  1998/09/08  michael
%% Initial revision for version 0.53.
%% Revision 0.01  1998/09/20  michael
%% several extensions and corrections
%% revision 0.10  1998/10/01  michael
%% final proofreading for release
%% revision 0.11  1998/11/01  michael
%% support files Section completely re-written
%% revision 0.20  1999/06/04  michael
%% some updates and corrections, corrected links
%% revision 0.3 2000/04/20 michael
%% minor updates, reference to click map
%% revision 0.4 2001/05/12 michael
%% completely re-written, added MacOS, Debian, SGI secions
%% separate sections on global scenery and help
%% revision 0.5 2002/01/01 michael
%% minor tweaks
%% Installing on Mac newly written based on material by Darrell
%% revision 0.6 2002/01/01 michael
%% Added installation of W. Riley's VMap0 scenery
%% Dave Perry edits: p.35 removed "again using a.", p.36 added /Scenery, etc. to the directory
%% structures to match what I thought was intended.
%% revision 0.7 2005/11/10 stuart
%% Update for v0.9.8 installers, removing out of date build information.
%% revision 0.8 2008/06/10 stuart
%% Update for v1.0.0 installers, simplifying instructions
%% Revision 16/10/08: changed "pdf" to "PDF" and removed an erroneous instance of "\." (backslash dot) at the same location.

