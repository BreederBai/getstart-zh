%%
%% getstart.tex -- Flight Gear documentation: The FlightGear Manual
%% Chapter file
%%
%% Written by Michael Basler, started September 1998.
%%
%% Copyright (C) 2002 Michael Basler
%%
%%
%% This program is free software; you can redistribute it and/or
%% modify it under the terms of the GNU General Public License as
%% published by the Free Software Foundation; either version 2 of the
%% License, or (at your option) any later version.
%%
%% This program is distributed in the hope that it will be useful, but
%% WITHOUT ANY WARRANTY; without even the implied warranty of
%% MERCHANTABILITY or FITNESS FOR A PARTICULAR PURPOSE.  See the GNU
%% General Public License for more details.
%%
%% You should have received a copy of the GNU General Public License
%% along with this program; if not, write to the Free Software
%% Foundation, Inc., 675 Mass Ave, Cambridge, MA 02139, USA.
%%
%% $Id: prefligh.tex,v 0.6 2002/09/09 michael
%% (Log is kept at end of this file)

%%%%%%%%%%%%%%%%%%%%%%%%%%%%%%%%%%%%%%%%%%%%%%%%%%%%%%%%%%%%%%%%%%%%%%%%%%%%%%%%%%%%%%%%%%%%%%%
\chapter{Preflight: Installing \FlightGear{} \label{prefligh}}
%%%%%%%%%%%%%%%%%%%%%%%%%%%%%%%%%%%%%%%%%%%%%%%%%%%%%%%%%%%%%%%%%%%%%%%%%%%%%%%%%%%%%%%%%%%%%%%

To run \FlightGear{} you need to install the binaries. Once you've done this you may install additional scenery and aircraft if you wish.

Pre-compiled binaries for the latest release are available for 

\begin{itemize}
\item Windows - any flavor,
\item Macintosh OSX,
\item Linux,
\item SGI Irix.
\end{itemize}

To download them go to 

\medskip
\web{http://www.flightgear.org/Downloads/binary.shtml}
\medskip

and follow the instructions provided on the page.

If you are running on another OS, or wish to compile for yourself, see Appendix~\ref{building}, \textit{Building the plane: Compiling the program}.

%%%%%%%%%%%%%%%%%%%%%%%%%%%%%%%%%%%%%%%%%%%%%%%%%%%%%%%%%%%%%%%%%%%%%%%%%%%%%%%%%%%%%%%%%%%%%%%
\section{Installing scenery}\index{scenery!additional}\index{additional scenery}
%%%%%%%%%%%%%%%%%%%%%%%%%%%%%%%%%%%%%%%%%%%%%%%%%%%%%%%%%%%%%%%%%%%%%%%%%%%%%%%%%%%%%%%%%%%%%%%

The \FlightGear{} base package contains scenery for a small area around San Fransisco, but the entire world is available at a high level of detail, so you will almost certainly wish to install extra scenery at some point.

The scenery is based on \Index{SRTM} elevation data (accurate to 30m in the USA, and 90m elsewhere and) \Index{VMap0} land use data. Additionally, various people have created buildings, bridges and other features to enrich the environment.

You can download scenery in 10 degree by 10 degree chunks from a clickable map at

\medskip
\web{http://www.flightgear.org/Downloads/scenery.html}
\medskip

Curt Olson\index{Olson, Curt} also provides the USA or the entire world along with the latest FlightGear release on DVD from here:

\medskip
\web{http://cdrom.flightgear.org/}
\medskip

If you are interested in generating your own scenerey, have a look at TerraGear - the tools that generate the scenery for \FlightGear{}:

\medskip
\web{http://www.terragear.org/}
\medskip

Finally, an alternative data set was produced by William Riley\index{Riley, William} and is available from here:

\medskip
\web{http://www.randdtechnologies.com/fgfs/newScenery/world-scenery.html}.
\medskip

\medskip
Whatever scenery you shoose to download, it should be kept in a separate directory from the scenery delivered with the binaries. 

To do this, create a \texttt{WorldScenery} directory in the \FlightGear{} data directory, usually 

\texttt{c:$\backslash$Program Files$\backslash$FlightGear$\backslash$data} 

\noindent on Windows or 

\texttt{/usr/local/share/FlightGear/data} 

\noindent on *nix.

Underneath this directory create \texttt{Terrain} and \texttt{Objects} subdirectories. These are used for terrain information and buildings/bridges/structures respectively.

Unpack the downloaded scenery into the \texttt{WorldScenery/Terrain}. Do not de-compress the numbered
scenery files like 958402.gz! This will be done by \FlightGear{} on the fly. 

As an example, consider installation of the scenery package w120n30 containing the Grand
Canyon Scenery into an installation located at 

\texttt{/usr/local/share/FlightGear}.

Once your installation is complete, you'll have the following directories
\medskip

\begin{verbatim}
/usr/local/FlightGear/data/WorldScenery/Objects/
/usr/local/FlightGear/data/WorldScenery/Terrain/w120n30/w112n30
/usr/local/FlightGear/data/WorldScenery/Terrain/w120n30/w112n31
...
/usr/local/FlightGear/data/WorldScenery/Terrain/w120n30/w120n39
\end{verbatim}
\medskip

As well as the scenery itself, objects such as bridges, skyscrapers, radio masts can be downloaded
from \web{http://fgfsdb.stockill.org}. See the website for more information.
\medskip
You can exploit \texttt{FG\_SCENERY} environmental variable or the \texttt{-$ $-fg-scenery={\it path}} command line option if you want to install different scenery sets in parallel or want to have scenery sitting in another place. These are more fully described in Chapter \ref{takeoff}.

%%%%%%%%%%%%%%%%%%%%%%%%%%%%%%%%%%%%%%%%%%%%%%%%%%%%%%%%%%%%%%%%%%%%%%%%%%%%%%%%%%%%%%%%%%%%%%%
\section{Installing aircraft}\index{aircraft!installation}\index{installing aircraft}
%%%%%%%%%%%%%%%%%%%%%%%%%%%%%%%%%%%%%%%%%%%%%%%%%%%%%%%%%%%%%%%%%%%%%%%%%%%%%%%%%%%%%%%%%%%%%%%

The base \FlightGear{} package contains only a small subset of the aircraft that are available for \FlightGear{}. 
Developers have created a wide range of aircraft, from WWII fighters like the Spitfire, to passenger planes like the Boeing 747.

You can download aircraft from 

\medskip
\web{http://www.flightgear.org/Downloads/aircraft/index.shtml}
\medskip

Simply download the file and unpack it into the 
\texttt{FlightGear/data/Aircraft} subdirectory of your installation. The 
aircraft are downloaded as \texttt{.tar.gz} files. Some computers will download
them as \texttt{.tar.gz.zip} files. If so, simply rename the file to 
\texttt{.tar.gz} and unpack. If you are successful, there will be a new 
sub-directory in your \texttt{FG\_ROOT/data/Aircraft} directory containing the
aircraft. Next time you run \FlightGear{}, the new aircraft will be available.
 
%%%%%%%%%%%%%%%%%%%%%%%%%%%%%%%%%%%%%%%%%%%%%%%%%%%%%%%%%%%%%%%%%%%%%%%%%%%%%%%%%%%%%%%%%%%%%%%
\section{Installing documentation}\index{documentation!installation}
%%%%%%%%%%%%%%%%%%%%%%%%%%%%%%%%%%%%%%%%%%%%%%%%%%%%%%%%%%%%%%%%%%%%%%%%%%%%%%%%%%%%%%%%%%%%%%%

Most of the packages named above include the complete \FlightGear{}
documentation including a \.pdf version of \textit{The FlightGear
Manual} intended for pretty printing using Adobe's Acrobat Reader being
available from
 \medskip

\web{http://www.adobe.com/acrobat}
 \medskip

 \noindent
 Moreover, if properly installed, the \.html version can be accessed via
\FlightGear{}'s \texttt{help} menu entry.

Besides, the source code contains a directory \texttt{docs-mini} containing numerous
ideas on and solutions to special problems. This is also a good place for further
reading.

%% Revision 0.00  1998/09/08  michael
%% Initial revision for version 0.53.
%% Revision 0.01  1998/09/20  michael
%% several extensions and corrections
%% revision 0.10  1998/10/01  michael
%% final proofreading for release
%% revision 0.11  1998/11/01  michael
%% support files Section completely re-written
%% revision 0.20  1999/06/04  michael
%% some updates and corrections, corrected links
%% revision 0.3 2000/04/20 michael
%% minor updates, reference to click map
%% revision 0.4 2001/05/12 michael
%% completely re-written, added MacOS, Debian, SGI secions
%% separate sections on global scenery and help
%% revision 0.5 2002/01/01 michael
%% minor tweaks
%% Installing on Mac newly written based on material by Darrell
%% revision 0.6 2002/01/01 michael
%% Added installation of W. Riley's VMap0 scenery
%% Dave Perry edits: p.35 removed "again using a.", p.36 added /Scenery, etc. to the directory
%% structures to match what I thought was intended.
%% revision 0.7 2005/11/10 stuart
%% Update for v0.9.8 installers, removing out of date build information.
