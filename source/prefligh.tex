%%
%% getstart.tex -- Flight Gear documentation: Installation and Getting Started
%% Chapter file
%%
%% Written by Michael Basler, started September 1998.
%%
%% Copyright (C) 2002 Michael Basler (pmb@epost.de)
%%
%%
%% This program is free software; you can redistribute it and/or
%% modify it under the terms of the GNU General Public License as
%% published by the Free Software Foundation; either version 2 of the
%% License, or (at your option) any later version.
%%
%% This program is distributed in the hope that it will be useful, but
%% WITHOUT ANY WARRANTY; without even the implied warranty of
%% MERCHANTABILITY or FITNESS FOR A PARTICULAR PURPOSE.  See the GNU
%% General Public License for more details.
%%
%% You should have received a copy of the GNU General Public License
%% along with this program; if not, write to the Free Software
%% Foundation, Inc., 675 Mass Ave, Cambridge, MA 02139, USA.
%%
%% $Id: prefligh.tex,v 0.6 2002/09/09 michael
%% (Log is kept at end of this file)

%%%%%%%%%%%%%%%%%%%%%%%%%%%%%%%%%%%%%%%%%%%%%%%%%%%%%%%%%%%%%%%%%%%%%%%%%%%%%%%%%%%%%%%%%%%%%%%
\chapter{Preflight: Installing \FlightGear{} \label{prefligh}}
%%%%%%%%%%%%%%%%%%%%%%%%%%%%%%%%%%%%%%%%%%%%%%%%%%%%%%%%%%%%%%%%%%%%%%%%%%%%%%%%%%%%%%%%%%%%%%%
\markboth{\thechapter.\hspace*{1mm} PREFLIGHT}{\thesection\hspace*{1mm} INSTALLING THE
BINARIES}

You can skip this Section if you built \FlightGear{} along the lines described in the
previous Chapter. If you did not and you're jumping in here, your first step will consist in
installing the binaries. At present, there are pre-compiled \Index{binaries} available
for

\begin{itemize}
\item Windows (any flavor),
\item Macintosh OSX,
\item Linux,
\item SGI Irix.
\end{itemize}

%%%%%%%%%%%%%%%%%%%%%%%%%%%%%%%%%%%%%%%%%%%%%%%%%%%%%%%%%%%%%%%%%%%%%%%%%%%%%%%%%%%%%%%%%%%%%%%
\section{Installing the binary distribution on a Windows system}\index{binaries!Windows}
%%%%%%%%%%%%%%%%%%%%%%%%%%%%%%%%%%%%%%%%%%%%%%%%%%%%%%%%%%%%%%%%%%%%%%%%%%%%%%%%%%%%%%%%%%%%%%%

The following supposes you are on a Windows (95/98/Me/NT/2000/XP) \index{Windows} system.
Installing the binaries is quite simple. Go to
 \medskip

 \web{ftp://www.flightgear.org/pub/flightgear/Win32/}
  \medskip

 \noindent
and download the three files \texttt{fgfs-base-X.X.X.zip}, \texttt{fgfs-manual-X.X.X.zip}, and \texttt{fgfs-win32-bin-X.X.X.zip} from
 \medskip

\web{ftp://www.flightgear.org/pub/flightgear/Win32/}
\medskip

 \noindent
to a drive of your choice. Windows XP includes a program for unpacking *.zip files. If you are working under an older version of Windows, we suggest getting Winzip from
\medskip

\web{http://www.winzip.com/}.
\medskip

\noindent
For a free alternative, you may consider \texttt{unzip} from Info-ZIP,
\medskip

http://www.info-zip.org/pub/infozip/
\medskip

 \noindent
Extract the files named above. If you choose drive \texttt{c:} you should find a file
\texttt{runfgfs.bat} under \texttt{c:/Flightgear} now. Double-clicking it should invoke
the simulator.

In case of doubt about the correct directory structure, see the summary at the
end of chapter \ref{building}.

%%%%%%%%%%%%%%%%%%%%%%%%%%%%%%%%%%%%%%%%%%%%%%%%%%%%%%%%%%%%%%%%%%%%%%%%%%%%%%%%%%%%%%%%%%%%%%%
\section{Installing the binary distribution on a Macintosh system}\index{binaries!Macintosh}
%%%%%%%%%%%%%%%%%%%%%%%%%%%%%%%%%%%%%%%%%%%%%%%%%%%%%%%%%%%%%%%%%%%%%%%%%%%%%%%%%%%%%%%%%%%%%%%

If your \Index{Macintosh} is running the conventional \Index{Mac OS 9} or earlier, there are versions up to \FlightGear{} 0.7.6 available being provided courtesy Darrell
Walisser)\index{Walisser, Darrell}. Download the file \verb/FlightGear_Installer_0.X.X.sit/ from the corresponding subdirectory under
 \medskip

\web{http://icdweb.cc.purdue.edu/~walisser/fg/}.
 \medskip

 \noindent
This file contains the program as well as the required base package files (scenery etc.).
For unpacking, use \texttt{Stuffit Expander 5.0}\index{Stuffit Expander} or later.

The latest build available for Mac OS 9.x is 0.7.6, located in the same place. The base package is part of the download for Mac OS 9.x, but not for Mac OSX.

Alternatively, if you are running \Index{Mac OS X}, download \texttt{fgfs-0.X.X.gz} from
the same site named above. The Mac OS X  builds are in a gzip file in the same directory. There is a Readme file in the directory to help people identify what to download. 

Mac OS X requires that you first download the base package. Then extract it with

\noindent
\texttt{tar -zxvf fgfs-base-X.X.X.tar.gz}\\
\texttt{gunzip fgfs-X.X.X.-date.gz}

\noindent
Note that there is no \texttt{runfgfs} script for Mac OS X yet.

%%%%%%%%%%%%%%%%%%%%%%%%%%%%%%%%%%%%%%%%%%%%%%%%%%%%%%%%%%%%%%%%%%%%%%%%%%%%%%%%%%%%%%%%%%%%%%%
\section{Installing the binary distribution on a Debian Linux system}\index{binaries!Debian}
%%%%%%%%%%%%%%%%%%%%%%%%%%%%%%%%%%%%%%%%%%%%%%%%%%%%%%%%%%%%%%%%%%%%%%%%%%%%%%%%%%%%%%%%%%%%%%%

Download the file \verb/flightgear_0.7.6-6_i386.deb/ (being provided courtesy Ove
Kaaven)\index{Kaaven, Ove} from any of the \Index{Debian} mirror sites listed at
 \medskip

\web{http://packages.debian.org/unstable/games/flightgear.html}.
 \medskip

 \noindent
 Like any Debian package, this can be installed via
 \medskip

  \verb/dpkg --install flightgear_0.7.6-6_i386.deb/.
\medskip

 \noindent
 After installation, you will find the directory \texttt{/usr/local/Flightgear}
containing the script \texttt{runfgfs} to start the program.


%%%%%%%%%%%%%%%%%%%%%%%%%%%%%%%%%%%%%%%%%%%%%%%%%%%%%%%%%%%%%%%%%%%%%%%%%%%%%%%%%%%%%%%%%%%%%%%
\section{Installing the binary distribution on a SGI IRIX system}\index{binaries!SGI Irix}
%%%%%%%%%%%%%%%%%%%%%%%%%%%%%%%%%%%%%%%%%%%%%%%%%%%%%%%%%%%%%%%%%%%%%%%%%%%%%%%%%%%%%%%%%%%%%%%

If there are binaries available for SGI IRIX systems, download all the required files (being provided courtesy Erik Hofman)\index{Hofman, Erik}
from
 \medskip

\web{http://www.a1.nl/~ehofman/fgfs/}
 \medskip

 \noindent
 and install them. Now you can start \FlightGear{} via running the script\\
\texttt{/opt/bin/fgfs}.

%%%%%%%%%%%%%%%%%%%%%%%%%%%%%%%%%%%%%%%%%%%%%%%%%%%%%%%%%%%%%%%%%%%%%%%%%%%%%%%%%%%%%%%%%%%%%%%
\section{Installing add-on scenery}\index{scenery!add-on}\index{add-on scenery}
%%%%%%%%%%%%%%%%%%%%%%%%%%%%%%%%%%%%%%%%%%%%%%%%%%%%%%%%%%%%%%%%%%%%%%%%%%%%%%%%%%%%%%%%%%%%%%%

There are two complete sets of scenery files with worldwide coverage available, now, being based on different source data. One data set was created by Curt Olson\index{Olson, Curt} and can be downloaded via a clickable map\index{map, clickable} from
 \medskip

\web{http://www.flightgear.org/Downloads/world-scenery.html}
 \medskip

 \noindent
Moreover, Curt provides the complete set of US Scenery on \Index{CD-ROM} for those who
really would like to fly over all of the USA. For more detail, check the remarks on the
downloads page above.

An alternative data set was produced by William Riley\index{Riley, William} and is available from
\medskip

\web{http://www.randdtechnologies.com/fgfs/newScenery/world-scenery.html}
 \medskip

While the first data set is based on the \Index{USGS} data, the second one is based on the so-called \Index{VMap0} data set. While there may be more differences to discover, the first one has much better coast lines, while the latter sports world-wide coverage of streets, rivers, lakes, and more. Scenery provided in the base package is based on the second data set (though covering a small area around San Francisco, only).

Installation of both data sets is identical. You have to unpack them under \texttt{/Flightgear/Scenery}. Do not de-compress the numbered scenery files like 958402.gz! This will be done by \FlightGear{} on the fly.

As an example, consider installation of the scenery package w120n30 containing the Grand
Canyon Scenery.

After having installed the \Index{base package}, you should have ended up with the
following directory structure:
\medskip

\noindent
 \texttt{/usr/local/FlightGear/Scenery}

\noindent
 \texttt{/usr/local/FlightGear/Scenery/w130n30}

\noindent
 \texttt{/usr/local/FlightGear/Scenery/w130n30/w122n37}\\
 \texttt{/usr/local/FlightGear/Scenery/w130n30/w123n37}
 \medskip

\noindent 
 with the directories w122n37 and w123n37, resp. containing numerous *.gz
files. Installation of the Grand Canyon scenery adds to this the directories
\medskip

\noindent
 \texttt{/usr/local/FlightGear/Scenery/w120n30/w112n30}\\
 \texttt{/usr/local/FlightGear/Scenery/w120n30/w112n31}\\
 \texttt{...}\\
 \texttt{/usr/local/FlightGear/Scenery/w120n30/w120n39}.
 \medskip

You can exploit the \texttt{-$ $-fg-scenery={\it path}} command line option, if you want to install different scenery sets in parallel or want to have scenery sitting in another place.

%%%%%%%%%%%%%%%%%%%%%%%%%%%%%%%%%%%%%%%%%%%%%%%%%%%%%%%%%%%%%%%%%%%%%%%%%%%%%%%%%%%%%%%%%%%%%%%
\section{Installing documentation}\index{documentation!installation}
%%%%%%%%%%%%%%%%%%%%%%%%%%%%%%%%%%%%%%%%%%%%%%%%%%%%%%%%%%%%%%%%%%%%%%%%%%%%%%%%%%%%%%%%%%%%%%%

Most of the packages named above include the complete \FlightGear{} documentation
including a .pdf version of this \textit{Installation and Getting Started} Guide intended
for pretty printing using Adobe's Acrobat Reader being available from
 \medskip

\web{http://www.adobe.com/acrobat}
 \medskip

 \noindent
 Moreover, if properly installed, the .html version can be accessed via
\FlightGear{}'s \texttt{help} menu entry.

Besides, the source code contains a directory \texttt{docs-mini} containing numerous
ideas on and solutions to special problems. This is also a good place for further
reading.

%% Revision 0.00  1998/09/08  michael
%% Initial revision for version 0.53.
%% Revision 0.01  1998/09/20  michael
%% several extensions and corrections
%% revision 0.10  1998/10/01  michael
%% final proofreading for release
%% revision 0.11  1998/11/01  michael
%% support files Section completely re-written
%% revision 0.20  1999/06/04  michael
%% some updates and corrections, corrected links
%% revision 0.3 2000/04/20 michael
%% minor updates, reference to click map
%% revision 0.4 2001/05/12 michael
%% completely re-written, added MacOS, Debian, SGI secions
%% separate sections on global scenery and help
%% revision 0.5 2002/01/01 michael
%% minor tweaks
%% Installing on Mac newly written based on material by Darrell
%% revision 0.6 2002/01/01 michael
%% Added installation of W. Riley's VMap0 scenery
%% Dave Perry edits: p.35 removed "again using a.", p.36 added /Scenery, etc. to the directory
%% structures to match what I thought was intended.