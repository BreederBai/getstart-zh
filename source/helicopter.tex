%%%%%%%%%%%%%%%%%%% CHINESE VERSION %%%%%%%%%%%%%%%%%%%%%%%%%%%%%%%%
\chapter{直升机教程}
\label{helicopter}

\section{前言}

首先:任何适用于真实直升机的原理都适用于 \FlightGear。相关的基本操作也可以在这里找到:\\\weblong{http://www.cybercom.net/~copters/pilot/maneuvers.html}{http://www.cybercom.net/\~{}copters/pilot/maneuvers.html} 很多细节在 \FlightGear 里面都有所简化。

%%%%%%%%%%%%%%%%%%% ENGLISH VERSION %%%%%%%%%%%%%%%%%%%%%%%%%%%%%%%%
\iffalse
\chapter{A Helicopter Tutorial}
\label{helicopter}
%%%%%%%%%%%%%%%%%%%%%%%%%%%%%%%%%%%%%%%%%%%%%%%%%%%%%%%%%%%%%%%%%%%%

\section{Preface}
 
First: in principle everything that applies to real helicopters, applies 
to \FlightGear. Fundamental maneuvers are well described here:\\ 
\weblong{http://www.cybercom.net/~copters/pilot/maneuvers.html}
{http://www.cybercom.net/\~{}copters/pilot/maneuvers.html} Some details are 
simplified in \FlightGear, in particular the engine handling and some 
overstresses are not simulated or are without any consequence. 
In \FlightGear{} it is (up to now) not possible to damage a helicopter in 
flight. 

\photo{}{}{Bo105_cockpit} 

The helicopter flight model of \FlightGear{} is 
quite realistic. The only exceptions are ``vortex ring conditions''. 
These occur if you descend too fast and perpendicularly 
(without forward speed). The heli can get into its own rotor downwash 
causing the lift to be substantially reduced. Recovering from this condition 
is possible only at higher altitudes. On the Internet you can find a video 
of a Seaking helicopter, which got into this condition during a flight 
demonstration and touched down so hard afterwards that it was completely 
destroyed. 

For all \FlightGear{} helicopters the parameters are not completely optimized 
and thus the performance data between model and original can deviate slightly. 
On the hardware side I recommend the use of a ``good'' joystick. A joystick 
without springs is recommended because it will not center by itself. You 
can either remove the spring from a normal joystick, or use a force feedback 
joystick, with a disconnected voltage supply. Further, the joystick should 
have a ``thrust controller'' (throttle). For controlling the tail rotor you 
should have pedals or at least a twistable joystick - using a keyboard is hard.
\FlightGear{} supports multiple joysticks attached at the same time. 

\section{Getting started}

The number of available helicopters in \FlightGear{} is limited. In my opinion 
the Bo105 is the easiest to fly, since it reacts substantially more directly 
than other helicopters. For flight behavior I can also recommend the S76C. 
The S76C reacts more retarded than the Bo. 

Once you have loaded \FlightGear, take a moment to centralize the controls by 
moving them around. In particular the collective is often at maximum on startup. 

\photo{}{}{S76c_landed} 

The helicopter is controlled by four functions. The stick (joystick) controls 
two of them, the inclination of the rotor disc (and thus the inclination of 
the helicopter) to the right/left and forwards/back. Together these functions 
are called ``cyclic blade control''. Next there is the ``collective blade 
control'', which is controlled by the thrust controller. This causes a change 
of the thrust produced by the rotor. Since the powering of the main rotor 
transfers a torque to the fuselage, this must be compensated by the 
tail rotor. Since the torque is dependent on the collective and on the flight 
condition as well as the wind on the fuselage, the tail rotor is also 
controlled by the pilot using the pedals. If you push the right pedal, 
the helicopter turns to the right (!). The pedals are not a steering wheel. 
Using the pedals you can yaw helicopter around the vertical axis. The 
number of revolutions of the rotor is kept constant (if possible) by the 
aircraft. 

\photo{}{}{Ec135_in_the_air} 

\section{Lift-Off}

First reduce the collective to minimum. To increase the rotor thrust, you have 
to ``pull'' the collective. Therefore for minimum collective you have to push 
the control down (that is the full acceleration position (!) of the thrust 
controller). Equally, ``full power'' has the thrust controller at idle. 
Start the engine with \key{\}}. After few seconds the rotor will start to 
turn and accelerates slowly. Keep the stick and the pedals approximately 
centered. Wait until the rotor has finished accelerating. For the Bo105 there 
is an instruments for engine and rotor speed on the left of the upper row. 

Once rotor acceleration is complete, pull the collective very slowly. Keep 
your eye on the horizon. If the heli tilts or turns even slightly, stop 
increasing the collective and correct the position/movement with stick and 
pedals. If you are successful, continue pulling the collective (slowly!). 

As the helicopter takes off, increase the collective a little bit more and try 
to keep the helicopter in a leveled position. The main challenge is reacting 
to the inadvertent rotating motion of the helicopter with the correct control 
inputs. Only three things can help you: practice, practice and practice. 
It is quite common for it to take hours of practice to achieve a halfway good 
looking hovering flight. Note: The stick position in a stable hover is not 
the center position of the joystick. 

\section{In the air}
 
To avoid the continual frustration of trying to achieve level flight, you may 
want to try forward flight. After take off continue pulling the collective a 
short time and then lower the nose a slightly using the control stick. The 
helicopter will accelerate forward. With forward speed the tail rotor does not 
have to be controlled as precisely due to the relative wind coming from 
directly ahead. Altogether the flight behavior in forward flight is quite 
similar to that of an badly trimmed airplane. The ``neutral'' position of the 
stick will depend upon airspeed and collective. 

Transitioning from forward flight to hovering is easiest if you reduce speed 
slowly by raising the nose of the helicopter. At the same time, reduce the 
collective to stop the helicopter from climbing. As the helicopter slows, 
``translation lift'' is reduced, and you will have to compensate by pulling 
the collective. When the speed is nearly zero, lower the nose to the position 
it was when hovering. Otherwise the helicopter will accelerate backwards! 

\section{Back to Earth I}
 
To land the helicopter transition to a hover as described above while reducing 
the altitude using the collective. Briefly before hitting the ground reduce 
the rate of descent slowly. A perfect landing is achieved if you managed to 
zero the altitude, speed and descent rate at the same time (gently). 
However, such landing are extremely difficult. Most pilots perform a hover 
more or less near to the ground and then decent slowly to the ground. Landing 
with forward velocity is easier, however you must make sure you don't land 
with any lateral (sideways) component to avoid a rollover. 

\photo{}{}{Bo105_landed} 

\section{Back to Earth II}

It is worth mentioning autoration briefly. This is a unpowered flight condition, 
where the flow of air through the rotors rotates the rotor itelf. At an 
appropriate altitude select a landing point (at first in the size of a larger 
airfield) and then switch the engine off by pressing \key{\{}. Reduce 
collective to minimum, place the tail rotor to approximately 0$\textdegree$ 
incidence (with the Bo push the right pedal about half, with As350 the left). 
Approach at approximately 80 knots. Don't allow the rotor speed to rise more 
than a few percent over 100\%, otherwise the rotor will be damaged (though 
this is not currently simulated). As you reach the ground, reduce the airspeed 
by lifting the nose. The descent rate will drop at the same time, so you do 
not need to pull the collective. It may be the case that the rotor speed 
rises beyond the permitted range. Counteract this by raising the collective 
if required. Just above the ground, reduce the descent rate by pulling the 
collective. The goal is it to touch down with a very low descent rate and no 
forward speed. With forward speed it is easier, but there is a danger of a 
roll over if the skids are not aligned parallel to the flight direction. 
During the approach it is not necessary to adjust the tail rotor, since 
without power there is almost no torque. If you feel (after some practice), 
that autorotation is too easy, try it with a more realistic payload via 
the \command{payload} menu. 

\photo{}{}{Bo105_auto} 
\fi












