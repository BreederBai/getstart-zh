%%
%% getstart.tex -- Flight Gear documentation: The FlightGear Manual
%% Chapter file
%%
%% Written by Michael Basler, started September 1998.
%%
%% Copyright (C) 2002 Michael Basler
%%
%%
%% This program is free software; you can redistribute it and/or
%% modify it under the terms of the GNU General Public License as
%% published by the Free Software Foundation; either version 2 of the
%% License, or (at your option) any later version.
%%
%% This program is distributed in the hope that it will be useful, but
%% WITHOUT ANY WARRANTY; without even the implied warranty of
%% MERCHANTABILITY or FITNESS FOR A PARTICULAR PURPOSE.  See the GNU
%% General Public License for more details.
%%
%% You should have received a copy of the GNU General Public License
%% along with this program; if not, write to the Free Software
%% Foundation, Inc., 675 Mass Ave, Cambridge, MA 02139, USA.
%%
%% $Id: landing.tex,v 0.6 2002/09/09 michael
%% (Log is kept at end of this file)

%%%%%%%%%%%%%%%%%%%%%%%%%%%%%%%%%%%%%%%%%%%%%%%%%%%%%%%%%%%%%%%%%%%%%%%%%%%%%%%%%%%%%%%%%%%%%%%
\ifchinese
\chapter{{\\}落地:离开飞机之前的一些想法}
\fi
\iffalse
\IfLanguageName{english}{
\chapter{Landing: Some further thoughts before leaving the plane}
}{}
\IfLanguageName{french}{
\chapter{Atterrissage : quelques r\'{e}flexions compl\'{e}mentaires avant de quitter l'avion}
}{}
\fi
\label{landing}
%%%%%%%%%%%%%%%%%%%%%%%%%%%%%%%%%%%%%%%%%%%%%%%%%%%%%%%%%%%%%%%%%%%%%%%%%%%%%%%%%%%%%%%%%%%%%%%
\ifchinese
\markboth{\thechapter.\hspace*{1mm}
落地}{\thesection\hspace*{1mm} 那些贡献者}
\fi
\iffalse
\IfLanguageName{english}{
\markboth{\thechapter.\hspace*{1mm}
LANDING}{\thesection\hspace*{1mm} THOSE, WHO DID THE WORK}
}{}
\IfLanguageName{french}{
\markboth{\thechapter.\hspace*{1mm}
ATTERRISSAGE}{\thesection\hspace*{1mm} CEUX QUI ONT TRAVAILLE}
}{}
\fi
%%%%%%%%%%%%%%%%%%%%%%%%%%%%%%%%%%%%%%%%%%%%%%%%%%%%%%%%%%%%%%%%%%%%%%%%%%%%%%%%%%%%%%%%%%%%%%%
\ifchinese
\section{\FlightGear{} 的简略\Index{历史}}
\fi
\iffalse
\IfLanguageName{english}{
\section{A Sketch on the \Index{History} of \FlightGear{}}
}{}
\IfLanguageName{french}{
\section{Une \'{e}bauche de l'\Index{histoire} de \FlightGear{}}
}{}
\fi
%%%%%%%%%%%%%%%%%%%%%%%%%%%%%%%%%%%%%%%%%%%%%%%%%%%%%%%%%%%%%%%%%%%%%%%%%%%%%%%%%%%%%%%%%%%%%%%
\ifchinese
讲述历史往往很无聊。然而,一直以来有很多人询问 \FlightGear{} 的历史。那么我们不妨就来讲述一下吧。

\FlightGear{} 项目肇始于 1996 年的一次网民的讨论,最终 David Murr\index{Murr, David} 根据这次讨论写了提案。很不幸,他后来放弃了此项目(也包括网络)。原始的\Index{提案}依然可以在 \FlightGear{} 网站找到:

\fi
\iffalse
\IfLanguageName{english}{
History may be a boring subject. However, from time to time there are people asking for the history of \FlightGear{}. As a result, we'll give a short outline.

The \FlightGear{} project goes back to a discussion among a group of net citizens in 1996 resulting in a proposal written by David Murr\index{Murr, David} who, unfortunately, dropped out of the project (as well as the net) later. The original \Index{proposal} is still available
from the \FlightGear{} web site and can be found under
}{}
\IfLanguageName{french}{
L'histoire peut rapidement devenir un sujet barbant. Cependant, de temps en temps il y a des personnes qui s'int\'{e}ressent \`{a} l'histoire de \FlightGear{}. En cons\'{e}quence, nous allons en dresser un aper\c{c}u rapide.

A l'origine, le projet \FlightGear{} remonte \`{a} une discussion entre un groupe de citoyens du net en 1996 qui a eu pour r\'{e}sultat l'\'{e}criture d'une proposition par David Murr\index{Murr, David} qui, malheureusement, a quitt\'{e} le projet (ainsi que le net) plus tard. La \Index{proposition} d'origine est toujours disponible
\`{a} partir du site Internet de \FlightGear{} \`{a} l'adresse :
}{}
\fi

 \medskip

\web{http://www.flightgear.org/proposal-3.0.1}.
 \medskip

\noindent
\ifchinese
虽然开发者和各种细节随着时间早已地覆翻天,然而这份提案的精神却保留至今。

实际上在 1996 年夏天开发就已经开始了,到年底时图形程序也已完成。那时,主要编码工作由伯克利大学的 Eric Korpela\index{Korpela, Eric} 负责。早期代码可以在 \Index{Linux}、\Index{DOS}、\Index{OS/2}、\Index{Windows 95/NT}和 \Index{Sun-OS} 上运行。所以这是一个雄心勃勃的项目,而且编写系统独立的\Index{图形界面}完全是白手起家。

开发工作很缓慢并最终在 1997 年初停滞,因为 Eric 完成了他的论文。就在此时,项目似乎要死掉,邮件列表也没有什么讨论。

1997 年年中的时候,在明尼苏达州大学就读的 Curt Olson\index{Olson, Curt} 为这个项目带来了希望。他的理念简单却强大:为什么要重新发明轮子呢?在 \Index{UNIX} 的各种\Index{工作站}上有很多自由的飞行模拟器可用。其中之一就是 \Index{LaRCsim}(由 NASA 工程师 Bruce Jackson\index{Jackson,Bruce} 开发),看起来非常适合。Curt 重拾这个项目,并重写了很多界面代码以便可以运行在不同的平台。这么做的关键就是为了使用系统平台无关的图形平台——\Index{OpenGL}。

另一项明智的决定是在第一版就选择基础\Index{地景}数据。\FlightGear{} 的地景是基于\Index{美国地质勘探局}发布的卫星数据。这些地形数据可以从这里下载:

\fi
\iffalse
\IfLanguageName{english}{
Although the names of the people and several of the details have changed over time,
the spirit of that proposal has clearly been retained up to the present time.

Actual coding started in the summer of 1996 and by the end of that year essential
graphics routines were completed. At that time, programming was mainly performed and
coordinated by Eric Korpela\index{Korpela, Eric} from Berkeley University. Early code ran
under \Index{Linux} as well as under \Index{DOS}, \Index{OS/2}, \Index{Windows 95/NT},
and \Index{Sun-OS}. This was found to be quite an ambitious project as it involved, among
other things, writing all the \Index{graphics routines} in a system-independent way
entirely from scratch.

Development slowed and finally stopped in the beginning of 1997 when Eric was completing
his thesis. At this point, the project seemed to be dead and traffic on the mailing list
went down to nearly nothing.

It was Curt Olson\index{Olson, Curt} from the University of Minnesota who re-launched the
project in the middle of 1997. His idea was as simple as it was powerful: Why invent the
wheel a second time? There have been several free flight simulators\index{Flight
simulator!free} available running on \Index{workstation}s under different flavors of
\Index{UNIX}. One of these, \Index{LaRCsim} (developed by Bruce Jackson\index{Jackson,
Bruce} from NASA), seemed to be well suited to the approach. Curt took this one apart and
re-wrote several of the routines such as to make them build as well as run on the
intended target platforms. The key idea in doing so was to exploit a system-independent
graphics platform: \Index{OpenGL}.

In addition, a clever decision on the selection of the basic \Index{scenery}
data was made in the very first version. \FlightGear{} scenery is created
based on satellite data published by the \Index{U.\,S. Geological Survey}.
These terrain data are available from
}{}
\IfLanguageName{french}{
Bien que les noms des personnes et des d\'{e}tails aient \'{e}volu\'{e} au fil du temps,
l'esprit de cette proposition a clairement \'{e}t\'{e} maintenu au fil du temps jusqu'\`{a}
aujourd'hui.

La v\'{e}ritable \'{e}criture du code a d\'{e}but\'{e} \`{a} l'\'{e}t\'{e} 1996 et, d\`{e}s
la fin de l'ann\'{e}e, l'essentiel des routines graphiques \'{e}tait \'{e}crit. A cette \'{e}poque,
la programmation \'{e}tait essentiellement r\'{e}alis\'{e}e et coordonn\'{e}e par Eric Korpela\index{Korpela, Eric}
de l'universit\'{e} de Berkeley. Le tout premier code fonctionnait sous \Index{Linux} ainsi que sous \Index{DOS},
\Index{OS/2}, \Index{Windows 95/NT}, et \Index{Sun-OS}. C'\'{e}tait un projet assez ambitieux car il n\'{e}cessitait
notamment d'\'{e}crire, en partant de rien, toutes les \Index{routines graphiques} d'une mani\`{e}re ind\'{e}pendante
du syst\`{e}me.

Le d\'{e}veloppement ralentit pour finalement s'arr\^{e}ter au d\'{e}but de l'ann\'{e}e 1997 lorsque Eric
termina sa th\`{e}se. A ce moment, le projet semblait mort et le trafic sur la liste de diffusion \'{e}tait
quasiment r\'{e}duit \`{a} n\'{e}ant.

Curt Olson\index{Olson, Curt}, de l'Universit\'{e} du Minnesota, relan\c{c}a le projet au milieu de l'ann\'{e}
1997. Son id\'{e}e \'{e}tait aussi simple que puissante : pourquoi r\'{e}inventer la route une seconde fois ?
Il y avait d\'{e}j\`{a} eu plusieurs simulateurs de vol libres\index{Simulateur de vol!libre} disponibles sur
des \Index{stations de travail} sous diff\'{e}rentes moutures d'\Index{UNIX}. L'un de ceux-\c{c}i, \Index{LaRCsim}
(d\'{e}velopp\'{e} par Bruce Jackson\index{Jackson,Bruce} de la NASA), semblait \^{e}tre particuli\`{e}rement
appropri\'{e} \`{a} cette approche. Curt s'y int\'{e}ressa et en r\'{e}-\'{e}crit plusieurs routines afin de les
faire compiler et fonctionner sur les plate-formes cibl\'{e}es. Ce-faisant, l'id\'{e}e ma\^{i}tresse \'{e}tait d'exploiter
une plate-forme graphique ind\'{e}pendante du syst\`{e}me : \Index{OpenGL}.

En compl\'{e}ment, la (bonne) d\'{e}cision du choix des donn\'{e}es des \Index{sc\`{e}nes} de base a \'{e}t\'{e} prise
d\`{e}s la toute premi\`{e}re version. Les sc\`{e}nes de \FlightGear{} sont g\'{e}n\'{e}r\'{e}es en se basant sur les
donn\'{e}es satellite publi\'{e}es par le \Index{U.\,S. Geological Survey}.
Ces donn\'{e}es de terrain sont disponibles respectivement \`{a} partir de, respectivement :
}{}
\fi

 \medskip

 \href{http://edc.usgs.gov/geodata/}{http://edc.usgs.gov/geodata/}
 \medskip

\noindent
\ifchinese
美国和
\fi
\iffalse
 \IfLanguageName{english}{
 for the U.S., and
 }{}
 \IfLanguageName{french}{
 pour les Etats-Unis et
 }{}
\fi
  \medskip

 \href{http://edcdaac.usgs.gov/gtopo30/gtopo30.html}{http://edcdaac.usgs.gov/gtopo30/gtopo30.html},
  \medskip

\noindent
\ifchinese
和其他国家。这些可访问的地景数据,连同地景构建工具都包含在了 \FlightGear{} 里面,这项重要的特性可以让任何人都创建自己的地景。

新的 \FlightGear{} 代码(依旧大量基于原始 \Index{LaRCsim} 项目)首先在 1997 年 7 月发布。从这时起,这个项目重获新生。下面介绍最近开发历史中的一些重要里程碑。
\fi
\iffalse
\IfLanguageName{english}{
 resp., for other countries. Those freely accessible scenery data, in
 conjunction with scenery building tools included with
 \FlightGear{}, are an important feature enabling anyone to
  create his or her own scenery.

This new \FlightGear{} code - still largely being based on the original \Index{LaRCsim}
code - was released in July 1997. From that moment the project gained momentum again.
Here are some milestones in the more recent development history.
}{}
\IfLanguageName{french}{
 pour les autres pays. Ces donn\'{e}es de sc\`{e}nes accessibles librement, en conjonction
 avec les outils de construction des sc\`{e}nes inclus dans \FlightGear{}, sont des
 fonctionnalit\'{e}s importantes permettant \`{a} chacun(e) de cr\'{e}er ses propres sc\`{e}nes.

Ce nouveau code de \FlightGear{} - toujours largement bas\'{e} sur le code original de \Index{LaRCsim}
- a \'{e}t\'{e} mis \`{a} disposition en juillet 1997. A partir de cet instant, le projet reprit de
l'inertie. Voi\c{c}i les principales \'{e}tapes dans l'histoire r\'{e}cente du d\'{e}veloppement.
}{}
\fi

%%%%%%%%%%%%%%%%%%%%%%%%%%%%%%%%%% List of Development %%%%%%%%%%%%%%%%%%%%%%%%%%%%%
%%%%%%%%%%%%%%%%%%%%%%%%%Scenery%%%%%%%%%%%%%%%%%%%%%%%%%%%%%%%%%%%%%%
\ifchinese
\subsection{地景}\index{历史!地景}
\fi
\iffalse
\IfLanguageName{english}{
\subsection{Scenery}\index{history!scenery}
}{}
\IfLanguageName{french}{
\subsection{Sc\`{e}nes}\index{histoire!sc\`{e}nes}
}{}
\fi
\begin{itemize}
\ifchinese
\item Curt Olson\index{Olson, Curt} 在 1998 年春天加入了纹理贴图支持\index{纹理贴图}。这对现实性有非常大的提高。很多高质量的\Index{纹理贴图}由 Eric Mitchell\index{Mitchell, Eric} 提交到 \FlightGear{} 项目。Hofman\index{Hofman, Erik} 则提交了另外一系列高质量贴图。
\fi
\iffalse
\IfLanguageName{english}{
\item Texture support\index{textures} was added by Curt
Olson\index{Olson, Curt} in spring 1998. This marked a significant
improvement in terms of reality. Some high-quality \Index{textures} were
submitted by Eric Mitchell\index{Mitchell, Eric} for the \FlightGear{}
project. Another set of high-quality textures was added by Erik
Hofman\index{Hofman, Erik} ever since.
}{}
\IfLanguageName{french}{
\item La prise en charge des textures\index{textures} a \'{e}t\'{e} ajout\'{e}
par Curt Olson\index{Olson, Curt} \`{a} l'automne 1998. Ceci a marqu\'{e} un tournant
important en termes de r\'{e}alisme. Quelques \Index{textures} de haute qualit\'{e} pour
le projet \FlightGear{} ont \'{e}t\'{e} ajout\'{e}es par Eric Mitchell\index{Mitchell, Eric}.
Depuis, d'autres textures de haute qualit\'{e} ont \'{e}t\'{e} ajout\'{e}es par Erik Hofman\index{Hofman, Erik}.
}{}
\fi

\ifchinese
\item 1998 年春天,在提升了\Index{地景}和\Index{纹理贴图}支持之后,\FlightGear{} 的\Index{帧速率}掉到了几乎无法飞行。这个问题因为 \Index{OpenGL} 对硬件支持而解决,同时 Curt Olson\index{Olson, Curt} 实现了\Index{视域剔除}(一种渲染技术,可以将视野里不可见的地景剃除)。

关于\Index{帧速率},有些人始终没有放下,是时,没有什么方法去优化,一直留待某人去完成。
\item 1998 年 9 月,Curt Olson\index{Olson, Curt} 成功完成了美国的地形模型。现在地景已经覆盖全球,可以使用这个可点击的地图\index{地图, 可点击}:
\fi
\iffalse
\IfLanguageName{english}{
 \item After improving the \Index{scenery} and \Index{texture} support frame rate\index{frame rate} dropped down to a point where \FlightGear{} became unflyable in spring 1998. This issue was resolved by exploiting hardware \Index{OpenGL} support, which became available at that time, and implementing \Index{view frustrum culling} (a rendering technique that ignores the
 part of the scenery not visible in a scene), done by Curt Olson\index{Olson, Curt}.
 With respect to \Index{frame rate} one should keep in mind that the code, at present, is in no way optimized, which leaves room for further improvements.
 \item In September 1998 Curt Olson\index{Olson, Curt} succeeded in creating a complete terrain model for the U.S. The scenery is available worldwide now, via a clickable map \index{map, clickable} at:
}{}
\IfLanguageName{french}{
 \item Apr\`{e}s avoir am\'{e}lior\'{e} la prise en charge des \Index{sc\`{e}nes} et des \Index{textures}, le \index{taux de rafraichissement} de l'image diminua \`{a} tel point qu'il \'{e}tait impossible de faire voler un avion dans \FlightGear{} au printemps 1998. Cette difficult\'{e} fut contourn\'{e}e en exploitant la prise en charge mat\'{e}rielle d'\Index{OpenGL}, qui devint disponible \`{a} cette date, et par l'impl\'{e}mentation de la ''\Index{vue frustrum culling}'', r\'{e}alis\'{e}e par Curt Olson\index{Olson, Curt}, (une technique de rendu qui ignore la
 partie du paysage non visible d'une sc\`{e}ne).
 En ce qui concerne le \Index{taux de rafra\^{i}chissement}, il convient de se rappeler que le code, tel qu'il est aujourd'hui, n'est en aucun cas optimis\'{e}, ce qui laisse des possibilit\'{e}s pour de futures am\'{e}liorations.
 \item En september 1998, Curt Olson\index{Olson, Curt} parvint \`{a} cr\'{e}er un mod\`{e}le de terrain complet pour les Etats-Unis. Les sc\`{e}nes sont aujourd'hui mondiales, et disponibles via une carte cliquable \index{carte, cliquable}, situ\'{e}e \`{a} l'adresse :
}{}
\fi

 \medskip
\web{http://www.flightgear.org/download/scenery/}.
 \medskip

\ifchinese
\item 后来\Index{地景}更加入了\Index{地理特征},包括了湖泊、河流和海岸线。 Dave Cornish\index{Cornish, Dave}在 2001 年春为跑道添加了纹理贴图。\Index{光线贴图}也为晚间增加了视觉色彩。为了应对不断增加的地景数据,2001 年春一种二进制地景格式引入到项目中。同时 Curt Olson\index{Olson, Curt} 也引入了\Index{跑道光线}。最终到 2002 年夏天,William Riley\index{Riley, William} 基于 David Megginson\index{Megginson, David} 的筹备文档,创建了一套覆盖全球的完整\Index{地景}文件。基于的数据称为 VMap0\index{VMap0 数据} 而现在使用的是 \Index{GSHHS 数据}。这套覆盖全球的地景包括主要的街道、河流等等,然而海岸线还不够精确。\FlightGear{} 的基础地景从 2002 年夏开始就基于这套地景文件。
\fi
\iffalse
\IfLanguageName{english}{
\item Scenery\index{scenery} was further improved by adding \Index{geographic features} including lakes, rivers, and coastlines later. Textured runways were added by Dave Cornish\index{Cornish, Dave} in spring 2001. Light textures\index{light textures} add to the visual impression at night. To cope with the constant growth of scenery data, a binary scenery format was introduced in spring 2001. Runway lighting\index{runway lighting} was introduced by Curt Olson\index{Olson, Curt} in spring 2001. Finally, a completely new set of \Index{scenery} files for the whole world was created by William Riley\index{Riley, William} based on preparatory documentation by David Megginson\index{Megginson, David} in summer 2002. This is based on a data set called VMap0\index{VMap0 data} as an alternative to the \Index{GSHHS data} used so far. This scenery is a big improvement as it has world wide coverage of main streets, rivers, etc., while it's downside are much less accurate coast lines. \FlightGear{}'s base scenery is based on these new scenery files since summer 2002.
}{}
\IfLanguageName{french}{
\item Les sc\`{e}nes\index{sc\`{e}nes} ont ensuite \'{e}t\'{e} am\'{e}lior\'{e}es par l'ajout de \Index{fonctionnalit\'{e}es g\'{e}ographiques} comme les lacs, les rivi\`{e}res et, par la suite, les traits de c\^{o}te. Les textures des pistes ont \'{e}t\'{e} ajout\'{e}es par Dave Cornish\index{Cornish, Dave} au printemps 2001. Les textures de lumi\`{e}re\index{textures de lumi\`{e}re} s'ajoutent \`{a} l'impression visuelle la nuit. Pour faire face \`{a} la croissance constante des donn\'{e}es de sc\`{e}ne, un format binaire des sc\`{e}nes a \'{e}t\'{e} introduit au printemps 2001. L'\'{e}clairage des pistes\index{\'{e}clairage des pistes} a \'{e}t\'{e} introduit par Curt Olson\index{Olson, Curt} au printemps 2001. Enfin, un nouvel ensemble de fichiers de donn\'{e}es de \Index{sc\`{e}nes} pour le monde entier a \'{e}t\'{e} cr\'{e}\'{e} par William Riley\index{Riley, William} en se basant sur un travail de documentation pr\'{e}alable de David Megginson\index{Megginson, David} \`{a} l'\'{e}t\'{e} 2002. Il repose sur un ensemble de donn\'{e}es appel\'{e} VMap0\index{donn\'{e}es VMap0}, comme alternative aux \Index{donn\'{e}es GSHHS} utilis\'{e}es jusqu'alors. Ces donn\'{e}es sont une grande am\'{e}lioration car elles offrent une couverture mondiale des rues principales, rivi\`{e}res, etc., au d\'{e}triment d'un trait de c\^{o}te moins pr\'{e}cis. Les sc\`{e}nes de base de \FlightGear{} s'appuient sur ces nouvelles donn\'{e}es depuis l'\'{e}t\'{e} 2002.
}{}
\fi
 \medskip

\ifchinese
\item 2001 年还增加了对\Index{静态物体}的支持,这样就可以在地景上放置建筑物、静止的飞机、树木等等。
\item 2002 年夏覆盖全球的\Index{随机地面物体}也引入了,以合适的类型和精细度覆盖本地地面。这项现实性增强要感谢 D. Megginson\index{Megginson, David} 的贡献。
\item 今天,我们始终在努力,随着 Terrasyc 的引入,可以让用户一边飞行一边下载地景,强大的 mapserver 和地景建模基础架构作后盾,网站也让我们可以快速增加和更新地景。地景生成工具也大幅更新,以便可以将精细度提升到 8.50 apt.dat 格式。同时,当协议适当的时候,也会引入外部数据源比如 OpenStreetMap。
\fi
\iffalse
\IfLanguageName{english}{
\item There was support added for \Index{static objects} to the scenery in 2001, which permits placing buildings, static planes, trees and so on in the scenery.
\item The world is populated with \Index{random ground objects} with appropriate type and density for the local ground cover type since summer 2002. This marks
a major improvement of reality and is mainly thanks to work by D. Megginson\index{Megginson, David}.
\item Today, the effort is still going on, with the use of Terrasync, the tool for on-the-fly scenery download, the powerful mapserver and scenemodels infrastructure
laying behind, as well as webforms enabling to quickly add or update objects. Scenery generation tools have been updated in order to use more accurate data such as the
8.50 apt.dat format, as well as external data such as OpenStreetMap, when the licence is adequate.
}{}
\IfLanguageName{french}{
\item La prise en charge des \Index{objets statiques} a \'{e}t\'{e} impl\'{e}men't\'{e}e dans les sc\`{e}nes en 2001, ce qui a permis d'ajouter, entres autres, des immeubles, des avions statiques, des arbres, dans les sc\`{e}nes.
\item Le monde est couvert d'\Index{objets al\'{e}atoires au sol} de densit\'{e} et de type appropri\'{e}s avec le type d'occupation du sol depuis l'\'{e}t\'{e} 2002. Ceci marque une am\'{e}lioration majeure dans le r\'{e}alisme, et est principalement \`{a} attribuer au travail de D. Megginson\index{Megginson, David}.
\item Aujourd'hui, l'effort se poursuit, avec l'utilisation de Terrasync, l'outil de t\'{e}l\'echargement des sc\`{e}nes \`{a} la vol\'{e}e, l'infrastructure puissante
de cartographie ''mapserver'' et notre base de donn\'{e}es ''scenemodels'', des formulaires web permettant l'ajout et la mise \`{a} jour rapide d'objets. Les outils de
g\'{e}n\'{e}ration des sc\`{e}nes ont \'{e}t\'{e} am\'{e}lior\'{e}s pour utiliser des donn\'{e}es plus pr\'{e}cises, comme le format de donn\'{e}es 8.50 d'apt.dat,
ainsi que les donn\'{e}es externes comme OpenStreetMap, lorsque leur licence le permettra.
}{}
\fi

\end{itemize}

%%%%%%%%%%%%%%%%%%%%%%%%%Aircraft%%%%%%%%%%%%%%%%%%%%%%%%%%%%%%%%%%%%%%
\ifchinese
\subsection{飞行器}\index{历史!飞行器}
\begin{itemize}
\item 1997 年秋,基于 Michele America\index{America, Michele} 和 Charlie Hotch\-kiss\index{Hotchkiss, Charlie} 贡献的代码,\Index{HUD}(\Index{Head Up Display},平视显示器)引入到了项目中,随后 Norman Vine 又做了大幅提高。虽然对默认的\Index{塞斯纳 172} 并没有什么帮助,不过因为可以显示飞行姿态,对后面引入军用喷气机有很大帮助。
\item 1998 年 4 月,低级的\Index{自动驾驶仪}航向保持模式由 Jeff Goeke-Smith\index{Goeke-Smith, Jeff} 贡献到项目中。随后当年 10 月,Norman Vine\index{Vine, Norman} 又增加了高度保持和地形追踪开关。
\item 1998 年 6 月,Friedemann Reinhard \index{Reinhard, Friedemann} 开发了早期的\Index{仪表板}代码。不幸的是,后来开发慢了下来。最终 David Megginson \index{Megginson, David} 在 2000 年 1 月决定开始重构仪表板代码。这大幅增加了仪表和特性,这样到 2001 年春,几乎所有主要仪表都已经包含进去了。一个小型的迷你仪表板在 2001 年夏天开发出来。
\item 最终,\Index{LaRCsim} 里的 \Index{Navion} 飞机被现在的默认\Index{塞斯纳 172}所替代,塞斯纳 172 在 2000 年 2 月变得足够稳定,这招致大多数用户的欢迎。这样在运行时就可以选择多种\Index{飞行模型}和飞机选项了。Jon Berndt\index{Berndt, Jon, S.} 投入大量时间实现了一个飞行器配置方式,提高了真实感和灵活性。\JSBSim,后来替代了 LaRCsim 成为默认的\Index{飞行动态模型}(\Index{FDM}),它也打算包含一些特性比如燃油搅拌特效、乱流、完整的飞行控制系统和其他飞行模拟器里不太常见的特性。作为替代,Andy Ross\index{Ross, Andy} 于 2001 年底开发了另一个飞行动态模型 \YASim{}(Yet Another Flight Dynamics Simulator),使用简单特性并基于流体力学原理。此飞行动态模型带来了 747、A4 和 DC-3。另外,Michael Selig\index{Selig, Michael} 领导的团队在 2000 年开发了 \Index{UIUC} 飞行动态模型和一系列相应的飞行器。
\item 一套完全可用的 \Index{无线电栈} 和无线电系统由 Curt Olson\index{Olson, Curt} 于 2000 年春开发完成。大量导航台数据库,这样就可以使用 ILS 也由 Robin Peel\index{Peel, Robin} 贡献进来。基本的\Index{ATC}支持也在 2001 年底由 David Luff\index{Luff, David} 开发进来。这还没有完全实现,但显示 \Index{ATIS 消息}成为可能。\Index{磁电机开关}和相应的程序在 2001 年底由 John Check\index{Check, John} 和 David Megginson\index{Megginson, David} 开发完成。同时在 2001 年到 2002 年间,大量仪表也由 John 和其他人持续完善。\FlightGear{} 现在就可以允许飞 ILS 进近并可以使用\Index{Bendix 应答机}了。
\item 2002 年\Index{多发动机支持}引入到了\FlightGear{}。\JSBSim{} 成为 \FlightGear{} 默认的 FDM。
\item 2002 春,John Check\index{Check, John} 和其他人的努力使“真实的” \Index{3D 仪表板}变得足够稳定。同时可移动的操作面\index{控制面, 可移动} 比如螺旋桨等等,由 David Megginson.\index{Megginson, David} 开发完成。

\end{itemize}
\fi
\iffalse
\IfLanguageName{english}{
\subsection{Aircraft}\index{history!aircraft}
\begin{itemize}
\item A \Index{HUD} (\Index{head up display}) was added based on code
 provided by Michele America\index{America, Michele} and Charlie Hotch\-kiss\index{Hotchkiss, Charlie} in the fall of 1997 and was improved later by Norman Vine. While not generally available for real \Index{Cessna 172}, the HUD conveniently reports the actual flight performance of the simulation and may be of further use in military jets later.
 \item A rudimentary \Index{autopilot} implementing heading hold was
contributed by Jeff Goeke-Smith\index{Goeke-Smith, Jeff} in April 1998. It was improved
by the addition of an altitude hold and a terrain following switch in October 1998 and
further developed by Norman Vine\index{Vine, Norman} later.
\item Friedemann Reinhard \index{Reinhard, Friedemann} developed early \Index{instrument panel} code, which was added in June 1998. Unfortunately, development of that panel slowed down later. Finally, David Megginson \index{Megginson, David} decided to rebuild the panel code from scratch in January 2000. This led to a rapid addition of new instruments and features to the panel, resulting in nearly all main instruments being included until spring 2001. A handy minipanel was added in summer 2001.
\item Finally, \Index{LaRCsim}s \Index{Navion} was replaced as the default aircraft
 when the \Index{Cessna 172} was stable enough in February 2000 - a move most users will welcome. There are now several \Index{flight model} and airplane options to choose from at runtime. Jon Berndt\index{Berndt, Jon, S.} has invested a lot of time in a more
realistic and versatile flight model with a more powerful aircraft configuration method.
\JSBSim, as it has come to be called, did replace LaRCsim as the default
\Index{flight dynamics model} (\Index{FDM}), and it is planned to include such features as fuel slosh effects, turbulence, complete flight control systems, and other features not often found
all together in a flight simulator. As an alternative, Andy Ross\index{Ross, Andy} added another flight dynamics model called \YASim{} (Yet Another Flight Dynamics Simulator) which aims at simplicity of use and is based on fluid dynamics, by the end of 2001. This one bought us flight models for a 747, an A4, and a DC-3. Alternatively, a group around Michael Selig\index{Selig, Michael} from the \Index{UIUC} group provided another flight model along with several planes since around 2000.
\item A fully operational \Index{radio stack} and working radios were added to the panel by Curt Olson\index{Olson, Curt} in spring 2000. A huge database of Navaids contributed by Robin
Peel\index{Peel, Robin} allows IFR navigation since then. There was basic \Index{ATC} support added in fall 2001 by David Luff\index{Luff, David}. This is not yet fully implemented, but displaying \Index{ATIS messages} is already possible. A \Index{magneto switch} with proper functions was added at the end of 2001 by John Check\index{Check, John} and David Megginson\index{Megginson, David}. Moreover, several panels were continually improved during 2001 and 2002 by John and others. \FlightGear{} now allows flying ILS approaches and features a \Index{Bendix transponder}.
\item In 2002 functional \Index{multi-engine support} found its way into \FlightGear{}. \JSBSim{} is now the default FDM in \FlightGear{}.
\item Support of `'true'' \Index{3D panels} became stable via contributions from John Check\index{Check, John} and others in spring 2002. In addition, we got movable control surfaces\index{control surface, movable} like propellers etc., thanks to David Megginson.\index{Megginson, David}
\end{itemize}
}{}

\IfLanguageName{french}{
\subsection{A\'{e}ronefs}\index{histoire!a\'{e}ronefs}
\begin{itemize}
\item Un collimateur t\^{e}te haute (ou \Index{HUD}, \Index{head up display}) a \'{e}t\'{e} ajout\'{e} en se basant sur du code
 fourni par Michele America\index{America, Michele} et Charlie Hotch\-kiss\index{Hotchkiss, Charlie} \`{a} l'automne 1997 et a \'{e}t\'{e} par
 la suite am\'{e}lior\'{e} par Norman Vine. Bien que n'\'{e}tant pas g\'{e}n\'{e}ralement disponible pour le v\'{e}ritable
 \Index{Cessna 172}, le HUD rend compte de mani\`{e}re pratique de la performance en vol actuelle de la simulation et peut-\^{e}tre
 d'utilit\'{e} compl\'{e}mentaire sur des a\'{e}ronefs militaires par la suite.
 \item Un \Index{pilote automatique} rudimentaire impl\'{e}mentant le maintien de cap a \'{e}t\'{e} propos\'{e} par
 Jeff Goeke-Smith\index{Goeke-Smith, Jeff} en avril 1998. Il a \'{e}t\'{e} am\'{e}lior\'{e} par l'ajout d'un maintien d'altitude
 et d'un interrupteur de suivi de terrain en octobre 1998 et d\'{e}velopp\'{e} par la suite par Norman Vine\index{Vine, Norman}.
\item Friedemann Reinhard \index{Reinhard, Friedemann} a d\'{e}velopp\'{e} un code de prise en charge pour \Index{tableau d'instruments},
 qui a \'{e}t\'{e} ajout\'{e} en juin 1998. Malheureusement, le d\'{e}veloppement de ce panneau a ralenti par la suite. Enfin,
 David Megginson \index{Megginson, David} d\'{e}cida de reconstruire le code du paneau \`{a} partir de z\'{e}ro en janvier 2000.
 Ceci conduit \`{a} des ajouts rapides de nouveaux instruments et de nouvelles fonctionnalit\'{e}s au tableau, avec comme r\'{e}sultats
 l'ajout de pratiquement tous les principaux instruments jusqu'au printemps 2001. Un petit tableau bien pratique a \'{e}t\'{e} ajout\'{e}
 \`{a} l'\'{e}t\'{e} 2001.
\item Enfin, le \Index{Navion} de \Index{LaRCsim} fut remplac\'{e} comme avion par d\'{e}faut lorsque le \Index{Cessna 172} devint suffisamment
 stable en f\'{e}vrier 2000 - un changement que la plupart des utilisateurs appr\'{e}cia. Il y a dor\'{e}navant plusieurs \Index{mod\`{e}les de vol} et
 options d'a\'{e}ronefs qui peuvent \^{e}tre s\'{e}lectionn\'{e}s au d\'{e}marrage. Jon Berndt\index{Berndt, Jon, S.} a investi beaucoup de
 son temps dans un mod\`{e}le de vol plus r\'{e}alistique et verstatile, avec une m\'{e}thode de configuration des a\'{e}ronefs plus puissante.
\JSBSim, comme il fut appel\'{e} par la suite, rempla\c{c}a LaRCsim comme \Index{mod\`{e}le de dynamique de vol} (\Index{FDM}) par d\'{e}faut,
 et il est pr\'{e}vu d'ajouter des fonctionnalit\'{e}s comme les effets d\^{u}s de car\`{e}ne dus au carburant, la turbulence, des syst\`{e}mes complets de contr\^{o}le du vol,
 et d'autres fonctionnalit\'{e}s g\'{e}n\'{e}ralement non pr\'{e}sentes ensembles dans un simulateur de vol. Comme alternative,
 Andy Ross\index{Ross, Andy} ajouta un autre mod\`{e}le de dynamique de vol appel\'{e} \YASim{} (Yet Another Flight Dynamics Simulator, encore
 un autre simulateur de dynamique de vol), qui a pour but de pouvoir \^{e}tre utilis\'{e} faciliment, en s'appuyant sur la dynamique des fluides, fin 2001.
 Il nous permit d'obtenir des mod\`{e}les de vol pour un 747, un A4 et un DC-3. Alternativement, un groupe form\'{e} autour de
 Michael Selig\index{Selig, Michael} du group \Index{UIUC} fournit un autre mod\`{e}le de vol avec d'autres a\'{e}ronefs autour de l'ann\'{e}e 2000.
\item Une \Index{pile radio} compl\`{e}te et fonctionnalle fut ajout\'{e}e au tableau par Curt Olson\index{Olson, Curt} au printemps 2000. Une base de donn\'{e}es volumineuse comprenant
 les aides radio\'{e}lectriques pour la navigation fut fournie par Robin Peel\index{Peel, Robin} et permet, depuis, la navigation IFR.
 Une prise en charge basique du contr\^{o}le a\'{e}rien \Index{ATC} fut ajout\'{e}e \`{a} l'automne 2001 par David Luff\index{Luff, David}. Il n'est toujours pas compl\'{e}tement impl\'{e}ment\'{e},
 mais l'affichage des \Index{messages ATIS} est d\'{e}j\`{a} possible. Un \Index{commutateur magn\'{e}to} avec des fonctions qui lui sont
 propres a \'{e}t\'{e} ajout\'{e} \`{a} la fin de l'ann\'{e}e 2001 par John Check\index{Check, John} et David Megginson \index{Megginson, David}. De plus, plusieurs tableaux de bord ont \'{e}t\'{e}
 am\'{e}lior\'{e}s durant les ann\'{e}es 2001 et 2002 par John et d'autres. \FlightGear{} permet aujourd'hui d'effectuer des approches ILS et offre un \Index{transpondeur Bendix}.
\item En 2002, une \Index{prise en charge multimoteurs} fut \'{e}galement int\'{e}gr\'{e}e \`{a} \FlightGear{}. \JSBSim{} est dor\'{e}navant le FDM par d\'{e}faut dans \FlightGear{}.
\item La prise en charge de `'v\'{e}ritables'' \Index{tableaux de bord 3D} devint stable gr\^{a}ce aux contributions de John Check\index{Check, John} et d'autres au printemps 2002.
 En compl\'{e}ment, nous avons alors obtenu des surfaces de contr\^{o}le mobiles \index{surface de contr\^{o}le, mobile} comme les
 propulseurs, etc., gr\^{a}ce \`{a} David Megginson.\index{Megginson, David}
\end{itemize}
}{}
\fi
%%%%%%%%%%%%%%%%%%%%%%%%%Environment%%%%%%%%%%%%%%%%%%%%%%%%%%%%%%%%%%%%%%
\ifchinese
\subsection{环境}\index{历史!环境}
\begin{itemize}
\item 长期以来,太阳、月亮和星星的显示一直是 PC 飞行模拟器的弱点。\FlightGear{} 的一大成就是早期就包含了精确的太阳、月亮和星星的建模。相应的\Index{天文学代码}在 1997 年秋由 Durk Talsma\index{Talsma, Durk} 实现。
\item Christian Mayer\index{Mayer, Christian} 与 Durk Talsma,\index{Talsma, Durk} 一起于 1999 年冬贡献了天气相关代码。包括了\Index{云}、\Index{风},甚至还有\Index{暴风雨}。
\end{itemize}
\fi
\iffalse
\IfLanguageName{english}{
\subsection{Environment}\index{history!environment}
\begin{itemize}
\item The display of sun, moon and stars have been a weak point for PC flight simulators
 for a long time. It is one of the great achievements of \FlightGear{} to include accurate modeling and display of sun, moon, and planets very early. The corresponding \Index{astronomy code} was implemented in fall 1997 by Durk Talsma\index{Talsma, Durk}.
\item Christian Mayer, \index{Mayer, Christian}  together with Durk Talsma,\index{Talsma, Durk}
contributed weather code in the winter of 1999. This included \Index{clouds}, \Index{winds}, and even \Index{thunderstorms}.
\end{itemize}
}{}

\IfLanguageName{french}{
\subsection{Environnement}\index{histoire!environnement}
\begin{itemize}
\item L'affichage du soleil, de la lune et des \'{e}toiles a \'{e}t\'{e} pendant longtemps un point faible des simulateurs de vol pour PC.
Une des grandes r\'{e}ussites de \FlightGear{} a \'{e}t\'{e} d'inclure une mod\'{e}lisation et un affichage du soleil, de la lune
et des plan\`{e}tes de mani\`{e}re tr\`{e}s anticip\'{e}e. Le \Index{code astronomique} correspondant a \'{e}t\'{e} impl\'{e}ment\'{e}
\`{a} l'automne 1997 par Durk Talsma\index{Talsma, Durk}. \item Christian Mayer, \index{Mayer, Christian} avec Durk Talsma,\index{Talsma, Durk}
a contribu\'{e} au code m\'{e}t\'{e}orologique durant l'hiver 1999. Ce qui comprenait les \Index{nuages}, \Index{vents}, et m\^{e}me
les \Index{orages}.
\end{itemize}
}{}
\fi
%%%%%%%%%%%%%%%%%%%%%%%%%User Interface%%%%%%%%%%%%%%%%%%%%%%%%%%%%%%%%%%%%%%
\ifchinese
\subsection{用户界面}\index{历史!用户界面}
\begin{itemize}
\item 1998 年 6 月菜单系统\index{菜单}基于另一个库,可移动库\PLIB\index{PLIB} 开发而成。经历一段时间停滞以后,1999 年春,第一版可工作的菜单项问世。

后来 \PLIB{} 经过快速开发。Steve Baker\index{Baker, Steve} 在 1999 年春将其分离单独打包,脑中也有了更广泛的应用前景。从 1999 年秋开始提供给 \FlightGear{} 一个基本的图形渲染引擎。
\item 1998 年有了基本的\Index{音频支持},一个音频库和一些基本的发动机背景声音。后来这些被集合到了上文提到的 \PLIB\index{PLIB} 中。这个库在 1999 年 10 月扩展了对游戏杆、驾驶盘和脚踏板\index{游戏杆}的支持,为提升真实感迈出了巨大的一步。为了适应不同的游戏杆,2000 年秋引入了游戏杆配置选项。2002 年夏天 David Megginson\index{Megginson, David} 基于 xml 游戏杆文件开发了游戏杆自动检测\index{游戏杆/自动检测}特性。
\item 1999 年秋 Oliver Delise \index{Delise, Oliver} 和 Curt Olson\index{Olson, Curt} 投入时间开始开发网络/多人游戏\index{网络代码}\index{多人游戏代码}。目标是让 \FlightGear{} 可以通过\Index{网络}在多台机器上同时运行,无论是内部网还是\Index{互联网},可以和运行在另一个机器上的\Index{飞行计划程序}媾和在一起。2001 年出现了多种通过网络远程控制 \FlightGear{} 的方法。值得注意的是还有对“Atlas”\index{Atlas}的支持,也就是一个可移动的地图。另外,一个嵌入式的 \Index{HTTP 服务器}由 Curt Olson\index{Olson, Curt} 于 2001 年晚些时候开发出来,还能为外部程序提供\Index{属性管理器}。
\item 飞行模拟器里手动改变\Index{视角}在某种意义上说,总是“不真实”的,尽管如此当前状况也需要它。一个可能的解决方案在 1999 年冬天由 Norman Vine\index{Vine, Norman} 实现用鼠标改变视角。另外,现在你也可以用苦力帽达成这个目的。
\item 一个\Index{属性管理器}由 David Megginson\index{Megginson, David} 在 2000 年秋天开发出来。它可以使用一个配置文件,比如 UNIX/Linux 系统下的 \texttt{.fgfsrc}\index{.fgfsrc} 文件,Windows 系统下的 \texttt{system.fgfsrc}\index{system.fgfsrc} 文件。这个纯 ASCII 文件可以避免提供大量的输入选项,还有\Index{游戏杆配置}。2001 年春天时,游戏杆、键盘和仪表板的选项不再是硬编码,而是使用 *.xml 文件,这要感谢 David Megginson 和 John
Check\index{Check, John} 的贡献。
\end{itemize}

在开发过程中,大量代码被重新组织和整理。很多代码子系统分立打包,结果代码组织就是现在的样子:
\medskip

主图形引擎是 \textbf{\Index{OpenGL}},一个系统平台无关的库。基于 \Index{OpenGL} 的是可移动库 \PLIB{}\index{PLIB} 提供基本的渲染、音频、游戏杆等等程序。基于  \PLIB\index{PLIB} 的是 \SimGear{}\index{SimGear} 包括了 \FlightGear{} 和构建地景所需的所有基本程序。在 \SimGear{}\index{SimGear} 之上的是 \FlightGear{}\index{FlightGear}(模拟器本身),以及 \TerraGear{}\index{TerraGear} 包括了地景构建工具。

这不是一份详尽的历史,极有可能某些人做出重大贡献,而没有列在其中。除了上面所列的贡献者还有大量工作涉及内部结构:Jon S. Berndt\index{Berndt,Jon, S.}、Oliver Delise\index{Delise, Oliver}、Christian Mayer\index{Mayer, Christian}、Curt Olson\index{Olson, Curt}、Tony Peden\index{Peden, Tony}、Gary R. Van Sickle\index{van Sickle, Gary, R.}、Norman Vine\index{Vine, Norman} 等人。一份完整的贡献者列表可以在附录 ~\ref{landing} 里面以及源码目录的 \texttt{Thanks} 文件里找到。同样 \FlightGear{} 网站也包含了一份详细的历史值的一读,包括了所有值的注意的开发里程碑:
 \medskip

 \web{http://www.flightgear.org/version.html}
 
%%%%%%%%%%%%%%%%%%%%%%%%%%%%%%%%%%%%%%%%%%%%%%%%%%%%%%%%%%%%%%%%%%%%%%%%%%%%%%%%%%%%%%%%%%%%%%%
\section{那些贡献其中的人}\index{贡献者}

你是否享受飞行的乐趣呢?如果是这样的话,不要忘了那些人在这上面付出了数百小时。所有这些工作都是在业余时间志愿完成,如果不符合你的期待,不要责怪这些\Index{程序员}。请坐下来并给他们写一封邮件,讲述你的想法和改善的希望。或者,你可以订阅 \FlightGear{} 的\Index{邮件列表}并向里面贡献你的想法。相关介绍可以在这里找到:
 \medskip

 \web{http://www.flightgear.org/mail.html}.
  \medskip

\noindent
注意有两个列表,一个是由开发者维护另一个是由用户维护。其他一些主要用来发通知。
\medskip

 \noindent
以下这些是贡献者的名字及其贡献(信息从源码目录的 \texttt{Thanks} 文件提取出来的)
 \medskip
 
\noindent \textbf{A1 Free Sounds}\index{A1 Free Sounds}\\
授权 \FlightGear{} 项目使用其网站上的声音效果。主页在此:
   \medskip

   \href{http://www.a1freesoundeffects.com/}{http://www.a1freesoundeffects.com/}
   \medskip

\noindent \textbf{Syd Adams}\index{Adams, Syd}\\
  增加 2D 仪表,ATC 音量控制并创建了大量飞行器。
 \medskip

\noindent \textbf{Raul Alonzo}\index{Alonzo, Raul}\\
   Alonzo 先生是 Ssystem 的作者并授权使用月亮贴图。他的一些成为添加贴图的模板。
  Ssystem 的主页:
   \medskip
   
  \href{http://www1.las.es/~amil/ssystem/}{http://www1.las.es/\~{}amil/ssystem/}.
 \medskip
 
  \noindent \textbf{Michele America}\index{America, Michele}\\
  贡献 \Index{HUD} 相关代码。
 \medskip

\noindent \textbf{Michael Basler}\index{Basler, Michael}\\
  安装和使用入门的作者,飞行模拟主页:
  \medskip

 \web{http://www.geocities.com/pmb.geo/flusi.htm}
\medskip

\noindent \textbf{Jon S. Berndt}\index{Berndt, Jon, S.}\\
 用 C++ 重写/重构核心 \Index{FDM}。最开始他用 X15 来测试代码,开发完成后用其他飞机也可以。Jon 维护了一个网页处理飞行动态模型:
   \medskip

  \href{http://jsbsim.sourceforge.net/}{http://jsbsim.sourceforge.net/}
   \medskip

\noindent
  特别留意此网站的很多页面 X15 是付费的。另外,Jon 对此指南贡献了很多建议和修正。
\medskip

\noindent \textbf{Paul Bleisch}\index{Bleisch, Paul}\\
重构了调试系统以便更加灵活,这样在产品化的系统中可以很容易的禁用,也可以选择性的启用子系统的调试信息。同时他也贡献了第一个配置文件/命令行参数系统。
 \medskip

\noindent \textbf{Jim Brennan}\index{Brennan, Jim}\\
  为 \FlightGear{} $!$ 的美国地景提供了大量的在线存储空间。
 \medskip

\noindent \textbf{Bernie Bright}\index{Bright, Bernie}\\
很多 C++ 风格、使用和增强实现,STL 等很多很多方面。增加了多线程支持和线程平铺分页器。
 \medskip

\noindent \textbf{Stuart Buchanan}\index{Buchanan, Stuart}\\
更新此指南的多个部分,写了教程子系统的第一版,开发了随机植物和建筑物。
\medskip

\noindent \textbf{Bernhard H. Buckel}\index{Buckel, Bernhard}\\
贡献了 README.Linux。贡献了早期版本的安装和入门指南的多个部分。
 \medskip

\noindent \textbf{Gene Buckle}\index{Buckle, Gene}\\
大量工作让 \FlightGear{} 可以使用 \Index{MSVC}++ 编译器。也为细节提升做了很多提示。
 \medskip

\noindent \textbf{Ralph Carmichael}\index{Carmichael, Ralph}\\
支持此项目。公共领域航空软件网页:
\medskip

\href{http://www.pdas.com/}{http://www.pdas.com/}
 \medskip

 \noindent
 PDAS 销售的 CD-ROM 包含了很多航空工程师所需的程序。

\noindent \textbf{Didier Chauveau}\index{Chauveau, Didier}\\
  提供了解析 30 arcsec DEM 文件的基础代码
   \medskip

  \href{http://edcwww.cr.usgs.gov/landdaac/gtopo30/gtopo30.html}{http://edcwww.cr.usgs.gov/landdaac/gtopo30/gtopo30.html}.
 \medskip

\noindent \textbf{John Check}\index{Check, John}\\
 John 维护了基础包的 CVS 仓库。他贡献了云贴图,写了非常优秀的游戏杆文档和仪表板文档。同时,她也贡献了新仪表配置文件。他的 \FlightGear{} 主页:
 \medskip

 \href{http://www.rockfish.net/fg/}{http://www.rockfish.net/fg/}.
 \medskip

\noindent \textbf{Dave Cornish}\index{Cornish, Dave}\\
 Dave 创建了很酷的跑道贴图和云贴图。
 \medskip

\noindent \textbf{Oliver Delise} \index{Delise, Oliver}\\
开启了 FAQ、文档和公共关系。增加了一些网络/多人游戏代码\index{多人游戏代码}。是 FlightGear MuliPilot 的创始人。
\medskip

\noindent \textbf{Jean-Francois Doue}\index{Doue, Jean-Francois}\\
矢量 2D,3D,4D 以及矩阵 3D 和 4D 的 内联C ++类。(基于 Graphics Gems IV, Ed. Paul S. Heckbert)
  \medskip

\href{http://www.animats.com/simpleppp/ftp/public_html/topics/developers.html}{http://www.animats.com/simpleppp/ftp/public\_html/topics/developers.html}.
 \medskip

\noindent \textbf{Dave Eberly} \index{Eberly, Dave}\\
为 Christian Mayer 的天气数据库系统贡献了一些球面插值代码。
  \medskip

\noindent \textbf{Francine Evans}\index{Evans, Francine}\\
写了 GPL 协议的三正条纹。
  \medskip

\href{http://www.cs.sunysb.edu/~stripe/}{http://www.cs.sunysb.edu/\~{}stripe/}
\medskip

\noindent \textbf{Oscar Everitt}\index{Everitt, Oscar}\\
从 \Index{FS98} 游戏的 F4U 包里,创建了单发活塞发动机的发动机声音。它们非常棒而且 Oscar 也非常高兴可以贡献到我们的小项目中。
 \medskip

\noindent \textbf{Bruce Finney}\index{Finney, Bruce}\\
  贡献了一些补丁以支持 MSVC5。
 \medskip

\noindent \textbf{Olaf Flebbe}\index{Flebbe, Olaf}\\
增强了 Windows 下的构建系统并提供了预编译的依赖。
 \medskip

\noindent \textbf{Melchior Franz}\index{Franz, Melchior}\\
贡献了游戏杆苦力帽支持,LED 字体,增强的 telnet 和 HTTP 接口。值的注意的是还帮助发现了 \FlightGear{}、\SimGear{} 和 \JSBSim{} 的内存泄漏。
  \medskip

\noindent \textbf{Jean-loup Gailly}\index{Gailly, Jean-loup} 和 \textbf{Mark Adler}\index{Adler, Mark}\\
\Index{zilib 库}的作者。用于飞行时的压缩和解压缩程序。

  \href{http://www.gzip.org/zlib/}{http://www.gzip.org/zlib/}.
 \medskip

\noindent \textbf{Mohit Garg}\index{Garg, Mohit}\\
贡献了此指南。
 \medskip

\noindent \textbf{Thomas Gellekum}\index{Gellekum, Thomas}\\
修改并更新了 \Index{FreeBSD} 下的编译。
 \medskip

\noindent \textbf{Neetha Girish}\index{Girish, Neetha}\\
贡献了可用 xml 配置的 HUD。
 \medskip

\noindent \textbf{Jeff Goeke-Smith}\index{Goeke-Smith, Jeff}\\
贡献了我们第一个\Index{自动驾驶仪}(航向保持)。为外部时区/关照变量提供更好的 autoconf 检查。
 \medskip

\noindent \textbf{Michael I. Gold}\index{Gold, Michael, I.}\\
耐心地回答有关 \Index{OpenGL} 的问题。
 \medskip

\noindent \textbf{Habibe}\index{Habibe}\\
修改了 SimGear 的 RadHat 包构建。
 \medskip

\noindent \textbf{Mike Hill}\index{Hill, Mike}\\
响应我们的诉求使用其非常棒的飞机,可以从这里获得:
 \medskip

 \web{http://www.flightsimnetwork.com/mikehill/home.htm},

 \medskip

\noindent \textbf{Erik Hofman}\index{Hofman, Erik}\\
大幅修改和声音模块相关的参数,以便在运行时使用飞行器特由的声音配置模块。贡献了 SGI IRIX 支持(包括二进制文件)和一些很棒的贴图。
 \medskip

\noindent \textbf{Charlie Hotchkiss}\index{Hotchkiss, Charlie}\\
提升并增强了\Index{HUD}相关代码。大量代码风格和代码优化。
 \medskip

\noindent \textbf{Bruce Jackson}\index{Jackson, Bruce}(NASA)\\
受雇于 NASA 开发了 \Index{LaRCsim} 代码,提供给我们的飞行模型。Bruce 也回答了大量大量的问题。
  \medskip

\noindent \textbf{Maik Justus}\index{Justus, Maik} \\
增加了直升机支持,为 YASim FDM 提供起落架/地面交互以及滑翔机/铰链之间的交互。
  \medskip

\noindent \textbf{Ove Kaaven} \index{Kaaven, Ove}\\
贡献 Debian 二进制文件。
 \medskip

\noindent \textbf{Richard Kaszeta} \index{Kaszeta, Richard}\\
贡献了屏幕缓冲区到 ppm 截屏程序。也帮助了开发早期“高度保持和自动驾驶模块”\index{自动驾驶仪},教会 Curt Olson 基本控制理论并帮助他编码和调试早期版本。Curt 的“老板” Bob Hain 也贡献其中。更多详情:
 \medskip

  \href{http://www.menet.umn.edu/~curt/fgfs/Docs/Autopilot/AltitudeHold/AltitudeHold.html}{http://www.menet.umn.edu/\~{}curt/fgfs/Docs/Autopilot/AltitudeHold/AltitudeHold.html}.
  \medskip

\noindent
  Rich 的主页
  \medskip

  \href{http://www.kaszeta.org/rich/}{http://www.kaszeta.org/rich/}.
  \medskip

\noindent \textbf{Tom Knienieder}\index{Knienieder, Tom}\\
将\Index{音频库}首先移植到 OpenBSD 和 IRIX,然后又移植到 Win32。
 \medskip

\noindent \textbf{Reto Koradi}\index{Koradi, Reto}\\
帮助设置\Index{雾效果}
 \medskip

\noindent \textbf{Bob Kuehne}\index{Kuehne, Bob}\\
重构了 Makefile 系统使其更简单而可维护。
 \medskip

\noindent \textbf{Kyler B Laird}\index{Laird, Kyler B.}\\
贡献并修正了此指南
 \medskip

\noindent \textbf{David Luff}\index{Luff, David}\\
大量贡献到 IO360 活塞发动机建模。
 \medskip

\noindent \textbf{Sam van der Mac}\index{van der Mac, Sam}\\
贡献此指南帮助将 HTML 转成 \Latex{}。
 \medskip

\noindent \textbf{Christian Mayer}\index{Mayer, Christian}\\
为 fgfs 的技术演示贡献\Index{多国语言转换工具}。贡献了一些代码以便读取微软模拟飞行的地景贴图。Christian 正忙于全新的\Index{天气}子系统。为此项目贡献了\Index{热气球}建模。
 \medskip

\noindent \textbf{David Megginson}\index{Megginson, David}\\
贡献代码让鼠标输入可以控制视角。经济上贡献了硬盘空间为 FlightGear 项目使用。更新了 README.running 文件。贡献于无贴图的 fgfs 和 ssg。也增加了 2D 仪表和保存/载入支持。另外,它还开开发了全新的\Index{仪表板}代码,对 OpenGL 更友好,还有新的贴图。开发了属性管理器并贡献了游戏杆支持。贡献了生成随机地面物体。
 \medskip

 \medskip

\noindent \textbf{Cameron Moore}\index{Moore, Cameron}\\
FAQ 的维护者。现任的邮件列表管理员。提供 man 手册。
 \medskip

\noindent \textbf{Eric Mitchell}\index{Mitchell, Eric}\\
贡献了顶尖的地景\Index{贴图},所有原始的都是他创建的。
 \medskip

\noindent \textbf{Anders Morken}\index{Morken, Anders}\\
欧洲网站的前任维护者。
  \medskip

\noindent \textbf{Alan Murta}\index{Murta, Alan}\\
创建了多边形裁剪库。
 \medskip
 
  \web{http://www.cs.man.ac.uk/aig/staff/alan/software/}
   \medskip

\noindent \textbf{Phil Nelson}\index{Nelson, Phil}\\
GNU dbm 的作者,一套可以用来扩展散列的数据库程序,与标准 UNIX 下的 dbm 很相似。
  \medskip

\noindent \textbf{Alexei Novikov}\index{Novikov, Alexei}\\
创建了欧洲地景。贡献了一个脚本可以把 fgfs 地景转换成 2D 地图。写了第一版地景创建手册。
  \medskip

\noindent \textbf{Curt Olson}\index{Olson, Curt}\\
此项目的主要组织者。\\首先实现并修改了 \Index{LaRCsim}。\\除此之外,将分散各处的\Index{地景子系统}和其他图形资源组合在一起。其主页:

 \href{http://www.menet.umn.edu/~curt/}{http://www.menet.umn.edu/\~{}curt/}
 \medskip

\noindent \textbf{Brian Paul}\index{Paul, Brian}\\
我们使用了他的 TR 库,当然还有 Mesa:

 \web{http://www.mesa3d.org/brianp/TR.html}, \web{http://www.mesa3d.org}
 \medskip

\noindent \textbf{Tony Peden}\index{Peden, Tony}\\
贡献了飞行模型的开发,包括了基于 LaRCsim 的塞斯纳 172,贡献了 \JSBSim 的初始状态代码,更加完整的标准大气模型,以及其他 bug 修复和补充。
  \medskip

\noindent \textbf{Robin Peel}\index{Peel, Robin}\\
维护了 \FlightGear{} 和 X-Plane 的全球机场和跑道数据库。
 \medskip

\noindent \textbf{Alex Perry}\index{Perry, Alex}\\
贡献了更加精确的代码以便为 VSI、DG 和高度表建模。在邮件列表里建议和提高了模拟器的外观并帮助撰写文档。
 \medskip

\noindent \textbf{Friedemann Reinhard}\index{Reinhard, Friedemann}\\
开发了早期贴图的\Index{仪表板}。
 \medskip

\noindent \textbf{Petter Reinholdtsen}\index{Reinholdtsen, Petter}\\
并入了 GNU 的 automake/autoconf 系统(libtool)。为所有类 UNIX 平台简化和标准化了构建过程。对 IDE 的影响比较小,因为它不使用 UNIX make 系统
 \medskip

\noindent \textbf{William Riley}\index{Riley, William}\\
贡献代码增加了“\Index{制动}”。同时也写了第一版对超过两个轴的游戏杆支持。基于 VMap0 数据创建地景。
 \medskip

\noindent \textbf{Andy Ross}\index{Ross, Andy}\\
贡献了新的可配置的 FDM \YASim{}(Yet Another Flight Dynamics Simulator),基于几何信息,而不是空气动力系数。
 \medskip

\noindent \textbf{Paul Schlyter}\index{Schlyter, Paul}\\
提供了 Durk Talsma 及所有我们写天文学代码相关的信息。Schlyter 先生也很高兴回答天文学相关的问题。
  \medskip
  
 \href{http://www.welcome.to/pausch/}{http://www.welcome.to/pausch/}
 \medskip

\noindent \textbf{Chris Schoeneman}\index{Schoenemann, Chris}\\
贡献了音频支持。
 \medskip

\noindent \textbf{Phil Schubert}\index{Schubert, Phil}\\
贡献了各种贴图和发动机建模。
   \medskip

  \href{http://www.zedley.com/Philip/}{http://www.zedley.com/Philip/}.
  \medskip

\noindent \textbf{Jonathan R. Shewchuk}\index{Shewchuk, Jonathan}\\
\Index{Triangle 程序}的作者。Triangle 用来计算德劳内三角化和我们的不规则地形。
 \medskip

\noindent \textbf{Gordan Sikic}\index{Sikic, Gordan}\\
为 \Index{LaRCsim} 贡献了\Index{切诺基飞行模型}。现在已经无法工作并需要调试。使用配置选项 \texttt{-$ $-with-flight-model=cherokee} 来编译切诺基而不是\Index{塞斯纳}。
 \medskip

\noindent \textbf{Michael Smith}\index{Smith, Michael}\\
贡献了驾驶舱图形、3D 建模、商标和其他图片。Bonanza 项目。
   \medskip

\noindent \textbf{Martin Spott}\index{Spott, Martin}\\
此指南的联合作者。
  \medskip

\noindent \textbf{Jon Stockill}\index{Stockill, Jon}\\
维护了一个物体和其位置的数据库以便推广到全球地景。
\medskip

\noindent \textbf{Durk Talsma}\index{Talsma, Durk}\\
更加精确的太阳、月亮和行星。太阳可根据天空的位置修改颜色。月亮有正确的圆缺以及天空的位置。行星的位置做了调整。还帮助了时间函数、GUI 和其他一些。贡献了 2D 的云层\index{云}。网站:
   \medskip

 \href{http://people.a2000.nl/dtals/}{http://people.a2000.nl/dtals/}.
 \medskip

\noindent \textbf{UIUC}\index{UIUC} —— 航空宇航工程系\\
贡献了修改版的 LaRCsim 以便可以从文件载入飞行器参数。这些修改是结冰研究的一部分。
  \medskip

他们负责编码并让一切可以工作:\\
      Jeff Scott\\
      Bipin Sehgal\\
      Michael Selig
  \medskip

同时,他们帮助了此项目:\\
      Jay Thomas\\
      Eunice Lee\\
      Elizabeth Rendon\\
      Sudhi Uppuluri
  \medskip

\noindent
 \textbf{\Index{美国地质勘探局}}
  \medskip
提供了此项目使用的地质数据。
 \medskip

\href{http://edc.usgs.gov/geodata/}{http://edc.usgs.gov/geodata/}
 \medskip

\noindent \textbf{Mark Vallevand}\index{Vallevand, Mark}\\
贡献了一些解析 METAR 的代码,以及一些 win32 屏幕打印程序。
\medskip

\noindent \textbf{Gary R. Van Sickle}\index{van Sickle, Gary, R.}\\
贡献了一些早期的 \Index{GameGLUT} 支持和其他修正。他还帮助对二进制文件做解析,相关:
  \medskip

 \href{http://www.woodsoup.org/projs/ORKiD/fgfs.htm}{http://www.woodsoup.org/projs/ORKiD/fgfs.htm}.
 \medskip

\noindent
他的 \'Cygwin Tips\' 也许对你也有用
  \medskip

   \href{http://www.woodsoup.org/projs/ORKiD/cygwin.htm}{http://www.woodsoup.org/projs/ORKiD/cygwin.htm}.
  \medskip

\noindent \textbf{Norman Vine}\index{Vine, Norman}\\
为“FlightGear 社区”提供了大量网址。许多性能优化整个代码。对地景生成也贡献了大量代码。很多 Windows 相关的贡献。贡献了 wgs84 测距和航道程序。贡献了基于 wgs84 的圆形航路的自动驾驶仪模式。很多其他 GUI、HUD 和自动驾驶仪相关的贡献。贡献代码允许鼠标输入来控制视角的方向。拼接屏幕转储。贡献了初始的“Goto Airport”和“Reset”代码以及初始的 HTTP 图片服务器代码。
\medskip

\noindent \textbf{Roland Voegtli}\index{Voegtli, Roland}\\
贡献了图片级的贴图。他是 X-Plane 欧洲地景项目的创始人:
 \medskip

  \href{http://www.g-point.com/xpcity/esp/}{http://www.g-point.com/xpcity/esp/}
\medskip

\noindent \textbf{Carmelo Volpe}\index{Volpe, Carmelo}\\
移植 \FlightGear{} 到 \Index{Metro Works} 开发环境(PC/Mac)。
 \medskip

\noindent \textbf{Darrell Walisser}\index{Walisser, Darrell}\\
贡献了大量代码移植 \FlightGear{} 到 \Index{Metro Works} 开发环境(PC/Mac)。最后导致第一次苹果电脑移植,也贡献了入门指南的苹果电脑部分。
\medskip

\noindent \textbf{Ed Williams}\index{Williams, Ed}\\
贡献磁场相关代码(使用 Nima WMM 2000)。我们也从 Ed 那里多次借用了其航空定式。他的网站:
  \medskip
  \href{http://williams.best.vwh.net/}{http://williams.best.vwh.net/}.
 \medskip

\noindent \textbf{Jim Wilson}\index{Wilson, Jim}\\
主要修正了查看器代码,使其更加灵活和容易建模。贡献了很多小的修正和 Bug 报告。贡献到了 PUI 属性浏览器和自动驾驶仪。
 \medskip




%%%%%%%%%%%%%%%%%%%%%%%%%%%%%%%%%%%%%%%%%%%%%%%%%%%%%%%%%%%%%%%%%%%%%%%%%%%%%%%%%%%%%%%%%%%%%%%
\section{尚未完成的事业}
\subsection*{致谢}

%%%%%%%%%%%%%%%%%%%%%%%%%%%%%%%%%%%%%%%%%%%%%%%%%%%%%%%%%%%%%%%%%%%%%%%%%%%%%%%%%%%%%%%%%%%%%%%
\fi
%%%%%%%%%%%%%%%%%%%%%%%%%%%%%%%%%%%%%%%%%%%%%%%%%%%%%%%%%%%%%%%%%%%%%%%%%%%%%%%%%%%%%%%%%%%%%%%
\iffalse
\subsection{User Interface}\index{history!user interface}
\begin{itemize}
\item The foundation for a menu system\index{menu} was laid based on another library,
 the Portable Library \PLIB\index{PLIB}, in June 1998. After having been idle for a time, the first working menu entries came to life in spring 1999.

  \PLIB{} underwent rapid development later. It has been distributed as a separate package by
  Steve Baker\index{Baker, Steve} with a much broader range of applications in mind, since spring 1999. It  has provided the basic graphics rendering engine for \FlightGear{} since fall 1999.
\item In 1998 there was basic \Index{audio support}, i.\,e\. an audio library
and some basic background engine sound. This was later integrated into the
above-mentioned portable library, \PLIB\index{PLIB}. This same library was extended to
support joystick/yoke/rudder\index{joystick} in October 1999, again marking a huge step
in terms of realism. To adapt on different joystick, configuration options were
introduced in fall 2000. Joystick support was further improved by adding a self detection\index{joystick/self detection} feature based on xml joystick files, by David Megginson\index{Megginson, David} in summer 2002.
\item Networking/multiplayer\index{networking code}\index{multiplayer code}
 code has been integrated by Oliver Delise \index{Delise, Oliver} and Curt
Olson\index{Olson, Curt} starting fall 1999. This effort is aimed at enabling
\FlightGear{}  to run concurrently on several machines over a \Index{network}, either an Intranet or the \Index{Internet}, coupling it to a \Index{flight planner} running on a second
machine, and more. There emerged several approaches for remotely controlling \FlightGear{} over a Network during 2001. Notably there was added support for the ``Atlas''\index{Atlas} moving map program. Besides, an embedded \Index{HTTP server} developed by Curt Olson\index{Olson, Curt} late in 2001 can now act a \Index{property manager} for external programs.
\item Manually changing \Index{views} in a flight simulator is in a sense always ``unreal'' but
nonetheless required in certain situations. A possible solution was supplied by Norman
Vine\index{Vine, Norman} in the winter of 1999 by implementing code for changing views
using the mouse. Alternatively, you can use a hat switch for this purpose, today.
\item A \Index{property manager} was implemented by David Megginson\index{Megginson, David} in
fall 2000. It allows parsing a file called \texttt{.fgfsrc}\index{.fgfsrc} under
UNIX/Linux and \texttt{system.fgfsrc}\index{system.fgfsrc} under Windows for input
options. This plain ASCII file has proven useful in submitting the growing number of
input options, and notably the \Index{joystick settings}. This has shown to be a useful
concept, and joystick, keyboard, and panel settings are no longer hard coded but set
using *.xml files since spring 2001 thanks to work mainly by David Megginson and John
Check.\index{Check, John}
\end{itemize}
%%%%%%%%%%%%%%%%%%%%%%%%%%%%%%%%%% End List of Development %%%%%%%%%%%%%%%%%%%%%%%%%

During development there were several code reorganization efforts. Various code
subsystems were moved into packages. As a result, code is organized as follows at present:
\medskip

The base of the graphics engine is \textbf{\Index{OpenGL}}, a platform independent
graphics library. Based on \Index{OpenGL}, the Portable Library \PLIB{}\index{PLIB}
provides basic rendering, audio, joystick etc\. routines. Based on \PLIB\index{PLIB} is
\SimGear{}\index{SimGear}, which includes all of the basic routines required for the
flight simulator as well as for building scenery. On top of \SimGear{}\index{SimGear}
there are (i) \FlightGear{}\index{FlightGear} (the simulator itself), and (ii)
\TerraGear{}\index{TerraGear}, which comprises the scenery building tools.

This is by no means an exhaustive history and most likely some people who have made
important contributions have been left out. Besides the above-named contributions there
was a lot of work done concerning the internal structure by: Jon S. Berndt\index{Berndt,
Jon, S.}, Oliver Delise, \index{Delise, Oliver} Christian Mayer, \index{Mayer, Christian}
Curt Olson,\index{Olson, Curt} Tony Peden, \index{Peden, Tony} Gary R. Van
Sickle\index{van Sickle, Gary, R.}, Norman Vine\index{Vine, Norman}, and others. A more
comprehensive list of contributors can be found in Chapter~\ref{landing} as well as in
the \texttt{Thanks} file provided with the code. Also, the \FlightGear{}
Website\index{FlightGear Website} contains a detailed history worth reading of all of the
notable development milestones at
 \medskip

 \web{http://www.flightgear.org/version.html}

%%%%%%%%%%%%%%%%%%%%%%%%%%%%%%%%%%%%%%%%%%%%%%%%%%%%%%%%%%%%%%%%%%%%%%%%%%%%%%%%%%%%%%%%%%%%%%%
\section{Those, who did the work}\index{contributors}
%%%%%%%%%%%%%%%%%%%%%%%%%%%%%%%%%%%%%%%%%%%%%%%%%%%%%%%%%%%%%%%%%%%%%%%%%%%%%%%%%%%%%%%%%%%%%%%

Did you enjoy the flight? In case you did, don't forget those who devoted hundreds of
hours to that project. All of this work is done on a voluntary basis within spare time,
thus bare with the \Index{programmers} in case something does not work the way you want
it to. Instead, sit down and write them a kind (!) mail proposing what to change.
Alternatively, you can subscribe to the \FlightGear{} \Index{mailing lists} and
contribute your thoughts there. Instructions to do so can be found at
 \medskip

 \web{http://www.flightgear.org/mail.html}.
  \medskip

\noindent
 Essentially there are two lists, one of which being mainly for the developers
and the other one for end users. Besides, there is a very low-traffic list for
announcements.
\medskip

 \noindent
The following names the people who did the job (this information was essentially taken
from the file \texttt{Thanks} accompanying the code).
 \medskip

\noindent \textbf{A1 Free Sounds}\index{A1 Free Sounds}\\
   Granted permission for the \FlightGear{} project to use some of the sound effects from their
   site. Homepage under
   \medskip

   \href{http://www.a1freesoundeffects.com/}{http://www.a1freesoundeffects.com/}
   \medskip

\noindent \textbf{Syd Adams}\index{Adams, Syd}\\
  Added clipping for 2D instruments, ATC volume control and created a wide variety of aircraft.
 \medskip

\noindent \textbf{Raul Alonzo}\index{Alonzo, Raul}\\
   Mr. Alonzo is the
 author of Ssystem and provided his kind permission for using the moon texture.
 Parts of his code were used as a template when adding the texture.
  Ssystem Homepage can be found at:
   \medskip

  \href{http://www1.las.es/~amil/ssystem/}{http://www1.las.es/\~{}amil/ssystem/}.
 \medskip

 \noindent \textbf{Michele America}\index{America, Michele}\\
  Contributed to the \Index{HUD} code.
 \medskip

\noindent \textbf{Michael Basler}\index{Basler, Michael}\\
 Author of Installation and Getting Started. Flight Simulation Page at
  \medskip

 \web{http://www.geocities.com/pmb.geo/flusi.htm}
\medskip

\noindent \textbf{Jon S. Berndt}\index{Berndt, Jon, S.}\\
 Working on a complete C++ rewrite/reimplimentation of the core \Index{FDM}.
  Initially he is using X15 data to test his code, but once things are
  all in place we should be able to simulate arbitrary aircraft. Jon
  maintains a page dealing with Flight Dynamics at:
   \medskip

  \href{http://jsbsim.sourceforge.net/}{http://jsbsim.sourceforge.net/}
   \medskip

\noindent
  Special attention to X15 is paid in separate pages on this site. Besides, Jon
  contributed via a lot of suggestions/corrections to this Guide.
\medskip

\noindent \textbf{Paul Bleisch}\index{Bleisch, Paul}\\
  Redid the debug system so that it would be much more
  flexible, so it could be easily disabled for production system, and
  so that messages for certain subsystems could be selectively
  enabled. Also contributed a first stab at a config file/command line parsing
  system.
 \medskip


\noindent \textbf{Jim Brennan}\index{Brennan, Jim}\\
  Provided a big chunk of online space to store USA scenery for \FlightGear{}$!$.
 \medskip

\noindent \textbf{Bernie Bright}\index{Bright, Bernie}\\
  Many C++ style, usage, and implementation improvements, STL
  portability and much, much more.
  Added threading support and a threaded tile pager.
 \medskip

\noindent \textbf{Stuart Buchanan}\index{Buchanan, Stuart}\\
 Updated various parts of the manual, wrote the initial tutorial subsystem, developed
 random vegetation and buildings.
 \medskip

\noindent \textbf{Bernhard H. Buckel}\index{Buckel, Bernhard}\\
  Contributed the README.Linux.  Contributed several sections to earlier versions of
 Installation and Getting Started.
 \medskip

\noindent \textbf{Gene Buckle}\index{Buckle, Gene}\\
  A lot of work getting \FlightGear{} to compile with the \Index{MSVC}++
  compiler. Numerous hints on detailed improvements.
 \medskip


\noindent \textbf{Ralph Carmichael}\index{Carmichael, Ralph}\\
  Support of the project. The Public Domain Aeronautical Software web site at
\medskip

\href{http://www.pdas.com/}{http://www.pdas.com/}
 \medskip

 \noindent
 has the PDAS CD-ROM for sale containing great programs for astronautical engineers.

\noindent \textbf{Didier Chauveau}\index{Chauveau, Didier}\\
  Provided some initial code to parse the 30 arcsec DEM files found at:
   \medskip

  \href{http://edcwww.cr.usgs.gov/landdaac/gtopo30/gtopo30.html}{http://edcwww.cr.usgs.gov/landdaac/gtopo30/gtopo30.html}.
 \medskip

\noindent \textbf{John Check}\index{Check, John}\\
 John maintains the base package CVS repository. He contributed cloud textures, wrote an excellent Joystick Howto as well as a panel Howto. Moreover, he contributed new instrument panel configurations. \FlightGear{}
 page at
 \medskip

 \href{http://www.rockfish.net/fg/}{http://www.rockfish.net/fg/}.
 \medskip

\noindent \textbf{Dave Cornish}\index{Cornish, Dave}\\
 Dave created new cool runway textures plus some of our cloud textures.
 \medskip

\noindent \textbf{Oliver Delise} \index{Delise, Oliver}\\
 Started a FAQ, Documentation, Public relations. Working on adding some
  networking/multi-user code.\index{networking code} Founder of the FlightGear MultiPilot
\medskip

\noindent \textbf{Jean-Francois Doue}\index{Doue, Jean-Francois}\\
  Vector 2D, 3D, 4D and Matrix 3D and 4D inlined C++ classes.  (Based on
  Graphics Gems IV, Ed. Paul S. Heckbert)
  \medskip

\href{http://www.animats.com/simpleppp/ftp/public_html/topics/developers.html}{http://www.animats.com/simpleppp/ftp/public\_html/topics/developers.html}.
 \medskip

\noindent \textbf{Dave Eberly} \index{Eberly, Dave}\\
  Contributed some sphere interpolation code used by Christian Mayer's
  weather data base system.
  \medskip

\noindent \textbf{Francine Evans}\index{Evans, Francine}\\
  Wrote the GPL'd tri-striper we use.
  \medskip

\href{http://www.cs.sunysb.edu/~stripe/}{http://www.cs.sunysb.edu/\~{}stripe/}
\medskip

\noindent \textbf{Oscar Everitt}\index{Everitt, Oscar}\\
  Created single engine piston engine sounds as part of an F4U package
  for \Index{FS98}.  They are pretty cool and Oscar was happy to contribute
  them to our little project.
 \medskip

\noindent \textbf{Bruce Finney}\index{Finney, Bruce}\\
  Contributed patches for MSVC5 compatibility.
 \medskip

\noindent \textbf{Olaf Flebbe}\index{Flebbe, Olaf}\\
  Improved the build system for Windows and provided pre-built dependencies.
 \medskip

\noindent \textbf{Melchior Franz}\index{Franz, Melchior}\\
  Contributed joystick hat support, a LED font, improvements of the telnet and the http interface. Notable effort in hunting memory leaks in \FlightGear{}, \SimGear{}, and \JSBSim{}.
  \medskip

\noindent \textbf{Jean-loup Gailly}\index{Gailly, Jean-loup} and \textbf{Mark Adler}\index{Adler, Mark}\\
  Authors of the \Index{zlib library}.  Used for on-the-fly compression and
  decompression routines,

  \href{http://www.gzip.org/zlib/}{http://www.gzip.org/zlib/}.
 \medskip

\noindent \textbf{Mohit Garg}\index{Garg, Mohit}\\
 Contributed to the manual.
 \medskip

\noindent \textbf{Thomas Gellekum}\index{Gellekum, Thomas}\\
  Changes and updates for compiling on \Index{FreeBSD}.
 \medskip

\noindent \textbf{Neetha Girish}\index{Girish, Neetha}\\
  Contributed the changes for the xml configurable HUD.
 \medskip

\noindent \textbf{Jeff Goeke-Smith}\index{Goeke-Smith, Jeff}\\
  Contributed our first \Index{autopilot} (Heading Hold).
  Better autoconf check for external timezone/daylight variables.
 \medskip

\noindent \textbf{Michael I. Gold}\index{Gold, Michael, I.}\\
 Patiently answered questions on \Index{OpenGL}.
 \medskip

\noindent \textbf{Habibe}\index{Habibe}\\
 Made RedHat package building changes for SimGear.
 \medskip

\noindent \textbf{Mike Hill}\index{Hill, Mike}\\
 For allowing us to concert and use his wonderful planes, available form
 \medskip

 \web{http://www.flightsimnetwork.com/mikehill/home.htm},

 \noindent
 for \FlightGear{}.
 \medskip

\noindent \textbf{Erik Hofman}\index{Hofman, Erik}\\
  Major overhaul and parameterization of the sound module to allow
  aircraft-specific sound configuration at runtime.
  Contributed SGI IRIX support (including binaries) and some really great
  textures.
 \medskip

\noindent \textbf{Charlie Hotchkiss}\index{Hotchkiss, Charlie}\\
Worked on improving and enhancing the \Index{HUD} code.
Lots of code style tips and code tweaks.
 \medskip

\noindent \textbf{Bruce Jackson}\index{Jackson, Bruce} (NASA)\\
   Developed the \Index{LaRCsim} code under funding by NASA which we use to provide the
   flight model. Bruce has patiently answered many, many questions.
  \medskip

\noindent \textbf{Maik Justus}\index{Justus, Maik} \\
  Added helicopter support, gear/ground interaction and aerotow/winch support
  to the YASim FDM.
  \medskip

\noindent \textbf{Ove Kaaven} \index{Kaaven, Ove}\\
 Contributed the Debian binary.
 \medskip

\noindent \textbf{Richard Kaszeta} \index{Kaszeta, Richard}\\
  Contributed screen buffer to ppm screen shot routine.
  Also helped in the early development of the "altitude
  hold autopilot module"\index{autopilot} by teaching Curt Olson the basics of Control Theory
  and helping him code and debug early versions. Curt's \'Boss\' Bob Hain
  also contributed to that.  Further details available at:
 \medskip

  \href{http://www.menet.umn.edu/~curt/fgfs/Docs/Autopilot/AltitudeHold/AltitudeHold.html}{http://www.menet.umn.edu/\~{}curt/fgfs/Docs/Autopilot/AltitudeHold/AltitudeHold.html}.
  \medskip

\noindent
  Rich's Homepage is at
  \medskip

  \href{http://www.kaszeta.org/rich/}{http://www.kaszeta.org/rich/}.
  \medskip

\noindent \textbf{Tom Knienieder}\index{Knienieder, Tom}\\
  Ported the audio library\index{audio library} first to OpenBSD and IRIX and after that to Win32.
 \medskip

\noindent \textbf{Reto Koradi}\index{Koradi, Reto}\\
  Helped with setting up \Index{fog effects}.
 \medskip

\noindent \textbf{Bob Kuehne}\index{Kuehne, Bob}\\
  Redid the Makefile system so it is simpler and more robust.
 \medskip

\noindent \textbf{Kyler B Laird}\index{Laird, Kyler B.}\\
 Contributed corrections to the manual.
 \medskip

\noindent \textbf{David Luff}\index{Luff, David}\\
 Contributed heavily to the IO360 piston engine model.
 \medskip

\noindent \textbf{Sam van der Mac}\index{van der Mac, Sam}\\
 Contributed to The Manual by translating HTML tutorials to Latex.
 \medskip

\noindent \textbf{Christian Mayer}\index{Mayer, Christian}\\
 Working on \Index{multi-lingual conversion tools} for fgfs as a demonstration of technology.
 Contributed code to read Microsoft Flight Simulator scenery textures. Christian is working on a completely new \Index{weather} subsystem.
 Donated a \Index{hot air balloon} to the project.
 \medskip

\noindent \textbf{David Megginson}\index{Megginson, David}\\
  Contributed patches to allow mouse input to control view direction yoke.
  Contributed financially towards hard drive space for use by the
  flight gear project. Updates to README.running.
  Working on getting fgfs and ssg to work without textures.
  Also added the new 2-D panel and the save/load support.
  Further, he developed new \Index{panel} code, playing better with OpenGL, with new features.
  Developed the property manager and contributed to joystick support.
  Random ground cover objects
 \medskip

 \medskip

\noindent \textbf{Cameron Moore}\index{Moore, Cameron}\\
 FAQ maintainer. Reigning list administrator. Provided man pages.
 \medskip

\noindent \textbf{Eric Mitchell}\index{Mitchell, Eric}\\
  Contributed some topnotch scenery \Index{textures} being all original creations by him.
 \medskip

\noindent \textbf{Anders Morken}\index{Morken, Anders}\\
  Former maintainer of European web pages.
  \medskip

\noindent \textbf{Alan Murta}\index{Murta, Alan}\\
  Created the Generic Polygon Clipping library.
 \medskip

  \web{http://www.cs.man.ac.uk/aig/staff/alan/software/}
   \medskip

\noindent \textbf{Phil Nelson}\index{Nelson, Phil}\\
  Author of GNU dbm, a set of database routines that use extendible hashing and work
  similar to the standard UNIX dbm routines.
  \medskip

\noindent \textbf{Alexei Novikov}\index{Novikov, Alexei}\\
  Created European Scenery. Contributed a script to turn fgfs scenery into beautifully rendered
  2-D maps. Wrote a first draft of a Scenery Creation Howto.
  \medskip

\noindent \textbf{Curt Olson}\index{Olson, Curt}\\
 Primary organization of the project.\\
 First implementation and modifications based on \Index{LaRCsim}.\\
 Besides putting together all the pieces provided by others mainly concentrating on the \Index{scenery subsystem} as well as the graphics stuff. Homepage at

 \href{http://www.menet.umn.edu/~curt/}{http://www.menet.umn.edu/\~{}curt/}
 \medskip

\noindent \textbf{Brian Paul}\index{Paul, Brian}\\
 We made use of his TR library and of course of Mesa:

 \web{http://www.mesa3d.org/brianp/TR.html}, \web{http://www.mesa3d.org}
 \medskip

\noindent \textbf{Tony Peden}\index{Peden, Tony}\\
  Contributions on flight model development, including a LaRCsim based
  Cessna 172. Contributed to  {\JSBSim} the initial conditions code, a more complete
  standard atmosphere model, and other bugfixes/additions.
  \medskip


\noindent \textbf{Robin Peel}\index{Peel, Robin}\\
  Maintains worldwide airport and runway database for \FlightGear{} as well as X-Plane.
 \medskip

\noindent \textbf{Alex Perry}\index{Perry, Alex}\\
 Contributed code to more accurately model VSI, DG, Altitude.
 Suggestions for improvements of the layout of the simulator on the mailing list
 and help on documentation.
 \medskip

\noindent \textbf{Friedemann Reinhard}\index{Reinhard, Friedemann}\\
  Development of an early textured instrument \Index{panel}.
 \medskip

\noindent \textbf{Petter Reinholdtsen}\index{Reinholdtsen, Petter}\\
  Incorporated the GNU automake/autoconf system (with libtool).
  This should streamline and standardize the build process for all
  UNIX-like platforms.  It should have little effect on IDE type
  environments since they don't use the UNIX make system.
 \medskip

\noindent \textbf{William Riley}\index{Riley, William}\\
  Contributed code to add ''\Index{brakes}''. Also wrote a patch to support a first  joystick with more than 2 axis. Did the job to create scenery based on VMap0 data.
 \medskip

\noindent \textbf{Andy Ross}\index{Ross, Andy}\\
 Contributed a new configurable FDM called \YASim{} (Yet Another Flight Dynamics Simulator, based on geometry information rather than aerodynamic coefficients.
 \medskip

\noindent \textbf{Paul Schlyter}\index{Schlyter, Paul}\\
  Provided Durk Talsma with all the information he needed to write the
  astro code. Mr. Schlyter is also willing to answer astro-related questions
  whenever one needs to.
  \medskip

  \href{http://www.welcome.to/pausch/}{http://www.welcome.to/pausch/}
 \medskip

\noindent \textbf{Chris Schoeneman}\index{Schoenemann, Chris}\\
  Contributed ideas on audio support.
 \medskip

\noindent \textbf{Phil Schubert}\index{Schubert, Phil}\\
  Contributed various textures and engine modeling.
   \medskip

  \href{http://www.zedley.com/Philip/}{http://www.zedley.com/Philip/}.
  \medskip

 \noindent \textbf{Jonathan R. Shewchuk}\index{Shewchuk, Jonathan}\\
  Author of the Triangle\index{triangle program} program.  Triangle
  is used to calculate the  Delauney triangulation of our irregular terrain.
 \medskip

\noindent \textbf{Gordan Sikic}\index{Sikic, Gordan}\\
  Contributed a \Index{Cherokee flight model} for \Index{LaRCsim}.  Currently is not
  working and needs to be debugged.  Use configure
  \texttt{-$ $-with-flight-model=cherokee}
  to build the cherokee instead of the \Index{Cessna}.
 \medskip

\noindent \textbf{Michael Smith}\index{Smith, Michael}\\
  Contributed cockpit graphics, 3D models, logos, and other images.
  Project Bonanza
   \medskip

\noindent \textbf{Martin Spott}\index{Spott, Martin}\\
  Co-Author of The Manual.
  \medskip

\noindent \textbf{Jon Stockill}\index{Stockill, Jon}\\
  Maintains a database of objects and their location to populate the worldwide scenery.
\medskip

\noindent \textbf{Durk Talsma}\index{Talsma, Durk}\\
  Accurate Sun, Moon, and Planets.  Sun changes color based on
  position in sky. Moon has correct phase and blends well into the
  sky.  Planets are correctly positioned and have proper magnitude. Help with time
  functions, GUI, and other things. Contributed 2-D cloud layer.\index{clouds} Website
  at
   \medskip

 \href{http://people.a2000.nl/dtals/}{http://people.a2000.nl/dtals/}.
 \medskip

\noindent \textbf{UIUC}\index{UIUC} - Department of Aeronautical and Astronautical
Engineering\\
  Contributed modifications to LaRCsim to allow loading of aircraft
  parameters from a file.  These modifications were made as part of an
  icing research project.
  \medskip

  Those did the coding and made it all work:\\
      Jeff Scott\\
      Bipin Sehgal\\
      Michael Selig
  \medskip

  Moreover, those helped to support the effort:\\
      Jay Thomas\\
      Eunice Lee\\
      Elizabeth Rendon\\
      Sudhi Uppuluri
  \medskip


\noindent
 \textbf{\Index{U.\,S. Geological Survey}}
  \medskip

  Provided geographic data used by this project.
 \medskip

\href{http://edc.usgs.gov/geodata/}{http://edc.usgs.gov/geodata/}
 \medskip

\noindent \textbf{Mark Vallevand}\index{Vallevand, Mark}\\
  Contributed some METAR parsing code and some win32 screen printing routines.
\medskip

\noindent \textbf{Gary R. Van Sickle}\index{van Sickle, Gary, R.}\\
  Contributed some initial \Index{GameGLUT} support and other fixes. Has done
  preliminary work on a binary file format. Check
  \medskip

 \href{http://www.woodsoup.org/projs/ORKiD/fgfs.htm}{http://www.woodsoup.org/projs/ORKiD/fgfs.htm}.
 \medskip

\noindent
  His \'Cygwin Tips\' page might be helpful for you at
  \medskip



   \href{http://www.woodsoup.org/projs/ORKiD/cygwin.htm}{http://www.woodsoup.org/projs/ORKiD/cygwin.htm}.
  \medskip

\noindent \textbf{Norman Vine}\index{Vine, Norman}\\
  Provided more numerous URL's to the ``FlightGear Community''.
  Many performance optimizations throughout the code.  Many contributions
  and much advice for the scenery generation section.  Lots of Windows
  related contributions. Contributed wgs84 distance and course routines.
  Contributed a great circle route autopilot mode based on wgs84 routines.
  Many other GUI, HUD and autopilot contributions.  Patch to allow mouse input to control view direction. Ultra hires tiled screen dumps. Contributed the initial \'goto airport\' and \'reset\' functions and the initial http image server code
\medskip

\noindent \textbf{Roland Voegtli}\index{Voegtli, Roland}\\
 Contributed great photorealistic textures.   Founder of European Scenery Project for
 X-Plane:
 \medskip

  \href{http://www.g-point.com/xpcity/esp/}{http://www.g-point.com/xpcity/esp/}
\medskip


\noindent \textbf{Carmelo Volpe}\index{Volpe, Carmelo}\\
  Porting \FlightGear{} to the \Index{Metro Works} development environment
  (PC/Mac).
 \medskip

\noindent \textbf{Darrell Walisser}\index{Walisser, Darrell}\\
 Contributed a large number of changes to porting \FlightGear{} to the Metro Works development environment (PC/Mac). Finally produced the first Macintosh port. Contributed to the Mac part of Getting Started, too.
\medskip

\noindent \textbf{Ed Williams}\index{Williams, Ed}\\
  Contributed magnetic variation code (impliments Nima WMM 2000).
  We've also borrowed from Ed's wonderful aviation formulary at various
  times as well. Website at
  \medskip
  \href{http://williams.best.vwh.net/}{http://williams.best.vwh.net/}.
 \medskip


\noindent \textbf{Jim Wilson}\index{Wilson, Jim}\\
 Wrote a major overhaul of the viewer code to make it more flexible and modular. Contributed many small fixes and bug reports. Contributed to the PUI property browser and to the autopilot.
 \medskip

 \noindent \textbf{Jean-Claude Wippler}\index{Wippler, Jean-Claude}\\
  Author of \Index{MetaKit} - a portable, embeddible database with a
  portable data file format previously used in \FlightGear{}. Please
  see the following URL for more info:
 \medskip

  \href{http://www.equi4.com/metakit/}{http://www.equi4.com/metakit/}
  \medskip

\noindent \textbf{Woodsoup Project}\index{Woodsoup}\\

  While \FlightGear{} no longer uses Woodsoup servies we appreciate the
  support provided to our project during the time they hosted us. Once they
  provided computing resources and services so that the \FlightGear{} project
  could have a real home.

\href{http://www.woodsoup.org/}{http://www.woodsoup.org/}
  \medskip

\noindent \textbf{Robert Allan Zeh}\index{Zeh, Allan}\\
  Helped tremendously in figuring out the \Index{Cygnus} Win32 compiler and
  how to link with \.dll's.  Without him the first run-able Win32
  version of \FlightGear{} would have been impossible.
  \medskip

\noindent \textbf{Others}\index{Others}\index{Scenery Database}\\
The following individuals have contributed to the scenery object database:
   Jon Stockill, Martin Spott, Dave Martin, Thomas Foerster, Chris Metzler, Frederic
   Bouvier, Melchior Franz, Roberto Inzerillo, Erik Hofman, Mike Round,
   Innis Cunningham, David Megginson, Stuart Buchanan, Josh Babcock,
   Esa Hyytia, Mircea Lutic, Jens Thoms Toerring, Mark Akermann,
   Torsten Dreyer, Martin C. Doege, Alexis Bory, Sebastian Bechtold,
   Julien Pierru, Bertrand Augras, Gerard Robin, Jakub Skibinski,
   Morten Oesterlund Joergensen, Carsten Vogel, Dominique Lemesre,
   Daniel Leygnat, Bertrand Gilot, Morten Skyt Eriksen, Alex
   Bamesreiter, Oliver Predelli, Georg Vollnhals, and Paul Richter.


%%%%%%%%%%%%%%%%%%%%%%%%%%%%%%%%%%%%%%%%%%%%%%%%%%%%%%%%%%%%%%%%%%%%%%%%%%%%%%%%%%%%%%%%%%%%%%%
\section{What remains to be done}
%%%%%%%%%%%%%%%%%%%%%%%%%%%%%%%%%%%%%%%%%%%%%%%%%%%%%%%%%%%%%%%%%%%%%%%%%%%%%%%%%%%%%%%%%%%%%%%

If you read (and, maybe, followed) this guide up to this point you may probably agree: \FlightGear{}\, even in its present state, is not at all for the birds. It is
already a flight simulator which sports even several selectable flight models, several planes with panels and even a HUD, terrain scenery, texturing, all the basic controls and weather.

Despite, \FlightGear{} needs -- and gets -- further development. Except internal tweaks,
there are several fields where \FlightGear{} needs basics improvement and development. A
first direction is adding \Index{airport}s, buildings, and more of those things bringing
scenery to real life and belonging to realistic airports and cities. Another task is further
implementation of the \Index{menu system}, which should not be too hard with the basics
being working now. A lot of options at present set via command line or even during
compile time should finally make it into menu entries. Finally, \FlightGear{} lacks any
\Index{ATC} until now.

There are already people working in all of these directions. If you're a programmer and
think you can contribute, you are invited to do so.

%%%%%%%%%%%%%%%%%%%%%%%%%%%%%%%%%%%%%%%%%%%%%%%%%%%%%%%%%%%%%%%%%%%%%%%%%%%%%%%%%%%%%%%%%%%%%%%
\subsection*{Achnowledgements}
%%%%%%%%%%%%%%%%%%%%%%%%%%%%%%%%%%%%%%%%%%%%%%%%%%%%%%%%%%%%%%%%%%%%%%%%%%%%%%%%%%%%%%%%%%%%%%%
Obviously this document could not have been written without all those contributors
mentioned above making \FlightGear{} a reality.

First, I was very glad to see Martin Spott \index{Spott, Martin} entering
the documentation effort. Martin provided not only several updates and
contributions (notably in the OpenGL section) on the Linux side of the
project but also several general ideas on the documentation in general.

Besides, I would like to say special thanks to Curt Olson,\index{Olson, Curt} whose
numerous scattered Readmes, Thanks, Webpages, and personal eMails were of special help to
me and were freely exploited in the making of this booklet.

Next, Bernhard Buckel \index{Buckel, Bernhard} wrote several sections of early versions
of that Guide and contributed at lot of ideas to it.

Jon S. Berndt \index{Berndt, Jon, S.} supported me by critical proofreading of several
versions of the document, pointing out inconsistences and suggesting improvements.

Moreover, I gained a lot of help and support from Norman Vine\index{Vine, Norman}. Maybe,
without Norman's answers I would have never been able to tame different versions of the
\Cygwin{} -- \FlightGear{} couple.

We were glad, our Mac expert Darrell Walisser \index{Walisser, Darrell} contributed the section on compiling under Mac OS X. In addition he submitted several Mac related hints and fixes.

Further contributions and donations on special points came from John Check,\index{Check,
John} (general layout), Oliver Delise \index{Delise, Oliver} (several suggestions
including notes on that chapter), Mohit Garg \index{Garg, Mohit} (OpenGL), Kyler B. Laird
\index{Laird, Kyler B.} (corrections), Alex Perry\index{Perry, Alex} (OpenGL), Kai
Troester\index{Troester, Kai} (compile problems), Dave Perry \index{Perry, Dave} (joystick support), and Michael Selig\index{Selig, Michael} (UIUC models).

Besides those whose names got lost withing the last-minute-trouble we'd
like to express our gratitude to the following people for contributing
valuable `bug fixes' to this version of The FlightGear Manual (in
random order): Cameron Moore,\index{Moore, Cameron} Melchior
Franz,\index{Franz, Melchior} David Megginson,\index{Megginson, David}
Jon Berndt,\index{Berndt, Jon} Alex Perry,\index{Perry, Alex}, Dave
Perry,\index{Perry, Dave} Andy Ross,\index{Ross, Andy} Erik
Hofman\index{Hofman, Erik}, and Julian Foad\index{Foad, Julian}.

\fi

%% Revision 0.00  1998/09/08  michael
%% Initial revision for version 0.53.
%% Revision 0.01  1998/09/20  michael
%% several extensions and corrections
%% revision 0.10  1998/10/01  michael
%% final proofreading for release
%% revision 0.11  1998/11/01  michael
%% corrections on audio library, getting started
%% revision 0.12  1999/03/07  michael
%% Updated Credits
%% revision 0.20  1999/06/04  michael
%% added O. Delise, Ch. Mayer, R. Peel, R. Voegtli, several updates
%% revision 0.3 2000/03/01 michael
%% Supplemented to Jon Berndt, Oliver Delise, Christian Mayer, Durk Talsma
%% Norman Vine
%% Added David Meggison
%% Revised Acknowledgements
%% Several additions suggested and collected by Oliver Delise
%% revision 0.4 2001/05/12 michael
%% added several people entering the game during the last year
%% changed countless mail addresses
%% revision 0.5 2002/01/01 michael
%% update on history during the last year
%% added new people entering the game
%% revision 0.6 2002/09/09 michael
%% reorganized history section for better readability
%% added progress for versions 0.7.10 & 0.8
%% removed outdated Navion picture
